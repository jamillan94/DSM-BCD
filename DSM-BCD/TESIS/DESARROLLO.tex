\chapter{Metodología DSM-BCD}

Dado los problemas presentados de los proyectos basados en datos, y el estudio de diversas metodologías propuestas por varios autores, se propone la metodología \textit{DSM-BCD (Data Science Methodology for Breast Cancer Diagnosis)}. En la figura \ref{DSM-BCD} se puede visualizar las fases de la metodología DSM-BCD.

\begin{figure*}[!htb]
	\centering
	\includegraphics[width=0.9
	\linewidth]{IMAGENES/DSM-BCD_SPANISH.pdf}
	\caption{Metodología en ciencia de datos para el diagnóstico del cáncer de mama 
		(DSM-BCD)}
	\label{DSM-BCD}
\end{figure*}

Esta metodología tiene como base el \textit{manifestó ágil} aplicado a un contexto de resultados basados en datos. Dado lo anterior \textit{DSM-BCD} no se enfoca en evaluar la precisión de las técnicas de ML y DL sino su objetivo principal es generar valor a los datos en el tiempo mas corto posible para que los médicos diagnostiquen de manera ágil el cáncer de mama. Para lograrlo \textit{DSM-BCD} integra la perspicacia medica y los resultados obtenidos por las técnicas de ML y DL en una retroalimentación continua generada en cada \textit{Release} para producir mayor eficacia en la toma de decisiones

 En \textit{DSM-BCD}, se propone la conformación de un \textit{Data Analysis Team}. Este equipo debe estar encabezado por el medico experto en oncología, al menos un ingeniero de datos y un científico de datos. Se recomienda que el equipo este conformado por un máximo de 5 personas para facilitar el trabajo en equipo y la comunicación interna.
 
 Para comprender mejor el uso de DSM-BCD, se realizó un análisis descriptivo basados en datos genéticos característicos de tumores generados por los tipos de cáncer \textit{Carcinoma ductal invasivo (IDC)} y \textit{Carcinoma lobulillar invasivo (LBC)}, en donde se logró determinar que el IDC y LBC son enfermedades molecularmente distintas con rasgos genéticos característicos, lo que proporciona información importante para la estratificación de los pacientes permitiendo realizar diagnostico clínico ágil con precedentes para un tratamiento puntual.
 
 \section{Fase 1: Mapa de preguntas sobre el cáncer de mama (BCQM)} 
En esta fase se propone el uso de un mapa de preguntas sobre el cáncer de mama (BCQM). El proposito del \textit{BCQM} es que el \textit{Data Analysis Team} defina las preguntas que serán resueltas al finalizar cada \textit{Release} y que permitirán tomar decisiones medicas con respecto al diagnostico de esta enfermedad. En la figura \ref{BCQM} se observa la estructura del BCQM.

\begin{figure}
	\centering
	\includegraphics[width=0.6
	\linewidth]{IMAGENES/BCQM}
	\caption{Mapa de preguntas basados en los tipos de cáncer de mama \textit{inflamatorio (IBC)}, \textit{mucinoso (MBC)}, \textit{Lobulillar (LBC)}, \textit{Tumores mixtos (MTBC)}, \textit{Carcinoma ductal invasivo (IDC)} y \textit{Carcinoma ductal in situ (DCIS)}}
	\label{BCQM}
\end{figure}

El BCQM permite plantear preguntas relacionadas a los tipos de cáncer de mama y a las técnicas para el diagnostico de la misma. De modo que al finalizar el tiempo de cada \textit{Release}, el cual puede variar entre 1 y 4 semanas, las preguntas serán respondidas según el análisis de datos generado, y el medico podrá tomar una decisión de valor. Cabe resaltar, que es posible tener una o mas preguntas relacionadas a una técnica y a un tipo de cáncer de mama por cada \textit{Release}, razón por la cual es posible encontrar correlaciones entre las variables características de cada tipo de cáncer encontrando así patrones ocultos en los diferentes conjuntos de datos. Adicionalmente, el BCQM permite identificar a que técnica para el diagnostico de cáncer mama esta relacionada la pregunta a resolver, lo cual de antemano hace posible conocer el tipo información (imágenes o datos) y el algoritmo  ML o DL requerido para dar solución al problema. Así mismo, el BCQM permite definir desde la fase inicial el tipo de modelo predictivo o descriptivo según el enfoque analítico generado por la pregunta planteada. Sintetizando, el uso de BCQM facilita la comprensión del problema medico y permite identificar previamente la técnica, el tipo de información y enfoque que debe ser utilizado para el análisis de datos.  
 \section{Fase 2: Planeación de actividades}
En esta fase el \textit{Data Analysis Team} basado en las preguntas realizadas en el BCQM analiza todas las tareas que hay que llevar a cabo, las estiman en tiempo y las distribuyen entre las personas que las van a realizar durante el \textit{Release}. Dado que el BCQM nos permite conocer de antemano el tipo de cáncer de mama y la técnica para el diagnostico de esta enfermedad, el científico de datos con ayuda del medico puede definir el origen de datos, lo cual va a permitir conocer el tipo, cantidad y peso de la información. Dado lo anterior, es recomendable que el equipo tenga al menos un ingeniero datos, ya que el es el encargado de tomar los datos y convertirlos en información significativa para que el científico pueda realizar el respectivo análisis.  

\begin{figure}[!htb]
	\centering
	\includegraphics[width=0.36
	\linewidth]{IMAGENES/Data_Analysis_Team}
	\caption{Conformación del Data Analysis Team. }
	\label{Data_Analysis_Team}
\end{figure}

\begin{figure}[!htb]
	\centering
	\includegraphics[width=0.68
	\linewidth]{IMAGENES/Activity_Planning}
	\caption{Planeación de actividades realizada por el Data Analysis Team. }
	\label{Activity_Planning}
\end{figure}

 Una vez generadas las preguntas en el BCQM es posible que varias actividades este relacionadas a varias preguntas, por lo tanto se recomienda al \textit{Data Analysis Team} agrupar en una sola actividad las tareas a realizar para generar una mayor agilidad en la elaboración de interpretaciones y respuestas. Así mismo todas las acciones, propuestas y actividades deben estar orientadas en generar respuestas a las preguntas planteadas, siendo el principal objetivo que el experto en oncologia tome una decisión de valor. Dado lo anterior, durante el Release siempre debe existir una comunicación continua y efectiva entre el equipo técnico y el experto en oncologia. Finalmente, la planeación de actividades debe durar máximo 8 horas y su creación debe estar de acuerdo a las habilidades técnicas de cada participante del equipo.
 
Para un mayor entendimiento, basados en las 3 preguntas planteadas en el BCQM, se genero la siguiente planeación de actividades para generar respuestas basados en los datos de índole genómico responsables del desarrollo del cáncer de mama:
\begin{figure}[!htb]
	\centering
	\includegraphics[width=
	\linewidth]{IMAGENES/Planning_TCGA}
	\caption{Planeación de actividades realizada por el Data Analysis Team. }
	\label{Activity_Planning_TCGA}
\end{figure}

 \section{Fase 3: Adquisición de datos oncológicos}
En esta fase, con base a las tareas realizadas en la planeación de actividades, el medico experto en oncología junto con el ingeniero y el científico de datos identifican y reúnen los recursos de datos disponibles (estructurados, no estructurados y semiestructurados) y relevantes para solucionar las preguntas planteadas en el \textit{BCQM}. Cabe resaltar, que en la metodología \textit{\textit{DSM-BCD}} es factible tener varios conjuntos de datos o imágenes que están relacionados a un tipo de cáncer de mama y una técnica de diagnóstico, por lo tanto  el \textit{Data Analysis Team} puede tener a varios científicos respondiendo preguntas diferentes en un mismo \textit{Release}. Como consecuencia, al final se pueden obtener como resultado múltiples respuestas y una posible correlación entre las diversas variables oncológicas.  Asimismo, en esta fase el \textit{Data Analysis Team} debe definir la infraestructura de datos necesaria según la cantidad de información a procesar, lo cual permitirá proyectar la escalabilidad, alcance y distribución de dicha información. 

Para este caso de estudio, se utilizaron variables genéticas características de marcadores tumorales basados en los tipos de cáncer de mama \textit{Carcinoma ductal invasivo (IDC)} y \textit{Carcinoma lobulillar invasivo (ILC)}. Estas variables fueron obtenidas del conjunto de datos denominado \textit{“Breast Invasive Carcinoma (TCGA, Cell 2015)”}. Los datos fueron descargados del sitio público \textit{cBioPortal} para la genómica del cáncer (\url{https://www.cbioportal.org/study/summary?id=brca_tcga_pub2015}). 

Cabe resaltar, que el conjunto de datos contiene un total de 817 muestras de tumores de mama que se perfilaron con 6 plataformas moleculares: Análisis del número de copias somáticas basado en array, Secuenciación del exoma completo, perfil de metilación del ADN basado en array, secuenciación del ARN mensajero, Secuenciación de microARN (miARN) y Array de proteínas en fase inversa (RPPA), como se ha descrito previamente en \cite{Bass2014}. Un comité de patología revisó y clasificó todos los tumores en 490 IDC, 127 ILC, 88 casos con características mixtas de IDC y ILC, y 112 con otras histologías. Este conjunto de datos consta de un tamaño de $818$ filas y $110$ columnas. Las variables se describen en la tabla \ref{brca_tcga_pub2015_clinical_data}. 

\begin{table*} [!htb]
	\footnotesize
	\begin{threeparttable}
		\caption{Conjunto de datos del Carcinoma invasivo de mama (TCGA, Cell 2015).}
		\label{brca_tcga_pub2015_clinical_data}
		\begin{tabular}{p{1cm} p{4cm} p{10cm}} \toprule	
			\begin{center}$N$\end{center}   
			&\begin{center}Variable\end{center}             
			&\begin{center}Descripción\end{center}      
			\\ \hline	1	&	Study ID	&	Código de identificación del estudio
			\\ \hline	2	&	Patient ID	&	Código de identificación del paciente
			\\ \hline	3	&	Sample ID	&	Código de identificación de la muestra
			\\ \hline	4	&	Diagnosis Age	&	Edad a la que se diagnosticó por primera vez una afección o enfermedad
			\\ \hline	5	&	American Joint Committee on Cancer Metastasis Stage Code	&	Código para representar la ausencia o presencia definida de diseminación a distancia o metástasis (M) a localizaciones a través de canales vasculares o linfáticos más allá de los ganglios linfáticos regionales, utilizando los criterios establecidos por el Comité Conjunto Americano del Cáncer (AJCC)
			\\ \hline	6	&	Neoplasm Disease Lymph Node Stage American Joint Committee on Cancer Code	&	Los códigos que representan el estadio del cáncer en función de los ganglios presentes (estadio N) según criterios basados en múltiples ediciones del Manual de Estadificación del Cáncer de la AJCC
			\\ \hline	7	&	Neoplasm Disease Stage American Joint Committee on Cancer Code	&	Estadio de la extensión de un cáncer, especialmente si la enfermedad se ha propagado desde el sitio original a otras partes del cuerpo según los criterios de estadificación del AJCC
			\\ \hline	8	&	American Joint Committee on Cancer Publication Version Type	&	Versión o edición del American Joint Committee on Cancer Cancer Staging Handbooks, publicación del grupo formado con el propósito de desarrollar un sistema de estadificación clínica del cáncer que sea aceptable para la profesión médica estadounidense y compatible con otras clasificaciones aceptadas
			\\ \hline	9	&	American Joint Committee on Cancer Tumor Stage Code	&	Código de T patológico (tumor primario) para definir el tamaño o la extensión contigua del tumor primario (T), utilizando los criterios de estadificación del AJCC
			\\ \hline	10	&	Brachytherapy first reference point administered total dose	&	Primer punto de referencia dosis total administrada en la Braquiterapia
			\\ \hline	11	&	Cancer Type	&	Tipo de cáncer
			\\ \hline	12	&	Cancer Type Detailed	&	Detalle del tipo de cáncer
			\\ \hline	13	&	Cent17 Copy Number	&	Intervalo de resultados de la señal del cromosoma 17 del procedimiento de diagnóstico de hibridación in situ con fluorescencia.
			\\ \hline	14	&	Birth from Initial Pathologic Diagnosis Date	&	Intervalo de tiempo desde la fecha de nacimiento de una persona hasta la fecha del diagnóstico patológico inicial, representado como un número calculado de días.
			\\ \hline	15	&	Days to Sample Collection.	&	Días para la recolección de muestras
			\\ \hline	16	&	Death from Initial Pathologic Diagnosis Date	&	Intervalo de tiempo desde la fecha de muerte de una persona hasta la fecha del diagnóstico patológico inicial, representado como un número calculado de días.
			\\ \hline	17	&	Last Alive Less Initial Pathologic Diagnosis Date Calculated Day Value	&	Intervalo de tiempo desde el último día en que se sabe que una persona está viva hasta la fecha del diagnóstico patológico inicial, representado como un número calculado de días.
			\\ \hline	18	&	Days to Last Followup	&	Intervalo de tiempo desde la fecha del último seguimiento hasta la fecha del diagnóstico patológico inicial, representado como un número calculado de días.
			\\ \hline	19	&	Disease Free (Months)	&	Sin enfermedad (meses)
			\\ \hline	20	&	Disease Free Status	&	Estado libre de enfermedad
			\\ \hline	21	&	Disease code	&	Código de la enfermedad
			\\ \hline	22	&	ER positivity scale other	&	Escala de medición de otro receptor de estrógeno de hallazgo positivo
			\\ \hline	23	&	ER positivity scale used	&	Escala de hallazgo ER positivo de inmunohistoquímica de carcinoma de mama
			\\ \hline	24	&	ER Status By IHC	&	Estado del receptor de progesterona del carcinoma de mama
			\\ \hline
		\end{tabular}
	\end{threeparttable}
\end{table*}

\begin{table*} [!htb]
	\footnotesize
	\begin{threeparttable}
		\begin{tabular}{p{1cm} p{4cm} p{10cm}}         
			\\ \hline	25	&	ER Status IHC Percent Positive	&	Nivel de receptor de estrogeno con categoría de porcentaje celular
			\\ \hline	26	&	Ethnicity Category	&	Información sobre el origen étnico.
			\\ \hline	27	&	First surgical procedure other	&	Propósito del procedimiento quirúrgico
			\\ \hline	28	&	Form completion date	&	Fecha de finalización del formulario
			\\ \hline	29	&	Fraction Genome Altered	&	Fracción de genoma alterado
			\\ \hline	30	&	HER2 and cent17 cells count	&	HER2 neu y centrómero 17 número de copia análisis entrada total número recuento
			\\ \hline	31	&	HER2 and cent17 scale other	&	HER2 y centrómero 17 resultados positivos otra escala de medición
			\\ \hline	32	&	HER2 cent17 ratio	&	Valor de la relación de señal del cromosoma 17 de HER2 neu
			\\ \hline	33	&	HER2 copy number	&	Número total de entrada de análisis de copia de carcinoma de mama HER2 neu
			\\ \hline	34	&	HER2 fish method	&	Método de cálculo de hibridación in situ de fluorescencia de HER2 erbb pos
			\\ \hline	35	&	HER2 fish status	&	Procedimiento de laboratorio tipo de resultado híbrido in situ HER2 neu
			\\ \hline	36	&	HER2 ihc percent positive	& Porcentaje de HER2 ihc positivo
			\\ \hline	37	&	HER2 ihc score	&	Resultado del nivel de inmunohistoquímica HER2
			\\ \hline	38	&	HER2 positivity method text	&	Método de positividad HER2
			\\ \hline	39	&	HER2 positivity scale other	&	Otra medida escala pos hallazgo HER2 erbb2 
			\\ \hline	40	&	Neoplasm Histologic Type Name	&	Término que designa el patrón estructural de las células cancerosas utilizado para definir un diagnóstico microscópico.
			\\ \hline	41	&	Tumor Other Histologic Subtype	&	Subtipo histológico de un tumor o el diagnóstico mixto que es diferente de las opciones especificadas anteriormente.
			\\ \hline	42	&	Neoadjuvant Therapy Type Administered Prior To Resection Text	&	Término para describir el historial de tratamiento neoadyuvante del paciente y el tipo de tratamiento administrado antes de la resección del tumor.
			\\ \hline	43	&	Prior Cancer Diagnosis Occurrence	&	Término para describir los antecedentes de diagnóstico previo de cáncer del paciente y la ubicación espacial de cualquier aparición previa de cáncer.
			\\ \hline	44	&	ICD-10 Classification	&	Decima revisión de la Clasificación Estadística Internacional de Enfermedades y Problemas Relacionados con la Salud.
			\\ \hline	45	&	International Classification of Diseases for Oncology, Third Edition ICD-O-3 Histology Code	&	Tercera edición de la Clasificación Internacional de Enfermedades Oncológicas, publicada en 2000, utilizada principalmente en los registros de tumores y cáncer para codificar la localización (topografía) y la histología (morfología) de las neoplasias. Estudio de la estructura de las células y su disposición para constituir tejidos y, finalmente, la asociación entre éstos para formar órganos. En patología, proceso microscópico de identificación de las características morfológicas normales y anormales de los tejidos mediante el empleo de diversas tinciones citoquímicas e inmunocitoquímicas.
			\\ \hline	46	&	International Classification of Diseases for Oncology, Third Edition ICD-O-3 Site Code	&	Tercera edición de la Clasificación Internacional de Enfermedades Oncológicas, publicada en 2000, utilizada principalmente en los registros de tumores y cáncer para codificar la localización (topografía) y la histología (morfología) de las neoplasias. Sistema de categorías numeradas para la representación de datos.
			\\ \hline	47	&	IHC-HER2	&	Término que designa el estado de la prueba IHC-HER2
			\\ \hline	48	&	IHC Score	&	Puntuación IHC
			\\ \hline	49	&	Informed consent verified	&	Consentimiento informado verificado
			\\ \hline	50	&	Year Cancer Initial Diagnosis	&	Año del diagnóstico patológico inicial de cáncer de un individuo
			\\ \hline	51	&	Is FFPE	&	Si la muestra es de tejido fijado con formalina e incrustado en parafina (FFPE)
			\\ \hline
		\end{tabular}
	\end{threeparttable}
\end{table*}

\begin{table*} [!htb]
	\footnotesize
	\begin{threeparttable}
		\begin{tabular}{p{1cm} p{4cm} p{10cm}}
			\\ \hline	52	&	Primary Lymph Node Presentation Assessment Ind-3	&	Término que indica si se realizó una evaluación de los ganglios linfáticos en la presentación primaria de la enfermedad.
			\\ \hline	53	&	Positive Finding Lymph Node Hematoxylin and Eosin Staining Microscopy Count	&	Recuento de ganglios linfáticos positivos identificados mediante microscopía óptica con tinción de hematoxilina y eosina (H\&E).
			\\ \hline	54	&	Positive Finding Lymph Node Keratin Immunohistochemistry Staining Method Count	&	Recuento de ganglios linfáticos positivos identificados a través del método de tinción de inmunohistoquímica (IHC) de queratina
			\\ \hline	55	&	Lymph Node(s) Examined Number	&	Número total de ganglios linfáticos extirpados y evaluados patológicamente para la enfermedad
			\\ \hline	56	&	Margin status reexcision	&	Estado de los márgenes de la cirugía del cáncer de mama
			\\ \hline	57	&	Menopause Status	&	Estado de la menopausia de una mujer, el cese permanente de la menstruación, generalmente definido por 6 a 12 meses de amenorrea.
			\\ \hline	58	&	Metastatic Site	&	Localización anatómica a la que se ha extendido el cáncer
			\\ \hline	59	&	Metastatic Site Other	&	Otra localización anatómica a la que se ha extendido el cáncer
			\\ \hline	60	&	Metastatic tumor indicator	&	Se refiere al si el cáncer se ha extendido desde el tumor original (primario) a órganos o ganglios linfáticos distantes.
			\\ \hline	61	&	First Pathologic Diagnosis Biospecimen Acquisition Method Type	&	Nombre del procedimiento para asegurar el tejido utilizado para el diagnóstico patológico original
			\\ \hline	62	&	First Pathologic Diagnosis Biospecimen Acquisition Other Method Type	&	Método utilizado para obtener tejido para un diagnóstico patológico original que es diferente de otros métodos identificados
			\\ \hline	63	&	Micromet detection by ihc	&	Detección de micrometastatis por ihc
			\\ \hline	64	&	Mutation Count	&	Recuento de mutaciones
			\\ \hline	65	&	New Neoplasm Event Post Initial Therapy Indicator	&	Indicador para identificar si un paciente ha tenido un nuevo evento tumoral después del tratamiento inicial
			\\ \hline	66	&	Nte cent 17 HER2 ratio	&	Valor de la relación señal HER2 neu cromosoma 17 en el carcinoma metastásico de mama 
			\\ \hline	67	&	Nte er ihc intensity score	&	Carcinoma metastásico de mama inmunohistoquímica con puntuación de células con ER positivo
			\\ \hline	68	&	Nte er status	&	Estado del receptor de estrógenos del carcinoma metastásico de mama
			\\ \hline	69	&	Nte er status ihc positive	&	Categoría porcentual de células de carcinoma de mama metastásico con nivel de receptores de estrógenos
			\\ \hline	70	&	Nte HER2 fish status	&	Tipo de resultado de hibridación in situ de HER2 neu de procedimiento de laboratorio de carcinoma de mama metastásico
			\\ \hline	71	&	Nte HER2 positivity ihc score	&	Resultado del nivel de inmunohistoquímica erbb2 de carcinoma de mama metastásico
			\\ \hline	72	&	Nte HER2 status	&	Término que indica si se realizó el proceso de laboratorio de carcinoma de mama metastásico estado del receptor de inmunohistoquímica HER2 neu
			\\ \hline	73	&	Nte HER2 status ihc positive	&	Porcentaje de células Carcinoma de mama metastásico HER2 erbb positivos 
			\\ \hline	74	&	Nte pr ihc intensity score	&	Puntuación de intensidad Nte pr ihc
			\\ \hline	75	&	Nte pr status by ihc	&	Estado del receptor de progesterona del carcinoma de mama metastásico
			\\ \hline	76	&	Nte pr status ihc positive	&	Porcentaje de células de nivel de receptor de progesterona de carcinoma de mama metastásico
			\\ \hline	77	&	Oct embedded	&	Término que indica si se realizó una incrustación OCT (Optimal cutting temperature compound)
			\\ \hline	78	&	Oncotree Code	&	Código del tipo de cancer en formato Oncotree para cBioPortal
			\\ \hline	79	&	Overall Survival (Months)	&	Supervivencia general (meses)
			\\ \hline	80	&	Overall Survival Status	&	Estado de supervivencia global del paciente.
			\\ \hline	81	&	Other Patient ID	&	Otro código de identificación del paciente
			\\ \hline
		\end{tabular}
	\end{threeparttable}
\end{table*}

\begin{table*} [!htb]
	\footnotesize
	\begin{threeparttable}
		\begin{tabular}{p{1cm} p{4cm} p{10cm}}
			\\ \hline	82	&	Other Sample ID	&	Otro código de identificación de la muestra
			\\ \hline	83	&	Pathology Report File Name	&	Nombre del archivo del informe de patología
			\\ \hline	84	&	Disease Surgical Margin Status	&	Resultados concluyentes tras el examen de los márgenes tisulares para detectar la presencia de la enfermedad.
			\\ \hline	85	&	Adjuvant Postoperative Pharmaceutical Therapy Administered Indicator	&	Término para indicar si el paciente tuvo o no terapia farmacéutica.
			\\ \hline	86	&	Primary Tumor Site	&	Sitio del tumor para identificar la subdivisión de órganos en un individuo con cáncer
			\\ \hline	87	&	Project code	&	Código de proyecto
			\\ \hline	88	&	Tissue Prospective Collection Indicator	&	Indicador de recolección prospectiva de tejido
			\\ \hline	89	&	PR positivity define method	&	Método utilizado para definir PR positiva
			\\ \hline	90	&	PR positivity ihc intensity score	&	Puntuación de intensidad ihc positiva de PR
			\\ \hline	91	&	PR positivity scale other	&	Otra medida de Porcentaje positivo del receptor de progesterona
			\\ \hline	92	&	PR positivity scale used	&	Escala de inmunohistoquímica para el hallazgo de receptores de progesterona positiva en el  carcinoma de mama
			\\ \hline	93	&	PR status by ihc	&	Estado del receptor de progesterona del carcinoma de mama
			\\ \hline	94	&	PR status ihc percent positive	&	Categoría de porcentaje celular del nivel del receptor de progesterona
			\\ \hline	95	&	Race Category	&	Información sobre la raza
			\\ \hline	96	&	Did patient start adjuvant postoperative radiotherapy?	&	Término para indicar si el paciente inició radioterapia postoperatoria adyuvante.
			\\ \hline	97	&	Tissue Retrospective Collection Indicator	&	Término para indicar el marco de tiempo de obtención de tejido, identificando si el tejido fue obtenido y almacenado antes del inicio del proyecto.
			\\ \hline	98	&	Number of Samples Per Patient	&	Número de muestras por paciente
			\\ \hline	99	&	Sample Type	&	Tipo de muestra
			\\ \hline	100	&	Sex	&	Sexo del paciente
			\\ \hline	101	&	Somatic Status	&	Estado somático
			\\ \hline	102	&	Staging System	&	Sistema de estadificación
			\\ \hline	103	&	Staging System\_1	&	Otro sistema de estadificación
			\\ \hline	104	&	Surgery for positive margins	&	Nombre del procedimiento quirúrgico primario de carcinoma de mama
			\\ \hline	105	&	Surgery for positive margins other	&	Nombre del procedimiento quirúrgico para márgenes positivos
			\\ \hline	106	&	Surgical procedure first	&	Nombre del procedimiento quirúrgico de carcinoma de mama
			\\ \hline	107	&	Tissue Source Site	&	Sitio fuente recopilado de la muestra (tejido, células o sangre) y metadatos clínicos que luego se envían al recurso principal de bioespecímenes.
			\\ \hline	108	&	TMB (nonsynonymous)	&	Número total de mutaciones (cambios) que se encuentran en el ADN de las células cancerosas
			\\ \hline	109	&	Person Neoplasm Status	&	El estado o condición de la neoplasia de un individuo en un momento determinado
			\\ \hline	110	&	Tumor Disease Anatomic Site	&	Término que describe el sitio anatómico del tumor o enfermedad
			\\ \hline
		\end{tabular}
	\end{threeparttable}
\end{table*}


 \section{Fase 4: Análisis Exploratorio de datos oncológicos}

En esta fase, el científico de datos obtiene el conjunto de datos o imágenes que fueron organizados previamente por el ingeniero de datos y realiza un \textit{Análisis exploratorio de datos} para descubrir patrones generales en la información generada. Cabe resaltar, que en esta fase el acompañamiento del medico experto en oncología es de vital importancia, ya que los datos o imágenes que van ser explorados por el científico pueden contener variables que pueden tener o no un valor significativo para el experto, ayudando así a determinar si el análisis planteado para responder la pregunta va o no por un buen camino, de modo que es posible que se agreguen o eliminen diversas variables para lograr el resultado esperado. Adicionalmente, es necesario que los diversos análisis generados estén apoyados con gráficas que sean entendibles por todo el \textit{Data Analysis Team}, esto con el proposito de aportar ideas, y desde esta fase ir encontrando posibles correlaciones entre las variables oncológicas.

Se debe agregar, que en esta fase se abarcan todas las actividades para construir el conjunto de datos o imágenes que se utilizará en la siguiente etapa de modelado y ejecución. Entre las actividades se encuentran el procesamiento y transformación de datos oncológicos, en donde es necesario realizar la limpieza de datos, combinar datos de múltiples fuentes y transformar los datos en variables de valor. En esta fase, es importante el trabajo en equipo y la comunicación continua entre el ingeniero y el científico de datos para tratar los valores no válidos o faltantes, eliminar duplicados, dar un formato adecuado y combinar archivos, tablas y plataformas. Adicionalmente, el medico experto en oncología deberá proporcionar un visto bueno para proceder con la siguiente fase. Esto dado que al ser experto en el tema de dominio tiene un conocimiento mas profundo de las variables o imágenes que esta observando, y si existiese información innecesaria para el diagnostico del cáncer de mama es posible depurar dicha información para que no afecte el entrenamiento y posterior ejecución del modelo de ML y DL.

\subsection{Análisis parcial de datos crudos}
En primer lugar, se realizó un análisis parcial del conjunto de datos \textit{“Breast Invasive Carcinoma (TCGA, Cell 2015)”} para conocer su composición inicial(cruda) y así poder identificar los registros que deben ser eliminados, transformados ó imputados. Cabe resaltar, que esta etapa es propuesta como parte de esta investigación para los datos de tipo genómico relacionados cáncer de mama. Lo anterior, debido a que el \textit{EDA\footnote{Exploratory Data Analysis}} tradicional parte del análisis descriptivo, y en este caso los tipos de datos son obtenidos de diferentes fuentes medicas las cuales no presentan una estructura fija ni estandar  en la informacion recopilada de los pacientes que padecen esta enfermedad, por lo que seria incorrecto realizar un análisis sobre datos que dada su estructura y forma generan informacion errónea. En la figura \ref{EDA} se puede observar las composición estadística unidimensional de la 110 variables, las cuales permitieron identificar el comportamiento inicial de los datos. Con base a las gráficas obtenidas, se genero el siguiente análisis: 

\begin{itemize}[label=\HandPencilLeft]
	\item El conjunto de datos esta conformado por 95 variables \textit{Categóricas} y 15 variables \textit{Numéricas}.
	
	\item Dada la naturaleza de las preguntas planteadas en el BCQM en donde se busca la identificación de características genéticas, las variables \textit{Study ID, Patient ID, Sample ID, Other Patient ID, Other Sample ID, Form completion date y Pathologyc Report File Name} no generan un aporte significativo para encontrar una respuesta de valor, dado lo anterior fueron eliminadas del conjunto de datos con el cual se entrenaron a los modelos de ML.
\end{itemize}

\newpage
\begin{figure}
	\centering
	\includegraphics[width=1
	\linewidth]{NOTEBOOK/IMAGES_EDA/1}
\end{figure}

\begin{figure}
	\centering
	\includegraphics[width=1
	\linewidth]{NOTEBOOK/IMAGES_EDA/2}
\end{figure}

\begin{figure}
	\centering
	\includegraphics[width=1
	\linewidth]{NOTEBOOK/IMAGES_EDA/3}
\end{figure}

\begin{figure}
	\centering
	\includegraphics[width=1
	\linewidth]{NOTEBOOK/IMAGES_EDA/4}
\end{figure}

\begin{figure}
	\centering
	\includegraphics[width=1
	\linewidth]{NOTEBOOK/IMAGES_EDA/5}
	\label{EDA}
	\caption{Distribución del conjunto de datos del Carcinoma invasivo de mama.}\label{fig:foobar}
\end{figure}


\subsection{Detección de datos Ausentes}
En segundo lugar, basados en la obtención de los atributos del conjunto de datos \textit{“Breast Invasive Carcinoma (TCGA, Cell 2015)”}, se realizo un análisis de la cantidad de datos perdidos para identificar las variables y en la etapa posterior realizar la limpieza y el pre-procesamiento de los datos de destino hacerlos consistentes y sin ningún tipo de ruido. Los resultados obtenidos se pueden observar en la figura \ref{Missing_Bar_Chart}:


\begin{figure}[!htb]
	\centering
	\includegraphics[width=1\linewidth]{IMAGENES/Missing_Bar_Chart}
	\caption{Datos perdidos expresados en una gráfica de barras.}
	\label{Missing_Bar_Chart}
\end{figure}

\begin{figure}[!htb]
	\centering
	\includegraphics[width=1
	\linewidth]{IMAGENES/Missing_Spectrum}
	\caption{Datos perdidos expresados en un diagrama espectral.}
	\label{Missing_Spectrum}
\end{figure}


\subsection{Análisis Descriptivo }
En primer lugar, se realizo el respectivo análisis descriptivo para detectar cual es comportamiento de los atributos del conjunto de datos \textit{“Breast Invasive Carcinoma (TCGA, Cell 2015)”}. En la gráfica \ref{EDA} se puede observar las gráficas estadísticas unidimensionales de la 110 variables, las cuales permitieron extraer  las características mas representativas y permitieron identificar el comportamiento de los datos.

\begin{table*}[!htb]
	\footnotesize
	\begin{threeparttable}
		\caption{Conjunto de datos del Carcinoma invasivo de mama (TCGA, Cell 2015).}
		\label{Analisis_Descriptivo}
		\begin{tabular}{p{2.5cm} p{7cm} p{6.5cm}} \toprule
			\begin{center}Variable\end{center}   	 
			&\begin{center}Análisis descriptivo\end{center}             
			&\begin{center}Gráfico estadístico\end{center}\\ \hline
			%------------------------------------------------------	
			Diagnosis Age
			& La \textit{edad de diagnostico} del cáncer de mama tiene una tendencia central de 59 años, en donde la edad mínima presentada es de  26 años y la edad máxima presentada es de 90 años.
			
			& \begin{center}\includegraphics[width=1\linewidth]{NOTEBOOK/IMAGENES_DESCRIPTIVAS/1_diagnosis_age}\end{center}
			\\ \hline
			%------------------------------------------------------	
			AJCC Metastasis Stage Code 
			& El código AJCC para la \textit{estadificación metastásica(M) del cáncer} se visualiza en orden descendente de la siguiente manera: En primer lugar, el código \textit{m0} se presenta en 707 pacientes en donde el cáncer hizo metástasis pero no se  disemino a otras partes del cuerpo. En segundo lugar se encuentra el código \textit{mx} presentado en 96 pacientes a los cuales no fue posible medir la metástasis. En tercer lugar se encuentra el código \textit{m1} presentado en 13 pacientes en donde el cáncer se diseminó a otras partes del cuerpo. En ultimo lugar se encuentra el código \textit{cM0(i+)} presentado en 2 pacientes en los cuales no se detecto evidencia de metástasis a distancia, pero hubo un pequeño número de células en las cuales se encontró una metástasis diminuta (no mayor de 0.2 mm) detectada en ganglios linfáticos no regionales \cite{NCI}.
			
			& \begin{center}\includegraphics[width=1\linewidth]{NOTEBOOK/IMAGENES_DESCRIPTIVAS/2_metastasis_stage_code}\end{center}
			\\ \hline
			%------------------------------------------------------	
			AJCC Neoplasm Disease Lymph Node Stage Code
			& El código AJCC para la \textit{estadificación del cáncer por neoplasia del ganglio linfático(N)} se visualiza en orden descendente de la siguiente manera: En primer lugar, el código \textit{n0} se presenta en 250 pacientes en donde no hay cáncer en los ganglios linfáticos cercanos. En segundo lugar, el código \textit{n1a} se presento en 126 pacientes en donde el cáncer se diseminó a 1 ganglio linfáticos debajo del brazo con al menos un área de cáncer diseminada de más de 2 mm de ancho. En tercer lugar el código \textit{n0(i-)} se presento en 126 pacientes en donde, no hay evidencia histológica de metástasis en los ganglios linfáticos regionales. Del cuarto lugar en adelante, se refiere a la cantidad y ubicación de los ganglios linfáticos que contienen cáncer. Cuanto mayor sea el número después de la $n$, más ganglios linfáticos se vieron afectados.	
			
			& \begin{center}\includegraphics[width=1\linewidth]{NOTEBOOK/IMAGENES_DESCRIPTIVAS/3_neoplasm_lymph_code}\end{center}
			\\ \hline
			%------------------------------------------------------	
		\end{tabular}
	\end{threeparttable}
\end{table*}

\begin{table*}[!htb]
	\footnotesize
	\begin{threeparttable}
		\begin{tabular}{p{2.5cm} p{7cm} p{6.5cm}} \toprule
			%------------------------------------------------------	
			AJCC Neoplasm Disease Stage Code
			& Las etapas AJCC para la \textit{estadificación del cáncer por neoplasia} se visualiza en orden descendente de la siguiente manera: En primer lugar, la \textit{etapa iia} se presenta en 278 pacientes en donde el tumor mide más de 20 mm pero no más de 50 mm y no se ha propagado a los ganglios linfáticos axilares. En segundo lugar, la \textit{etapa iib} se presento en 190 pacientes en donde el tumor mide más de 50 mm pero no se ha propagado a los ganglios linfáticos axilares. En tercer lugar la \textit{etapa iiia} se presento en 112 pacientes en donde el tumor se diseminó de 4 a 9 ganglios linfáticos axilares o los ganglios linfáticos mamarios internos, pero no se ha propagado a otras partes del cuerpo. En cuarto lugar, la \textit{etapa i} se presento en 75 pacientes  en donde el tumor es pequeño, invasivo y no se ha propagado a los ganglios linfáticos. En quinto lugar, la \textit{etapa ia} se presento en 60 pacientes en donde el tumor mide menos de 20 mm  y no se ha propagado a los ganglios linfáticos. Del sexto lugar en adelante, el tumor puede tener cualquier tamaño y se ha propagado a otros órganos, como los huesos, los pulmones, el cerebro, el hígado, los ganglios linfáticos distantes o la pared torácica.
			
			& \begin{center}\includegraphics[width=1\linewidth]{NOTEBOOK/IMAGENES_DESCRIPTIVAS/4_neoplasm_disease_stage_code}\end{center}
			\\ \hline
			%------------------------------------------------------	
			AJCC Tumor Stage Code 
			& El código AJCC para la \textit{estadificación el tumor (T) primario del cáncer} se visualiza en orden descendente de la siguiente manera: En primer lugar, el código \textit{t2} se presenta en 459 pacientes en donde el tumor mide más de 20 mm pero no más de 50 mm. En segundo lugar, el código \textit{t1c} se presento en 173 pacientes en donde el tumor mide  de 10 mm a 20 mm o menos. En tercer lugar el código \textit{t3} se presento en 104 pacientes en donde el tumor mide más de 50 mm. En cuarto lugar el código \textit{t1} se presento en 34 pacientes en donde el tumor  mide 20 mm o menos en su área más ancha. En quinto lugar el código \textit{t4b} se presento en 26 pacientes en donde el tumor ha crecido dentro de la piel. Del sexto lugar en adelante, el código \textit{t1b} se presento en 11 pacientes en donde el tumor mide más de 5 mm pero menos de 10 mm, el código \textit{t4} se presento en 5 pacientes en donde el tumor ha crecido hacia la pared torácica, el código \textit{t4d} se presento en 2 pacientes en donde es cáncer de mama inflamatorio, el código \textit{t1a} se presento en 1 paciente en donde el tumor mide más de 1 mm pero menos de 5 mm, el código \textit{t2b} se presento en 1 paciente en donde el tumor mide más de el tumor mide más de 25 mm pero menos de 50 mm.
			
			& \begin{center}\includegraphics[width=1\linewidth]{NOTEBOOK/IMAGENES_DESCRIPTIVAS/5_tumor_stage_code}\end{center}
			\\ \hline
		\end{tabular}
	\end{threeparttable}
\end{table*}


\begin{table*}[!htb]
	\footnotesize
	\begin{threeparttable}
		\begin{tabular}{p{2.5cm} p{7cm} p{6.5cm}} \toprule
			%------------------------------------------------------	
			Cancer Type Detailed
			& El \textit{tipo de cáncer de mama} se visualiza en orden descendente de la siguiente manera: En primer lugar, el \textit{cáncer Ductal invasivo} se presento en 491 pacientes. En segundo  lugar, el \textit{cáncer Lobulillar invasivo} se presento en 127 pacientes. En tercer lugar, el \textit{cáncer invasivo con otros diagnósticos} se presento en 112 pacientes. En cuarto lugar,  el \textit{cáncer mixto (Ductal y Lobulillar)} se presento en 88 pacientes.
			
			& \begin{center}\includegraphics[width=1\linewidth]{NOTEBOOK/IMAGENES_DESCRIPTIVAS/6_cancer_type_detailed}\end{center}
			\\ \hline
			%------------------------------------------------------	
			Days to Sample Collection
			& El intervalo de días para la \textit{recolección de muestras} tiene una tendencia central aproximada de 451 días, en donde el tiempo mínimo presentado es de 16 días y el tiempo máximo presentado de 7804 días.
			
			& \begin{center}\includegraphics[width=1\linewidth]{NOTEBOOK/IMAGENES_DESCRIPTIVAS/7_days_sample_collection}\end{center}
			\\ \hline
			%------------------------------------------------------	
			Days to Last Followup
			& El Intervalo de tiempo desde la fecha del \textit{último seguimiento} hasta la fecha del diagnóstico patológico inicial tiene una tendencia central aproximada de 579 días, en donde el tiempo mínimo de diagnostico  presentado es de 1 día y el tiempo maximo de diagnostico presentado de 7067 días.
			
			& \begin{center}\includegraphics[width=1\linewidth]{NOTEBOOK/IMAGENES_DESCRIPTIVAS/8_days_last_followup}\end{center}
			\\ \hline
		\end{tabular}
	\end{threeparttable}
\end{table*}


\begin{table*}[!htb]
	\footnotesize
	\begin{threeparttable}
		\begin{tabular}{p{2.5cm} p{7cm} p{6.5cm}} \toprule
			%------------------------------------------------------	
			Disease Free (Months)
			& El Intervalo de \textit{meses sin enfermedad} tiene una tendencia central aproximada de 32 meses, en donde el tiempo mínimo presentado es de 16 meses y el tiempo máximo presentado de 281 meses.
			
			& \begin{center}\includegraphics[width=1\linewidth]{NOTEBOOK/IMAGENES_DESCRIPTIVAS/9_disease_free_months}\end{center}
			\\ \hline
			
			%------------------------------------------------------	
			Disease Free Status
			& El \textit{estado libre de enfermedad} se clasifica de forma nominal en dos categorías: La primera categoría corresponde al estado \textit{disease free} en donde se encuentran 733 pacientes que estuvieron libre de enfermedad durante un tiempo determinado. La segunda categoría corresponde al estado \textit{progressed} en donde se encuentran 85 pacientes en donde la enfermedad fue avanzando de manera progresiva. 
			
			& \begin{center}\includegraphics[width=1\linewidth]{NOTEBOOK/IMAGENES_DESCRIPTIVAS/10_disease_free_status}\end{center}
			\\ \hline
			
			%------------------------------------------------------	
			ER positivity scale other
			&La variable \textit{Otra escala de receptor de estrogeno positivo} se visualiza en orden descendente de la siguiente manera: En primer lugar, la escala \textit{allred score 8} se presenta en 671 pacientes que tienen células cancerosas con receptores positivos de estrogeno con un crecimiento fuerte. En segundo lugar, la escala $>$\textit{75\%} se presento en 48 pacientes que tienen células cancerosas con receptores positivos de estrogeno con un crecimiento anormal. En tercer lugar la escala \textit{h-score} se presento en 26 pacientes con un porcentaje moderado de células tumorales con tinción de membrana celular positivo de estrogeno. En cuarto, la escala \textit{allred score 0} se presenta en 15 pacientes que tienen células cancerosas con receptores negativos de estrogeno. Del quinto lugar en adelante, los pacientes presentan células cancerosas con receptores positivos de estrogeno con un comportamiento de crecimiento moderado.
			& \begin{center}\includegraphics[width=1\linewidth]{NOTEBOOK/IMAGENES_DESCRIPTIVAS/11_er_positivity_scale_other}\end{center}
			\\ \hline
		\end{tabular}
	\end{threeparttable}
\end{table*}

\begin{table*}[!htb]
	\footnotesize
	\begin{threeparttable}
		\begin{tabular}{p{2.5cm} p{7cm} p{6.5cm}} \toprule
			ER Status By IHC
			&El \textit{Estado del receptor de estrogeno por análisis IHC} se clasifica de forma nominal en dos categorías: La primera categoría corresponde al estado \textit{positive} en donde se encuentran 643 pacientes a los cuales al realizar el análisis de inmunohistoquímica(IHC) sobre tejido mamario canceroso se determino que las células cancerosas tienen receptores positivos de estrogeno y la segunda categoría corresponde al estado \textit{negative} en donde se encuentran 175 pacientes a los cuales al realizar el análisis IHC sobre tejido mamario canceroso se determino que las células cancerosas no presentan receptores de estrogeno.
			& \begin{center}\includegraphics[width=1\linewidth]{NOTEBOOK/IMAGENES_DESCRIPTIVAS/13_er_status_ihc}\end{center}
			\\ \hline
			
			%------------------------------------------------------	
			ER Status IHC Percent Positive
			&El \textit{Estado del porcentaje del receptor de estrogeno por análisis IHC} se visualiza en orden descendente de la siguiente manera: En primer lugar, 668 pacientes presentan células cancerosas con receptores positivos de estrogeno en un \textit{90-99\%} obre tejido mamario. En segundo lugar, 42 pacientes presentan células cancerosas con receptores positivos de estrogeno en un $<$\textit{10\%} sobre tejido mamario. En tercer lugar, 33 pacientes presentan células cancerosas con receptores positivos de estrogeno en un \textit{70-79\%} sobre tejido mamario. En cuarto lugar, 21 pacientes presentan células cancerosas con receptores positivos de estrogeno en un \textit{80-89\%} sobre tejido mamario. En quinto lugar, 20 pacientes presentan células cancerosas con receptores positivos de estrogeno en un \textit{10-19\%} sobre tejido mamario. Del sexto lugar en adelante, los pacientes presentan células cancerosas con receptores positivos de estrogeno con un porcentaje variable.
			& \begin{center}\includegraphics[width=1\linewidth]{NOTEBOOK/IMAGENES_DESCRIPTIVAS/14_er_status_ihc_percent_positive}\end{center}
			\\ \hline
			
			%------------------------------------------------------	
			Ethnicity Category
			&La \textit{Categoría étnica} se clasifica de forma nominal en dos categorías: La primera categoría corresponde al estado \textit{not hispanic or latino} en donde se encuentran 787 pacientes que no presentan una ascendencia latina o de origen español y la segunda categoría corresponde al estado estado \textit{hispanic or latino} en donde se encuentran 31 pacientes que presentan una ascendencia latina o de origen español.
			& \begin{center}\includegraphics[width=1\linewidth]{NOTEBOOK/IMAGENES_DESCRIPTIVAS/15_ethnicity_category}\end{center}
			\\ \hline
			
		\end{tabular}
	\end{threeparttable}
\end{table*}

\begin{table*}[!htb]
	\footnotesize
	\begin{threeparttable}
		\begin{tabular}{p{2.5cm} p{7cm} p{6.5cm}} \toprule
			%------------------------------------------------------	
			First surgical procedure other
			& La \textit{Primera intervención quirúrgica} se visualiza en orden descendente de la siguiente manera: En primer lugar, la intervención de \textit{resección quirúrgica} se realizo a 660 pacientes a los cuales se les extirpo todo el tumor o la mayor cantidad posible del mismo. En segundo lugar, la intervención quirúrgica \textit{Cirugía de Patey} se realizo a 44 pacientes a los cuales se les extirpo todo el seno, incluida la piel, el tejido mamario, la areola y el pezón junto con la mayoría de los ganglios linfáticos axilares. En tercer lugar, la intervención quirúrgica de \textit{mastectomía segmentaria} se realizo a 34 pacientes a los cuales se les extirpo el cáncer u otro tejido anormal del seno y parte del tejido normal que lo rodea, pero no el seno en sí. En cuarto lugar, la intervención quirúrgica de \textit{escisión local amplia} se realizo a 17 pacientes con los cuales se uso un bisturí para extirpar un tumor u otra lesión anormal y parte del tejido normal que lo rodeaba. En quinto lugar, la intervención quirúrgica \textit{biposia exicional} se realizo a 8 pacientes a los cuales se les extirpo una masa completa o área sospechosa. Del sexto lugar en adelante se utilizaron otros tipos de intervenciones quirúrgicas.
			
			& \begin{center}\includegraphics[width=1\linewidth]{NOTEBOOK/IMAGENES_DESCRIPTIVAS/16_surgical_other}\end{center}
			\\ \hline
			
			%------------------------------------------------------	
			Fraction Genome Altered
			& La \textit{Estabilidad genómica} o también llamado \textit{Fenotipo mutador} necesaria para que las células acumulen múltiples mutaciones estimulando el desarrollo del cáncer, presenta un tendencia central de 25.07\% relacionado a genes que se han visto afectados por las ganancias o pérdidas del número de copias celulares, en donde el porcentaje mínimo presentado es del 12,22\% y el máximo del 99,71\%
			
			& \begin{center}\includegraphics[width=1\linewidth]{NOTEBOOK/IMAGENES_DESCRIPTIVAS/17_fraction_genome_altered}\end{center}
			\\ \hline
			
			%------------------------------------------------------	
			HER2 fish status
			& La \textit{prueba FISH (hibridación fluorescente in situ)} analiza el ADN de las células cancerosas en busca de copias adicionales del gen \textit{HER2 (Receptor 2 del factor de crecimiento epidérmico humano)}. En este caso el \textit{estado FISH} se clasifica de forma nominal en dos categorías: La primera categoría corresponde al estado \textit{negativo} en donde se encuentran 758 pacientes en los cuales la proteína HER2 no está involucrada en el crecimiento del tumor mamario y la segunda categoría corresponde al estado \textit{Positivo} en donde se encuentran 60 pacientes en los cuales las células cancerígenas producen demasiada HER2 estimulando el crecimiento del tumor mamario. 
			& \begin{center}\includegraphics[width=1\linewidth]{NOTEBOOK/IMAGENES_DESCRIPTIVAS/19_her_2_fish_status}\end{center}
			\\ \hline
		\end{tabular}
	\end{threeparttable}
\end{table*}

\begin{table*}[!htb]
	\footnotesize
	\begin{threeparttable}
		\begin{tabular}{p{2.5cm} p{7cm} p{6.5cm}} \toprule
			%------------------------------------------------------	
			HER2 ihc percent positive
			& El \textit{Porcentaje positivo de HER2 según la prueba de inmunohistoquímica (IHC)}, se visualiza en orden descendente de la siguiente manera: En primer lugar, 757 pacientes presentan tinción apenas perceptible, observada en un valor $<$\textit{10\%} de las células tumorales. En segundo lugar, 17 pacientes presentan una tinción de membrana moderada observada en \textit{10-19\%} de las células tumorales. En tercer lugar, 12 pacientes presentan una tinción de membrana intensa observada en \textit{90-99\%} de las células tumorales. Del cuarto lugar en adelante, los pacientes presentan una tinción de membrana variable en un rango de débil a moderada.
			& \begin{center}\includegraphics[width=1\linewidth]{NOTEBOOK/IMAGENES_DESCRIPTIVAS/20_her_2_ihc_percent_positive}\end{center}
			\\ \hline
			
			%------------------------------------------------------	
			HER2 ihc score
			& El \textit{puntaje HER2 según la prueba de inmunohistoquímica (IHC)}, se visualiza en orden descendente de la siguiente manera: En primer lugar, 601 pacientes presentan una puntuación \textit{1+}, esto significa  que el tipo de cáncer es HER2-negativo y no es posible tratarlo con medicamentos que tienen a la proteína HER2 como blanco. En segundo lugar, 151 pacientes presentan una puntuación \textit{2+}, esto significa que el estado de HER2 del tumor no está claro, es decir es \textit{ambiguo} y es necesario hacer una prueba del estado FISH para clarificar el resultado. En tercer lugar, 66 pacientes presentan una puntuación \textit{3+}, esto significa que el cáncer es HER2-positivo y es posible tratarlo con medicamentos que tienen a la proteína HER2 como blanco.
			
			& \begin{center}\includegraphics[width=1\linewidth]{NOTEBOOK/IMAGENES_DESCRIPTIVAS/21_her_2_ihc_score}\end{center}
			\\ \hline
			
			%------------------------------------------------------	
			Neoplasm Histologic Type Name
			& El \textit{Nombre del tipo histológico de neoplasia}, se visualiza en orden descendente de la siguiente manera: En primer lugar, 602 pacientes presentan \textit{Carcinoma Ductal Invasivo (IDC)} también llamado \textit{Carcinoma Ductal Infiltrante}. En segundo lugar, 143 pacientes presentan \textit{Carcinoma Lobulillar Invasivo (ILC)} también llamado \textit{Carcinoma Lobulillar Infiltrante.} En tercer lugar, 28 pacientes presentan otro tipo de neoplasia. En cuarto lugar, 23 pacientes presentan \textit{Tumores o Neoplasia mixta (MTCB)}, es decir conformada por los tipos de cáncer LBC e IDC. En quinto lugar, 14 pacientes presentan \textit{Carcinoma Mucinoso (MBC)}. En quinto lugar, 5 pacientes presentan \textit{Carcinoma Medular (MC)}. En sexto lugar, 3 pacientes presentan \textit{Carcinoma Metaplastico (MMC)}.
			& \begin{center}\includegraphics[width=1\linewidth]{NOTEBOOK/IMAGENES_DESCRIPTIVAS/22_neoplasm_histologic_type}\end{center}
			\\ \hline
		\end{tabular}
	\end{threeparttable}
\end{table*}

\begin{table*}[!htb]
	\footnotesize
	\begin{threeparttable}
		\begin{tabular}{p{2.5cm} p{7cm} p{6.5cm}} \toprule
			%------------------------------------------------------	
			Neoadjuvant Therapy Type Administered Prior To Resection Text
			& El \textit{Tratamiento neoadyuvante del paciente} se clasifica de forma nominal en dos categorías: La primera categoría corresponde al estado \textit{No} en donde se encuentran 809 pacientes que no recibieron un tratamiento inicial antes del tratamiento principal y la segunda categoría corresponde al estado \textit{Si} en donde se encuentran 9 pacientes a los cuales se les realizó un tratamiento inicial como quimioterapia, radioterapia o terapia hormonal para reducir el tamaño del tumor antes del tratamiento principal que generalmente consiste en cirugía.
			& \begin{center}\includegraphics[width=1\linewidth]{NOTEBOOK/IMAGENES_DESCRIPTIVAS/23_neoadjuvant_therapy}\end{center}
			\\ \hline
			
			%------------------------------------------------------	
			ICD-10 Classification
			& El \textit{Diagnóstico de cáncer de mama por medio de la décima revisión ICD} se visualiza en orden descendente de la siguiente manera: En primer lugar, 810 pacientes presentan el código \textit{C50.9} el cual corresponde a una neoplasia maligna de sitio no especificado. En segundo lugar, 3 pacientes presentan el código \textit{C50.3} el cual corresponde a una neoplasia maligna de cuadrante inferior interno de la mama. En tercer lugar, 2 pacientes presentan el código \textit{C50.4} el cual corresponde a una neoplasia maligna de cuadrante superior externo de la mama. En cuarto lugar, 1 paciente presenta el código \textit{C50.5} el cual corresponde a una neoplasia maligna de cuadrante superior interno de la mama. En quinto lugar, 1 paciente presenta el código \textit{C50.2} el cual corresponde a una neoplasia maligna de cuadrante inferior externo de la mama. En sexto lugar, 1 paciente presentan el código \textit{C50.8} el cual corresponde a una neoplasia maligna de sitios superpuestos de la mama.
			
			& \begin{center}\includegraphics[width=1\linewidth]{NOTEBOOK/IMAGENES_DESCRIPTIVAS/25_icd_10_classification}\end{center}
			
			\begin{center}\includegraphics[width=0.75\linewidth]{NOTEBOOK/IMAGENES_DESCRIPTIVAS/25_icd_Breast_Clock_Position}\end{center}
			\\ \hline
			
			%------------------------------------------------------	
			IHC-HER2
			& El \textit{Estado de HER2 según la prueba de inmunohistoquímica (IHC)}, se visualiza en orden descendente de la siguiente manera: En primer lugar, 552 pacientes presentan un estado \textit{negativo}, es decir el cáncer no es posible tratarlo con medicamentos que tienen a la proteína HER2 como blanco. En segundo lugar, 145 pacientes presentan un estado \textit{ambiguo} esto significa que el estado de HER2 del tumor no está claro y es necesario hacer una prueba del estado FISH para clarificar el resultado. En tercer lugar, 121 pacientes presentan un estado \textit{positivo}, es decir el cáncer es posible tratarlo con medicamentos que tienen a la proteína HER2 como blanco.
			& \begin{center}\includegraphics[width=1\linewidth]{NOTEBOOK/IMAGENES_DESCRIPTIVAS/26_ihc_her_2}\end{center}
			\\ \hline
		\end{tabular}
	\end{threeparttable}
\end{table*}

\begin{table*}[!htb]
	\footnotesize
	\begin{threeparttable}
		\begin{tabular}{p{2.5cm} p{7cm} p{6.5cm}} \toprule
			%------------------------------------------------------	
			Year Cancer Initial Diagnosis
			& El intervalo de años para el \textit{Diagnóstico Patológico inicial del cáncer} tiene una tendencia central aproximada en el año 2009, en donde el año de diagnóstico mínimo presentado es 1998 y año de diagnóstico máximo presentado es 2013.
			
			& \begin{center}\includegraphics[width=1\linewidth]{NOTEBOOK/IMAGENES_DESCRIPTIVAS/27_year_initial_diagnosis}\end{center}
			\\ \hline
			
			%------------------------------------------------------	
			Primary Lymph Node Presentation Assessment Ind-3
			& La \textit{Evaluación de los ganglios linfáticos} en la presentación primaria de la enfermedad se clasifica de forma nominal en dos categorías: La primera categoría corresponde al estado \textit{Si} en donde se encuentran 798 pacientes en los que el oncólogo detecto la presencia de enfermedad metastásica en los ganglios linfáticos axilares y la segunda categoría al estado \textit{No} en donde se encuentran 20 pacientes en los que el oncólogo no detecto la presencia de enfermedad metastásica.
			
			& \begin{center}\includegraphics[width=1\linewidth]{NOTEBOOK/IMAGENES_DESCRIPTIVAS/28_lymph_presentation}\end{center}
			\\ \hline
			
			%------------------------------------------------------	
			Positive Finding Lymph Node Hematoxylin and Eosin Staining Microscopy Count
			& El \textit{Recuento de ganglios linfáticos positivos identificados mediante microscopía óptica con tinción de hematoxilina y eosina (H\&E)} tiene una tendencia central aproximada de 5 ganglios linfáticos con estado positivo, es decir que la tinción de hematoxilina presento una mayor proporción evidenciando la presencia de un tumor en crecimiento. Adicionalmente, se puede observar que  el valor mínimo ganglios linfáticos afectados fue 1 y el valor máximo de ganglios linfáticos afectados fue 34. 
			& \begin{center}\includegraphics[width=1\linewidth]{NOTEBOOK/IMAGENES_DESCRIPTIVAS/29_positive_lymph_hematoxylin}\end{center}
			\\ \hline
			
			%------------------------------------------------------	
			Lymph Node(s) Examined Number
			& El número de \textit{ganglios linfáticos extirpados} y evaluados patológicamente para la enfermedad tiene una tendencia central aproximada de 10 ganglios linfáticos extirpados , en donde el número mínimo de ganglios linfáticos extirpados es 1 y el número máximo de ganglios linfáticos extirpados es 44.
			& 
			\begin{center}\includegraphics[width=1\linewidth]{NOTEBOOK/IMAGENES_DESCRIPTIVAS/31_lymph_examined_number}\end{center}
			\\ \hline
		\end{tabular}
	\end{threeparttable}
\end{table*}

\begin{table*}[!htb]
	\footnotesize
	\begin{threeparttable}
		\begin{tabular}{p{2.5cm} p{7cm} p{6.5cm}} \toprule
			%------------------------------------------------------	
			Menopause Status
			& El \textit{estado de la menopausia}, se visualiza en orden descendente de la siguiente manera: En primer lugar, 593 pacientes presentan un estado de \textit{PostMenopausia}, en el cual los niveles hormonales permanecen bajos y ya no se presenta un período mensual debido a que los ovarios han dejado de liberar óvulos. En segundo lugar, 166 pacientes presentan un estado de \textit{PreMenopausia}, en el cual la reserva ovárica empieza a disminuir y la mujer pierde su fertilidad. Suele iniciarse sobre los 40 años y tener una duración muy variable, desde pocos años hasta 10 o más. En tercer lugar, 59 pacientes presentan un estado de \textit{PeriMenopausia} el cual se presenta en una etapa más corta que precede a la menopausia y dura hasta los 12 meses posteriores de la última menstruación. Suele durar unos 2 o 3 años y se caracteriza por las irregularidades en el ciclo menstrual. La edad más habitual en la que se presenta la perimenopausia es entre los 45 y los 50 años\cite{ReproAsistidaORG}.
			
			& \begin{center}\includegraphics[width=1\linewidth]{NOTEBOOK/IMAGENES_DESCRIPTIVAS/32_menopause_status}\end{center}
			  \begin{center}\includegraphics[width=1\linewidth]{IMAGENES/menopausia}\end{center}
			\\ \hline
			
			%------------------------------------------------------	
			Metastatic tumor indicator
			& El \textit{Indicador de tumor metastásico} se clasifica de forma nominal en dos categorías: La primera categoría corresponde al estado \textit{No}, en donde se encuentran 807 pacientes en los que el cáncer no se extendió desde el tumor primario a órganos o ganglios linfáticos distantes. La segunda categoría corresponde al estado \textit{Si}, en donde se encuentran 11 pacientes en los que el cáncer se extendió desde el tumor primario a órganos o ganglios linfáticos distantes.
			
			& \begin{center}\includegraphics[width=1\linewidth]{NOTEBOOK/IMAGENES_DESCRIPTIVAS/33_metastatic_tumor_indicator}\end{center}
			\\ \hline
			
			%------------------------------------------------------	
			First Pathologic Diagnosis Biospecimen Acquisition Method Type
			& El \textit{tipo de método para adquisición de muestras biológicas} realizado en el primer diagnostico patológico, se visualiza en orden descendente de la siguiente manera: En primer lugar, la intervención quirúrgica de \textit{Biopsia con aguja gruesa (CNB)} se realizo a 495 pacientes en los cuales el medico utilizo una aguja hueca unida a una herramienta con resorte para extraer pedazos de tejido mamario de un área sospechosa. En segundo lugar, la intervención quirúrgica de \textit{resección tumoral} se realizo a 495 pacientes a los cuales se les extirpo todo el tumor o la mayor cantidad posible del mismo. En tercer lugar, la la intervención quirúrgica de \textit{Aspiración con aguja fina (FNA)} se realizo a 83 pacientes en los cuales el medico utilizo una aguja hueca muy delgada unida a una jeringa para extraer una pequeña cantidad de tejido mamario o líquido de un área sospechosa. En cuarto lugar, 59 pacientes utilizaron \textit{otros} tipos de métodos para adquisición de muestras biológicas. 
			& \begin{center}\includegraphics[width=1\linewidth]{NOTEBOOK/IMAGENES_DESCRIPTIVAS/34_biospecimen_method}\end{center}
			\\ \hline
		\end{tabular}
	\end{threeparttable}
\end{table*}

\begin{table*}[!htb]
	\footnotesize
	\begin{threeparttable}
		\begin{tabular}{p{2.5cm} p{7cm} p{6.5cm}} \toprule
			%--Continuacion,First Pathologic Diagnosis Biospecimen Acquisition Method Type
			&  En quinto lugar, la intervención quirúrgica \textit{Biopsia por escisión} se realizo a 21 pacientes a los cuales se les extirpo una masa completa o área sospechosa.En sexto lugar, la \textit{Citología} se realizó a 18 pacientes en donde se registró la simetría, el tamaño y la forma de la mama, así como cualquier evidencia de edema (piel de naranja), retracción del pezón o de la piel y eritema. En séptimo lugar, la intervención quirúrgica \textit{Biopsia por incisión} se realizo a 21 pacientes a los cuales se les extirpo una masa completa o área sospechosa
			& 
			\\ \hline
			
			%------------------------------------------------------	
			Micromet detection by ihc 
			& La \textit{Detección de micrometástasis según el método de tinción inmunohistoquímica(IHC) de citoqueratina(CK)} se clasifica de forma nominal en dos categorías: La primera categoría corresponde al estado \textit{No} en donde se encuentran 643 pacientes en los que se detecto la presencia de micrometástasis. La segunda categoría corresponde al estado \textit{Si} en donde se encuentran 175  pacientes en los que se detecto la presencia de micrometástasis conformada por grupos de células cancerosas que tienen entre 0,2 milímetros y 2 milímetros  utilizando el método inmunohistoquímica (IHC) de citoqueratina(CK).
			& \begin{center}\includegraphics[width=1\linewidth]{NOTEBOOK/IMAGENES_DESCRIPTIVAS/35_micromet_detection_ihc}\end{center}
			\\ \hline
			
			%------------------------------------------------------	
			Mutation Count
			& El \textit{recuento de mutaciones} en el ADN el cual ocasiona que una célula sana  crezca y se divida con mayor rapidez, tiene  una tendencia central aproximada de 36 mutaciones, en donde el número mínimo de mutaciones es 1 y el número máximo de mutaciones es 3860.
			& \begin{center}\includegraphics[width=1\linewidth]{NOTEBOOK/IMAGENES_DESCRIPTIVAS/36_mutation_count}\end{center}
			\\ \hline
			
			%------------------------------------------------------	
			Oct embedded
			& El uso del \textit{compuesto de temperatura de corte óptimo(OCT)} se clasifica de forma nominal en dos categorías: La primera categoría corresponde al estado \textit{Verdadero}, en donde se encuentran 481 pacientes, en los cuales la adquisición de muestras se realizo mediante el proceso OCT para generar secciones de tejidos en un portaobjetos para posteriormente generar el análisis histopatológico. La segunda categoría corresponde al estado \textit{Falso}, en donde se encuentran 337 pacientes, en los cuales la adquisición de muestras no se utilizo el proceso OCT.
			& \begin{center}\includegraphics[width=1\linewidth]{NOTEBOOK/IMAGENES_DESCRIPTIVAS/37_oct_embedded}\end{center}
			\\ \hline
		\end{tabular}
	\end{threeparttable}
\end{table*}

\begin{table*}[!htb]
	\footnotesize
	\begin{threeparttable}
		\begin{tabular}{p{2.5cm} p{7cm} p{6.5cm}} \toprule
			%------------------------------------------------------	
			Oncotree Code
			& El \textit{Código del tipo de cáncer en formato Oncotree} obtenido de la base de datos del sitio público cBioPortal para la genómica del cáncer, se visualiza en orden descendente de la siguiente manera: En primer lugar 490 pacientes, presentan el cáncer de tipo \textit{Carcinoma de seno Ductal Invasivo (IDC)}. En segundo lugar 127 pacientes, presentan el cáncer de tipo \textit{Carcinoma de seno Lobulillar invasivo (ILC)}. En tercer lugar 113 pacientes, presentan el cáncer de tipo \textit{Carcinoma de seno Invasivo(BRCA)}. En cuarto lugar 88 pacientes, presentan el cáncer de tipo \textit{Carcinoma de seno mixto ductal y lobulillar(MDCL)}.
			
			& \begin{center}\includegraphics[width=1\linewidth]{NOTEBOOK/IMAGENES_DESCRIPTIVAS/16_surgical_other}\end{center}
			\\ \hline
			
			%------------------------------------------------------	
			Overall Survival (Months)
			& La \textit{Supervivencia general} tiene una tendencia central aproximada en 30 meses (2 años y medio), en donde el tiempo mínimo de supervivencia es 6 meses y el tiempo máximo de supervivencia es 283 meses (23 años).
			
			& \begin{center}\includegraphics[width=1\linewidth]{NOTEBOOK/IMAGENES_DESCRIPTIVAS/17_fraction_genome_altered}\end{center}
			\\ \hline
			
			%------------------------------------------------------	
			Overall Survival Status
			& El \textit{estado de supervivencia global del paciente} se clasifica de forma nominal en dos categorías: La primera categoría corresponde al estado \textit{living} en donde se encuentran 697 pacientes que aunque padecen cáncer de mama aún viven. La segunda categoría corresponde al estado \textit{deceased} en donde se encuentran 121 pacientes que fallecieron a causa del cáncer de mama. 
			& \begin{center}\includegraphics[width=1\linewidth]{NOTEBOOK/IMAGENES_DESCRIPTIVAS/40_overall_survival_status}\end{center}
			\\ \hline
		\end{tabular}
	\end{threeparttable}
\end{table*}

\begin{table*}[!htb]
	\footnotesize
	\begin{threeparttable}
		\begin{tabular}{p{2.5cm} p{7cm} p{6.5cm}} \toprule
			%------------------------------------------------------	
			Disease Surgical Margin Status
			& El borde de tejido normal que rodea el tumor que se extrae cuando el cáncer de mama se extirpa quirúrgicamente es conocido como \textit{Margen} y sirve para mostrar si se extirpó o no todo el tumor\cite{Susan}. Dado lo anterior el \textit{Estado del margen quirúrgico de la enfermedad} se visualiza en orden descendente de la siguiente manera: En primer lugar, en el estado \textit{negativo} se encuentran 738 pacientes a los que se detecto que los bordes exteriores no contienen células cancerosas, es decir solo hay tejido normal en los bordes del tejido extraído del seno. En segundo lugar, en el estado \textit{positivo} se encuentran 58 pacientes a los que se detecto que las margenes contienen células cancerosas, por lo tanto es posible que se necesite cirugía para obtener margenes negativas. En tercer lugar, en el estado \textit{cerrado} se encuentran 22 pacientes a los que se detecto que las células cancerosas se acercan, pero no tocan el borde del tejido mamario extirpado.
			
			& \begin{center}\includegraphics[width=1\linewidth]{NOTEBOOK/IMAGENES_DESCRIPTIVAS/41_disease_surgical_margin_status}\end{center}
			\\ \hline
			
			%------------------------------------------------------	
			Variable
			& Descripcion
			
			& \begin{center}\includegraphics[width=1\linewidth]{NOTEBOOK/IMAGENES_DESCRIPTIVAS/42_primary_tumor_site}\end{center}
			\\ \hline
			
			%------------------------------------------------------	
			Variable
			& Descripcion
			& \begin{center}\includegraphics[width=1\linewidth]{NOTEBOOK/IMAGENES_DESCRIPTIVAS/19_her_2_fish_status}\end{center}
			\\ \hline
		\end{tabular}
	\end{threeparttable}
\end{table*}








 \section{Fase 5: Procesamiento y transformación de datos oncológicos}
En esta fase, se abarcan todas las actividades para construir el conjunto de datos o imágenes que se utilizará en la siguiente etapa de modelado y ejecución. Entre las actividades del procesamiento y transformación de datos oncológicos, están la limpieza de datos, combinar datos de múltiples fuentes y transformar los datos en variables de valor. En esta fase, es importante el trabajo en equipo y la comunicación continua entre el ingeniero y el científico de datos para tratar los valores no válidos o faltantes, eliminar duplicados, dar un formato adecuado y combinar archivos, tablas y plataformas. Adicionalmente, el medico experto en oncología deberá proporcionar un visto bueno para proceder con la siguiente fase. Esto dado que al ser experto en el tema de dominio tiene un conocimiento mas profundo de las variables o imágenes que esta observando, y si existiese información innecesaria para el diagnostico del cáncer de mama es posible depurar dicha información para que no afecte el entrenamiento y posterior ejecución del modelo de ML y DL.
 \section{Fase 6: Modelado y Ejecución}
En esta fase, el científico de datos diseña, crea o utiliza un modelo predictivo o descriptivo y lo alimenta con la versión del conjunto de datos o imágenes obtenidos en la fase de procesamiento y transformación. En esta fase, el científico debe seleccionar el tipo de aprendizaje (supervisado, no supervisado y por refuerzo) y la técnica determinada (regresión, clasificación, clustering, CNN, RNN, etc.) acorde a las preguntas planteadas en el \textit{BCQM}. Hay que mencionar que en esta fase el \textit{Data Analysis Team} debe definir junto al medico experto en oncología la tolerancia de error permitida en el modelo, esto dado a que la sensibilidad de los análisis puede variar dependiendo del tipo de cáncer de mama y la técnica de diagnostico. Es probable que el científico de datos pruebe múltiples algoritmos con sus respectivos parámetros para encontrar el mejor modelo para las variables oncológicas disponibles. Cabe resaltar, que es de vital importancia que los modelos propuestos no tengan problemas de sobre-ajuste o infra-ajuste ya que esto puede generar resultados erróneos o poco significativos. Adicionalmente, el científico de datos en cuestión junto al \textit{Data Analysis Team} deben definir la infraestructura a nivel de servidor necesaria para el entrenamiento y prueba del modelo según la cantidad de información a procesar, esto con el proposito de generar resultados acertados en el menor tiempo posible en pro de cumplir las tareas definidas en la fase de planeación de actividades y dar valor a los datos oncológicos una vez finalice el \textit{Release}.
 \include{FASES/FASE_7_EVALUACION}
 \section{Fase 8: Retroalimentación medica }
En esta fase, el medico experto en oncología determina si los resultados generados por el modelo de ML o DL lograron responder las preguntas planteadas en el \textit{BCQM} y si la nueva informacion obtenida es suficiente para diagnosticar el cáncer de mama o si dichos resultados generaron información relevante para determinar la causa u origen de esta enfermedad, en pocas palabras, si los datos analizados produjeron un valor agregado al dominio medico. En el caso de que los resultados obtenidos no lograsen dar valor a los datos, el \textit{Data Analysis Team} deberá decidir si es necesario re-plantear las preguntas o si se debe adquirir nuevos datos para ajustar el modelo generado. Ademas, el experto en compañía del científico y el ingeniero de datos, basado en su perspicacia medica, deberá ayudar a decidir cual estrategia es las mas apropiada para generar resultados significativos. De forma similar, si el resultado fue satisfactorio el medico debe emitir un dictamen del \textit{nivel de impacto} que tuvo la información generada por los modelos al diagnosticar el padecimiento del cáncer de mama a un determinado paciente y unas vez comprobada la informacion, junto al \textit{Data Analysis Team} alimentar un conjunto de datos con la informacion obtenida de los diagnósticos generados a cada individuo. Lo anterior con el proposito de mejorar el desempeño de los modelos existentes y aumentar su precisión. Finalmente, en cada \textit{Release} se debe garantizar que el tiempo de diagnostico sea cada vez menor o que se genere nueva información que el medico pueda utilizar en sus funciones diarias y que ayude a determinar el origen, relación o posible tratamiento de esta enfermedad.
 \include{FASES/FASE_9_BCDL}

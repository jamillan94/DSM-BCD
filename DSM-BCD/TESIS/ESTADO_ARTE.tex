\newpage
\chapter{Estado del arte y Marco Teórico}

\section{Marco teórico}

El eje central esta investigación está basada principalmente en metodologías en ciencia de datos y las técnicas que la componen, tales como:  Machine Learning (ML), Deep Learning (DL), Inteligencia Artificial (IA) y los diferentes métodos estadísticos para comprobación de la hipótesis con base a la diagnosis del cáncer de mama. Dado lo anterior, a continuación se describen los conceptos teóricos relacionados con el tema de estudio:   

\subsection{Detección y diagnóstico del cáncer de mama}
El cáncer de mama, con su causa incierta, ha capturado la atención de los cirujanos en todas las épocas. A pesar de siglos de laberintos teóricos y preguntas científicas, el cáncer de mama aún es una de las enfermedades humanas más temibles \cite{Bland2009}. Según \cite{Fatima2020}, el cáncer de mama se origina a través de tumores malignos, cuando el crecimiento de la célula se descontrola provocando que muchos tejidos grasos y fibrosos de la mama inicien un crecimiento anormal, lo cual tiene como consecuencia que las células cancerosas se diseminen por los tumores causando las diferentes etapas del cáncer. \cite{Fatima2020} exponen que existen diferentes tipos de cáncer de mama \cite{Sun2017}, que se producen cuando las células afectadas y tejidos diseminados se esparcen por todo el cuerpo. Pongamos por caso, el primer tipo de cáncer denominado \textit{Carcinoma ductal in situ (DCIS)} el cual es un tipo de cáncer no invasivo \cite{Hou2020}, que se produce cuando las células anormales se propagan fuera de la mama. El segundo tipo de cáncer es el \textit{Carcinoma ductal invasivo (IDC)}, también conocido como carcinoma ductal infiltrante \cite{Chaudhury2011}. Este tipo de cáncer ocurre cuando las células anormales de la mama se diseminan por todos los tejidos mamarios. Generalmente este tipo de cáncer se encuentra en los hombres \cite{Page1982}. El tercer tipo de cáncer es el cáncer de \textit{Tumores mixtos (MDLC)} también conocido como cáncer de mama invasivo \cite{Tuck1997} causado por las células anormales de los conductos y las células lobulillares \cite{Lee2017}. El cuarto tipo de cáncer es el cáncer de mama \textit{Lobulillar (ILC)} \cite{Masciari2007} que ocurre dentro del lóbulo mamario y aumenta las posibilidades de otros cánceres invasivos. El quinto tipo cáncer es el cáncer de mama \textit{Mucinoso (MBC)} o de \textit{mama coloide} \cite{Memis2000} que ocurre debido a las células ductales invasivas cuando los tejidos anormales se extienden alrededor del conducto \cite{Gradilone2011}. El sexto y ultimo tipo de cáncer es el cáncer de mama \textit{inflamatorio (IBC)}, el cual causa hinchazón y enrojecimiento del pecho. Este tipo de cáncer de mama es de rápido crecimiento, y comienza a aparecer cuando los vasos linfáticos se obstruyen en células rotas \cite{Robertson2010}.

Según \cite{Brunicardi2010}, en la mayoría de casos detectados la mujer descubre una tumoración en su mama. Otros signos y síntomas que se presentan menos a menudo comprenden: crecimiento o asimetría de la mama, alteraciones y retracción del pezón o telorrea, ulceración o eritema de la piel de la mama, una masa axilar y molestia musculoesquelética. Cabe señalar, que si se detectan alguno de los síntomas anteriores este tipo de cáncer puede ser diagnosticado por medio de los procedimientos basados en \textit{Exploración Física}, \textit{Técnicas de imagen} y \textit{Biopsias}. 

A nivel de \textit{Exploración física}, el cáncer de mama puede ser detectado por el oncólogo por medio de los métodos de \textit{Inspección} y \textit{Palpación}. Con estos métodos, se registran la simetría, el tamaño y la forma de la mama, así como cualquier evidencia de edema (piel de naranja), retracción del pezón o de la piel y eritema .

En la actualidad, muchas \textit{Técnicas de imagen} se utilizan ampliamente para proporcionar un diagnóstico preciso de las lesiones mamarias\cite{Tamam2021}, entre estas técnicas las mas relevantes son las siguientes: \textit{La Mamografía} que hace uso de una unidad mamográfica que consta de un tubo de rayos X que encapsula un cátodo y un ánodo. La mama se coloca sobre el detector y se comprime mediante un dispositivo de placas paralelas, el cual mantiene la mama inmóvil y evita el desenfoque por movimiento, esto con el propósito de reducir el grosor del tejido que deben atravesar los rayos x \cite{Ebrahimi2019}; \textit{La ductografía} que identifica de lesiones en pacientes con secreción del pezón. Este método es eficaz para localizar e identificar las lesiones intraductales por medio de un examen mamográfico realizado tras el llenado retrógrado de los conductos galactóforos con material de contraste \cite{Hirose2007}; \textit{La Ecografía} que permite obtener imágenes de alta resolución por medio de un pequeño transductor (sonda) de alta frecuencia que envía ondas sonoras inaudibles al interior de la mama y recibe el eco de las ondas procedentes de los órganos internos, los fluidos y los tejidos \cite{Hasan2019}; y \textit{La Resonancia Magnética (MRI)}que es utilizada cuando las lesiones en la mama no se pueden evaluar fácilmente mediante otras técnicas. Para lograrlo, utiliza bobinas receptoras de radiofrecuencia (RF) para detectar una señal emitida por los tejidos tras la excitación de un campo electromagnético que obliga a los protones alinearse a la anatomía de la zona de interés en tamaño y forma \cite{Tse2014}.

Hay que mencionar, además que actualmente existen dos modalidades para obtener un diagnóstico por \textit{Biopsia} para un paciente que presenta una anomalía mamaria. Por un lado tenemos, la biopsia para mamas con \textit{lesiones palbables} tambien denominada \textit{percutánea} o \textit{mínimamente invasiva}. Estas biopsias incluyen la aspiración con aguja fina (FNA) y con con aguja gruesa (CNB). Las biopsias quirúrgicas abiertas se denominan a veces biopsias por escisión o biopsias por incisión. La biopsia por \textit {escisión} indica la extirpación completa de la lesión, mientras que la biopsia por \textit {incisión} indica la extirpación de parte de la lesión \cite{Greenfield2012}. En el caso de las \textit{lesiones no palpables}, las modalidades de imagen como la ecografía (US), la mamografía y la resonancia magnética (MRI) son complementos útiles para identificar y localizar la lesión de interés. La decisión de cuándo realizar una biopsia de mama depende de los antecedentes del paciente, los hallazgos de la exploración física y las imágenes radiológicas. El objetivo principal de la biopsia es obtener un diagnóstico tisular que pueda ayudar a dictar el tratamiento y la planificación preoperatoria, si está indicado. Por lo tanto, es imprescindible elegir una técnica de biopsia que optimice las posibilidades de obtener un diagnóstico preciso y que, al mismo tiempo, minimice los costes, limite las molestias del paciente y reduzca la necesidad de repetir el procedimiento \cite{Samilia2018}.

\newpage
\subsection{Ciencia de Datos}
La ciencia, en el lenguaje del método científico,es:
\begin{itemize}
	\item Formular hipótesis o conjeturas sobre cómo funciona el mundo, basadas en observaciones del mundo que nos rodea.
	\item Validar o invalidar esas hipótesis mediante la realización de experimentos.
\end{itemize}                   
Sin embargo, a diferencia de las ciencias puras, trabajar con datos no requiere necesariamente realizar experimentos. Más bien, muchas veces los datos ya han sido recopilados y organizados previamente. Entonces, el método científico, aplicado a los datos, se puede resumir como: \textit{“Formular hipótesis basadas en el mundo que nos rodea y luego analizar los datos relevantes para validar o invalidar dichas hipótesis”}. 

\subsection{Machine Learning(ML)}
Aprender significa: \textit{“Adquirir conocimientos o habilidades en algo a través de la experiencia”}.  Por lo tanto, se podría enmarcar al ML cómo la manera en la cual una máquina gana o adquiere conocimiento a través de la experiencia. Pero ¿Cómo adquiere experiencia una máquina? Todas las entradas de una máquina son esencialmente cadenas binarias de 0 y 1, que en el dominio de las ciencias de la computación dichos binarios son simple y llanamente Datos. Por consiguiente, el ML es realmente la forma en que una computadora adquiere conocimiento a través de los datos.  

La ciencia de datos es fundamentalmente un proceso, mientras que el Machine Learning es una herramienta que puede ser inmensamente útil para llevar a cabo el proceso dicho proceso\cite{Pillai2020}. Por supuesto, esto no da ninguna idea del como en absoluto; simplemente se resume el proceso como algo que se hace con los datos de entrada para generar este conocimiento como salida. Para hacer una analogía matemática, el ML es una función $f$ tal que:

\begin{equation}
	Conocimiento=f(Datos)
\end{equation}

Entonces, $f$ podría ser tan mecánico como una simple función matemática. Más específicamente, el ML es mecánico en el sentido de que la forma en que estos algoritmos aprenden se basa estrictamente en principios matemáticos. Por ejemplo, la regresión lineal es un algoritmo que aprende ajustando los coeficientes de los datos de entrada para predecir mejor un valor de salida. La forma en que cambian los coeficientes se basa completamente en protocolos matemáticos (en este caso, los gradientes de los datos de entrada)\cite{Pillai2020}. Y en la práctica, esta es la esencia de la mayoría de los algoritmos comunes de ML, los cuales se clasifican en:

\subsubsection{Aprendizaje Supervisado}
Se refiere a un tipo de modelos de Machine Learning que se entrenan con un conjunto de ejemplos en los que los resultados de salida son conocidos. Los modelos aprenden de esos resultados y realizan ajustes en sus parámetros interiores para adaptarse a los datos de entrada. Una vez el modelo es entrenado adecuadamente, y los parámetros internos son coherentes con los datos de entrada y los resultados del conjunto datos de entrenamiento, el modelo podrá realizar predicciones adecuadas ante nuevos datos no procesados previamente\cite{Roman2019}. Este tipo de aprendizaje está conformado de las siguientes técnicas:

\begin{enumerate}[label=\textbf{\arabic*})]
	
	\item \textbf{Regresión}
	
	La regresión se utiliza para asignar categorías a datos sin etiquetar. En este tipo de aprendizaje tenemos un número de variables predictoras (explicativas) y una variable de respuesta continua (resultado), y se tratará de encontrar una relación entre dichas variables que nos proporcione un resultado continuo\cite{Roman2019}. Los modelos de Regresión más conocidos son los siguientes:
	\begin{enumerate}[label=\textbf{(\alph*)}]
		\item \textit{\textbf{Regresión Lineal:}}
		Se utiliza para estimar los valores reales (costo de las viviendas, el número de llamadas, ventas totales, etc.) basados en variables continuas. La idea es tratar de establecer la relación entre las variables independientes y dependientes por medio de ajustar una mejor línea recta con respecto a los puntos\cite{BriegaLopez2015}. 
		
		\item \textit{\textbf{Regresión Logística:}}
		Los modelos lineales, también pueden ser utilizados para clasificaciones; es decir, que primero ajustamos el modelo lineal a la probabilidad de que una cierta clase o categoría ocurra y, a luego, utilizamos una función para crear un umbral en el cual especificamos el resultado de una de estas clases o categorías. La función que utiliza este modelo es denominada regresión logística\cite{BriegaLopez2015}.  
	\end{enumerate}
	
	\newpage
	\item \textbf{Clasificación}
	
	La Clasificación es una sub-categoría de aprendizaje supervisado en la que el objetivo es predecir las clases categóricas con base a un conjunto de datos existentes\cite{BriegaLopez2015}. Los modelos de clasificación más conocidos son los siguientes:
	
	\begin{enumerate}[label=\textbf{(\alph*)}]
		\item \textit{\textbf{Árboles de decisión:}}
		Los árboles de decisión (DT\footnote{Decision Tree}) son diagramas con construcciones lógicas, muy similares a los sistemas de predicción basados en reglas, que sirven para representar y categorizar una serie de condiciones que ocurren de forma sucesiva, para la resolución de un problema\cite{Roman2019}. 
		
		\item \textit{\textbf{Bosque aleatorio:}}
		La idea central detrás del algoritmo de Bosque aleatorio(RF\footnote{Random Forest}) es construir una gran cantidad de árboles de decisión y luego seleccionar y comparar la categoría que cada árbol eligió\cite{Roman2019}. 
		
		\item \textit{\textbf{Vecinos más cercanos:}}
		Los vecinos mas cercanos (KNN\footnote{K-Nearest Neighborhood}) es un método de clasificación no paramétrico, que estima el valor de la probabilidad a posteriori de que un elemento $x$ pertenezca a una clase en particular a partir de la información proporcionada por un conjunto de datos .Cuando se requiere un resultado para una nueva instancia de datos, el algoritmo KNN recorre dicho conjunto para encontrar las $k$ instancias más cercanas a la nueva instancia, o el número $k$ de instancias más similares al nuevo registro, y luego genera la media de los resultados (para un problema de regresión) o la moda para un problema de clasificación\cite{Shaw2019}.
		
		\item \textit{\textbf{Máquinas de vectores de soporte:}}La maquina de vectores de soporte (SVM\footnote{Support Vector Machine}) es clasificador discriminativo definido formalmente por un hiperplano separador que dados los datos de entrenamiento conocidos genera un hiperplano óptimo que categoriza nuevos registros. En un espacio bidimensional, este hiperplano es una línea que divide un plano en dos partes, donde en cada clase se encuentra a cada lado. En otras palabras, las máquinas de vectores de soporte calculan un límite de margen máximo que conduce a una partición homogénea de todos los puntos de datos\cite{Patel2017}. 
	\end{enumerate}
	
\end{enumerate}

\subsubsection{Aprendizaje No supervisado}
En el aprendizaje No supervisado, se trabaja con datos sin etiquetar cuya estructura es desconocida. El objetivo será la extracción de información significativa, sin la referencia de variables de salida conocidas, y mediante la exploración de la estructura de dichos datos sin etiquetar\cite{BriegaLopez2015}.Este tipo de aprendizaje está conformado de las siguientes técnicas:

\begin{enumerate}[label=\textbf{\arabic*})]
	
	\item \textbf{Agrupamiento(Clustering)}
	
	El agrupamiento es una técnica exploratoria de análisis de datos, que se usa para organizar información en grupos sin tener conocimiento previo de su estructura. Cada grupo es un conjunto de objetos similares que se diferencia de los objetos de otros grupos. El objetivo es obtener un número de grupos de características similares\cite{BriegaLopez2015}. Los modelos de Clustering más conocidos son los siguientes:
	
	\begin{enumerate}[label=\textbf{(\alph*)}]
		\item \textit{\textbf{K-means :}}
		Es un algoritmo diseñado para dividir datos no etiquetados en un cierto número ($k$) de agrupaciones distintas. En otras palabras, k-means encuentra observaciones que comparten características importantes, las une y las clasifica en grupos\cite{Jeffares2019}. 
		
		\item \textit{\textbf{K-modes:}}
		El algoritmo de agrupación de k-modes es una extensión del algoritmo k-means. Este algoritmo fue diseñado para agrupar grandes conjuntos de datos categóricos, y tiene como objetivo obtener las k modas que representan a un conjunto de datos determinado\cite{Ramirez2020}.
	\end{enumerate}
	
	\item \textbf{Reducción dimensional}
	
	Es común trabajar con datos en los que cada observación se presenta con alto número de características, en otras palabras, que tienen alta dimensionalidad. Este hecho es un reto para la capacidad de procesamiento y el rendimiento computacional de los algoritmos de Machine Learning. La reducción dimensional es una de las técnicas usadas para mitigar este efecto. Este modelo funciona encontrando correlaciones entre las características, lo que implica que existe información redundante, ya que alguna característica puede explicarse parcialmente con otras. Estas técnicas eliminan ruido de los datos, y comprimen los datos en un sub-espacio más reducido, al tiempo que retienen la mayoría de la información relevante\cite{BriegaLopez2015}. El modelos de Reducción Dimensional más conocido es el siguientes:
	
	\begin{enumerate}[label=\textbf{(\alph*)}]
		\item \textit{\textbf{Análisis de componentes principales:}}
		El análisis de componentes principales (PCA\footnote{Principal Component Analysis}) se utiliza para facilitar la exploración y visualización de los datos al reducir el número de variables. Esto se hace capturando la varianza máxima en los datos en un nuevo sistema de coordenadas con ejes llamados componentes principales. Cada componente es una combinación lineal de las variables originales y es ortogonal entre sí. La ortogonalidad entre componentes indica que la correlación entre estos componentes es cero\cite{Shaw2019}.
	\end{enumerate}
	
\end{enumerate}

\subsubsection{Aprendizaje por Refuerzo}

El aprendizaje por refuerzo es una de las ramas más importantes del aprendizaje profundo. El objetivo es construir un modelo con un agente que mejora su rendimiento, basándose en la recompensa obtenida del entorno con cada interacción que se realiza. La recompensa es una medida de lo correcta que ha sido una acción para obtener un objetivo determinado. El agente utiliza esta recompensa para ajustar su comportamiento futuro, con el objetivo de obtener la recompensa máxima\cite{BriegaLopez2015}. 

\newpage
\subsection{Deep Learning(DL)}

El aprendizaje profundo o Deep Learning, es un sub-campo del ML, que usa una estructura jerárquica de Redes Neuronales Artificiales(ANN\footnote{Artificial Neural Network}), que se construyen de una forma similar a la estructura neuronal del cerebro humano. Esta arquitectura permite abordar el análisis de datos de forma no lineal\cite{BriegaLopez2015}.  Las técnicas utilizadas en el aprendizaje profundo son los siguientes:

\subsubsection{Redes Neuronales}
Al igual que el cerebro humano, las Redes Neuronales constan de neuronas. Cada neurona recibe señales como entrada, las multiplica por pesos, las suma y aplica una función no lineal. Estas neuronas se apilan una al lado de la otra y se organizan en capas\cite{Karagiannakos2020}. La primera capa de la red neuronal toma datos en bruto como entrada, los procesa, extrae información y la transfiere a la siguiente capa como salida. Este proceso se repite en las siguientes capas, cada capa procesa la información proporcionada por la capa anterior, y así sucesivamente hasta que los datos llegan a la capa final, que es donde se obtiene la predicción. Esta predicción se compara con el resultado conocido, y así por análisis inverso el modelo es capaz de aprender los factores que conducen a salidas adecuadas\cite{BriegaLopez2015}.  Las Redes Neuronales se componen de los siguientes algoritmos:

\begin{enumerate}[label=\textbf{\arabic*})]
	
	\item \textbf{Backpropagation}
	
	Las Redes Neuronales pueden aprender una función deseada utilizando grandes cantidades de datos y un algoritmo iterativo llamado \textit{Backpropagation}.  En este caso se alimenta la red con datos, se produce una salida, se compara esa salida con la deseada (usando una función de pérdida) y se reajustan los pesos en función de la diferencia\cite{Karagiannakos2020}.
	
	\item \textbf{Redes Neuronales Feedforward }
	
	Las Redes Neuronales Feedforward (FNN\footnote{Feedforward Neural Network}) suelen estar completamente conectadas , lo que significa que cada neurona de una capa está conectada con todas las demás neuronas de las siguientes capas. La estructura descrita se llama perceptrón multicapa y se originó en 1958. El perceptrón de una sola capa solo puede aprender patrones linealmente separables, pero un perceptrón multicapa es capaz de aprender relaciones no lineales entre los datos. Son excepcionalmente buenos en tareas como clasificación y regresión. A diferencia de otros algoritmos de ML, no convergen tan fácilmente. Cuantos más datos tengan, mayor será su precisión\cite{Karagiannakos2020}.
	
	\item \textbf{Redes Neuronales Convolucionales}
	
	Las Redes Neuronales Convolucionales(CNN\footnote{Convolutional Neural Network}) emplean una función llamada convolución. El funcionamiento de este tipo de redes consiste en que en lugar de conectar cada neurona con todas las siguientes, se conectan con solo con una parte de ella denominada campo receptivo. En cierto modo se intenta regularizar a las redes de retro-alimentación para evitar el sobre-ajuste. Esta característica hace que esta red sea muy buena para identificar relaciones espaciales entre los datos. Es por eso que su caso de uso principal es la visión por computadora y aplicaciones como clasificación de imágenes, reconocimiento de vídeo, y análisis de imágenes médicas\cite{Karagiannakos2020}.
	
	\item \textbf{Redes Neuronales Recurrentes}
	
	Las Redes Neuronales Recurrentes(RNN\footnote{Recurrent Neural Network}) se utilizan para datos relacionados con el tiempo. La retro-alimentación se maneja en forma de bucle desde la salida a la entrada para pasar información a la red. Por lo tanto, son capaces de recordar datos pasados y utilizar esa información en su predicción. Para lograr un mejor rendimiento, los investigadores han modificado la neurona original en estructuras más complejas, como unidades GRU\footnote{Gated Recurrent Units} y unidades LSTM\footnote{Long Short-Term Memory}. Las unidades LSTM se han utilizado ampliamente en el procesamiento del lenguaje natural en tareas como traducción de idiomas, generación de voz y síntesis de texto a voz\cite{Karagiannakos2020}.
	
	\item \textbf{Redes Neuronales Recursivas}
	
	Las Redes Neuronales Recursivas (RNR\footnote{Recursive Neural Network}) son otra forma de redes recurrentes con la diferencia de que están estructuradas en forma de árbol. Como resultado, pueden modelar estructuras jerárquicas en el conjunto de datos de entrenamiento. Se utilizan tradicionalmente en aplicaciones como la transcripción de audio a texto y el análisis de sentimientos debido a sus vínculos con árboles binarios, contextos y analizadores basados en lenguaje natural. Sin embargo, su desempeño tiende a ser mucho más lento en comparación que las redes recurrentes\cite{Karagiannakos2020}.
\end{enumerate}

\subsection{Prueba de Hipótesis}
Una prueba de hipótesis es una prueba estadística que se utiliza para determinar si hay suficiente evidencia en una muestra de datos para inferir que una determinada condición es verdadera para toda la población. Lo que hace la prueba de hipótesis es proporcionar un camino a seguir por el cual se puede probar la hipótesis planteada de manera efectiva. Una prueba de hipótesis examina dos hipótesis opuestas sobre una población: la hipótesis nula y la hipótesis alternativa\cite{Aggarwal2018}. 


\subsubsection{Hipótesis nula ($H_{0}$)}

La hipótesis nula establece que un parámetro de población es igual a un valor. La hipótesis nula es a menudo una afirmación inicial que los investigadores especifican utilizando investigaciones o conocimientos previos\cite{Aggarwal2018}.

\subsubsection{Hipótesis alternativa ($H_{1}$)}

La hipótesis alternativa establece que el parámetro de población es diferente al valor del parámetro en la hipótesis nula. La hipótesis alternativa es lo que se podría creer que es cierto o esperar que sea cierto. Según los datos de la muestra, la prueba determina si se rechaza la hipótesis nula. Utiliza un valor p para hacer la determinación. Si el valor p es menor o igual que el nivel de significancia, que es un punto de corte definido, entonces se puede rechazar la hipótesis nula\cite{Aggarwal2018}.

\newpage
\section{Marco de referencia}
El proposito de esta investigación es proponer una metodología en ciencia de datos y no una técnica de ML y DL. De acuerdo con lo anterior, en la literatura científica no existe información de metodologías para la aplicación de la ciencia de datos en el diagnostico del cáncer de mama, sin embargo se encontraron metodologías cuyas etapas pueden ser utilizadas en proyectos enfocados en las ciencias de la salud. Por otra parte, en las investigaciones analizadas se encontró bastante información de técnicas de ML y DL para el diagnostico esta enfermedad. Por esta razón, a continuación se presentan las metodologías en ciencia de datos relevantes para el tema de estudio y las técnicas en ML y DL mas destacadas en la comunidad científica.  

\subsection{Metodologías en ciencia de datos}
A continuación se encuentra una síntesis de los resultados obtenidos de cada una de las metodologías que se consideraron relevantes para la investigación. Cabe resaltar, que ninguna de las metodologías analizadas se enfoca directamente en el cáncer de mama, sin embargo hacen un gran énfasis en el entendimiento del dominio y su importancia en la definición de metas claras y alcanzables para dar valor a los datos:
\\
\cite{Schroer2021} realizan una descripción general del enfoque de la investigación, las metodologías actuales, las mejores prácticas y las posibles brechas en la ejecución de las fases de la metodología para minería de datos \textit{CRISP-DM} \footnote{Cross-Industry Standard Process for Data Mining }. Esta metodología es una innovación cruzada entre industrias de diferentes sectores. Fue desarrollada por SPSS y Teradata en 1996 y describe un enfoque que es comúnmente utilizado por expertos en minería de datos. Esta metodología presenta un proceso iterativo estructurado, bien definido y documentado que consta de seis fases iterativas: comprensión del negocio, comprensión de datos, preparación de datos, modelado, evaluación y despliegue. Adicionalmente, esta metodología se divide en tareas genéricas con objetivos específicos y resultados concretos. Estos hitos permiten la evaluación intermedia de los resultados, una eventual replanificación y una colaboración más fácil. Las tareas genéricas pretenden abarcar el mayor número posible de situaciones en la minería de datos, y se dividen a su vez en tareas o actividades específicas \cite{Mladenic2012}. En conclusión, dado que esta esta metodología abarca desde la comprensión empresarial hasta la implementación, y además se compone de un proceso fácil y estructurado, confiable, de uso común e independiente de la industria, es la metodología más popular en la práctica y en la investigación.

\cite{Safhi2019} estudiaron y evaluaron la metodología KDD \footnote{Knowledge Discovery Databases}, la cual tiene como propósito principal extraer información previamente desconocida y patrones ocultos comprensibles en los datos. Esta metodología consta de cinco etapas: selección de datos,  preprocesamiento de datos, transformación de datos , minería de datos y evaluación/interpretación. En esta metodología los datos se obtienen de múltiples fuentes, y cada una de las etapas del proceso puede ser aplicada a diferentes organizaciones. En conclusión, esta metodología se adapta al crecimiento exponencial de los datos y ayuda a que las diferentes organizaciones generen valor a dichos datos basándose en un entendimiento previo del dominio.

\cite{Shafique2014} realizaron un estudio comparativo y una descripción puntual de los aspectos más importantes de la metodología denominada SEMMA\footnote{Sample, Explore, Modify, Model, and Access}. Esta metodología fue creada por la organización SAS Enterprise Miner y permite comprender, organizar, desarrollar y mantener proyectos de minería de datos a través de un ciclo de cinco etapas: muestreo, exploración, modificación, modelado y evaluación de datos.  En conclusión, esta metodología ofrece un ciclo de vida para la solución de los problemas típicos del ámbito empresarial al tratar de cumplir objetivos que necesiten modelos predictivos y descriptivos para el análisis de grandes volúmenes de datos.

\cite{Mladenic2012} proponen una metodología denominada RAMSYS \footnote{Remote Collaborative Data Mining System}, para llevar a cabo proyectos de minería de datos a distancia de manera colaborativa . Esta metodología se compone de seis fases: gestión del negocio,  libertad para resolver problemas, empezar en cualquier momento, parar en cualquier momento, intercambio de conocimientos en línea y seguridad. Esta metodología profundiza las fases de CRISP-DM, y permite que el esfuerzo de minería de datos se invierta en diferentes ubicaciones que se comunican a través de una herramienta basada en la web. En conclusión, el objetivo de la metodología es permitir el trabajo colaborativo de científicos de datos ubicados de forma remota, de modo que se facilite el intercambio de información a distancia, así como la libertad de experimentar con cualquier técnica de resolución de problemas \cite{Martinez2021}.

\cite{Elprin2022} fundadores de Domino Data Lab en Silicon Valley, crearon la metodología Domino DS, la cual abarca el ciclo de vida de un proyecto de ciencia de datos. Esta metodología se compone de las siguientes fases: ideación, adquisición y exploración de datos, investigación y desarrollo, validación, entrega y monitoreo. Esta metodología se fundamenta en CRISP-DM, la agilidad y las necesidades del cliente para guiar a un equipo de análisis de datos hacia un mejor desempeño. En conclusión, esta metodología integra eficazmente el proceso de la ciencia de datos, la ingeniería de software y los enfoques ágiles para generar valor a los datos a través de entregas iterativas enfocadas en primera instancia en el problema comercial y después en la implementación \cite{Martinez2021}.

\cite{Larson2016} realizaron un estudio, en donde analizaron cómo los principios y prácticas ágiles han evolucionado con la inteligencia empresarial. En este estudio los autores abordaron la metodología Agile BI Delivery Framework. Esta metodología sugiere que los desafíos a los que se enfrentan los proyectos de BI hacen que el enfoque Agile sea una respuesta atractiva debido a las semejanzas que existen entre ambas, por lo tanto, sugieren que dichas metodologías pueden evolucionar en paralelo. Por el lado de la metodología BI \footnote{Business Intelligence} tenemos cinco etapas: descubrimiento, diseño, desarrollo, despliegue y entrega de valor. Dado lo anterior, los autores proponen las siguientes etapas para Agile BI Delivery Framework: alcance, adquisición de datos, análisis, desarrollo de modelos, validación e implementación. En esta metodología el alcance se centra en un marco de entrega ágil que aborda la influencia de la ciencia de datos en BI. En conclusión, el propósito de la metodología es aplicar la agilidad a los problemas comunes que se encuentran en los proyectos de BI al promover la interacción y la colaboración entre las partes interesadas, debido a que esto garantiza requisitos más claros, una comprensión de los datos, una responsabilidad conjunta y la obtención de resultados de mayor calidad. Por lo tanto, se dedica menos tiempo a intentar determinar los requisitos de información y se dedica más tiempo a descubrir conocimiento oculto en los datos.

\cite{Maass2021} plantean que el ML requiere una comprensión profunda de los dominios a los que se pueden aplicar modelos y algoritmos de aprendizaje profundo debido a que los datos determinan la funcionalidad de un sistema de información. Por lo tanto, consideran que se requiere una evaluación para determinar si los datos de entrenamiento son representativos del dominio. Dado lo anterior, afirman que el poder real de la ciencia de datos se hace evidente al aprovechar el ML en big data con el propósito de tomar decisiones que tenga un soporte solido en los datos para generar estrategias claves en la transformación digital. De lo contrario, si no se tiene una claridad absoluta del dominio podrían surgir problemas que podrían contribuir a sesgos y errores en los modelos de aprendizaje automático. Dado lo anterior, proponen una metodología de desarrollo de siete fases basada en la unión del modelado conceptual con el aprendizaje automático: Entendimiento del problema, Recopilación de datos, Ingeniería de datos, Entrenamiento del modelo, Optimización del modelo, Integración y evaluación del modelo y la Toma de decisiones analíticas. Los autores consideran que, los modelos conceptuales son un \textit{lente} a través de la cual los seres humanos obtienen una representación mental intuitiva, fácil de entender, significativa, directa y natural de un dominio. Por el contrario, el aprendizaje automático utiliza los datos como un \textit{lente} a través de la cual obtiene representaciones internas sobre las regularidades de los datos tomados de un dominio. En consecuencia, la unión del modelado conceptual con el ML contribuye entre sí a: mejorar la calidad de los modelos de aprendizaje automático mediante el uso de modelos conceptuales durante la ingeniería de datos, el entrenamiento de modelos y las pruebas de modelos y mejorar la interpretabilidad de los algoritmos de aprendizaje automático mediante el uso de modelos conceptuales. Los modelos prácticos de ML solo son útiles dentro de un dominio determinado, como juegos, decisiones comerciales, atención médica, política o educación. Cuando se integran en los sistemas de información, los modelos de ML deben seguir las leyes, las regulaciones, los valores sociales, la moral y la ética del dominio, y obedecer los requisitos derivados de los objetivos comerciales. Esto destaca la necesidad de un modelo conceptual para ayudar a transformar ideas comerciales en representaciones estructuradas y, a veces, incluso formales, que pueden utilizarse como pautas precisas para el desarrollo de software. Por lo tanto, ayudan a estructurar los procesos de pensamiento de los expertos en el dominio y los ingenieros de software.  

\cite{Grady2017} presentan un enfoque de una metodología de proceso moderno para el análisis de Big data (BDA\footnote{Big data analytic}) que se alinea con los nuevos cambios tecnológicos e implementa la agilidad en el ciclo de vida del análisis avanzado y el desarrollo de sistemas de ML, para minimizar el tiempo necesario para alcanzar un resultado deseado. La metodología propuesta, basada en el modelo de proceso de BDA, se llama Data Science Edge (DSE), la cual los autores consideran que junto con las metodologías ágiles sirve como un modelo de proceso para el descubrimiento de conocimientos en ciencia de datos. El ciclo de vida de DSE se compone de cinco fases: Planificar, Recopilar, Seleccionar, Analizar y Actuar. Para que DSE se alineara con una metodología ágil se proporcionó una guía en cada fase sobre qué actividades serían fundamentales para comenzar y generar de forma eficiente un producto mínimo viable aunque existiera la necesidad de un refinamiento para mejorar los análisis posteriores. Por consiguiente, el propósito de la metodología propuesta fue adoptar una filosofía ágil para el desarrollo del análisis, en donde los resultados obtenidos se compartan con más frecuencia para formar un circuito de retroalimentación de la opinión de las partes interesadas y utilizar esas necesidades para validar el estado actual e influir en su evolución hacia un estado final que cumpla con los requerimientos y objetivos de un dominio especifico.

\cite{Sfaxi2020} proponen una metodología denominada \textit{DECIDE\footnote{Decisional Big Data Methodology}} la cual se fundamenta en un diseño ágil basado en eventos y datos para proyectos decisionales de Big Data. El propósito de esta metodología es ayudar a las organizaciones a determinar los objetivos comerciales y analíticos deseados según la toma de decisiones con base al análisis de datos, en donde se debe: definir una estrategia de datos clara, encontrar a las personas adecuadas para llevar a cabo un cambio cultural impulsado por los datos y seguir la ética de la información. Esta metodología se basa en la metodología DSRM\footnote{Design Science Research Methodology} que permite realizar investigaciones en ciencias del diseño en sistemas de información y ayuda a definir un modelo de proceso a seguir para diseñar una solución de sistemas de información, dependiendo del enfoque realizado para encontrar dicha solución. La metodología DSRM define principalmente cuatro enfoques posibles: centrado en el problema, centrado en objetivos, centrado en el diseño y desarrollo, e impulsado por cliente y contexto. En el caso de la metodología DECIDE se utiliza un enfoque centrado en el problema, que consta de las siguientes fases: identificación y motivación del problema, definición de los objetivos de la solución, diseño y desarrollo, demostración, evaluación y comunicación. Cabe resaltar que el valor de esta metodología está basado en cinco fundamentos claves en la fase de diseño y desarrollo: Agilidad, Enfoque Bottom-up, Datos y eventos, Multi-arquitecturas y Multi-tecnologías. En conclusión, esta metodología está diseñada para respetar los conceptos y las mejores prácticas para la toma de decisiones soportada en Big data y para ser aplicada a grandes proyectos, con un tamaño de equipo significativo. 

\cite{Xu2021} aseguran que la ciencia de datos constituye la tecnología central del análisis y procesamiento de Big data que contiene un enfoque eficaz para dar valor a los datos generando oportunidades sin precedentes en al menos cuatro aspectos: la innovación en la gestión, el desarrollo industrial, el descubrimiento científico y el desarrollo de disciplinas. Además, consideran que en la ciencia de datos, la descripción de dichos datos necesita modelado, en donde el modelado es entendido como la formalización de objetos, objetivos y métodos de procesamiento. Por otra parte, consideran que el análisis es un proceso de juzgar las propiedades teóricas de la viabilidad, precisión, complejidad y eficiencia para realizar transformación de los datos en información, la información en conocimiento y  el conocimiento en toma de decisiones, por lo tanto, para dar valor a los datos una metodología debe cumplir las siguientes fases: recopilación, convergencia, almacenamiento, administración, consulta, clasificación, extracción, análisis  y la realización de descubrimientos científicos. Por consiguiente, los autores consideran que una metodología en ciencia de datos se resume como un híbrido de modelado, análisis, cálculo y aprendizaje, en donde el objetivo fundamental es realizar la cognición y el control del mundo real a través de la transformación de los datos.

\cite{Pacheco2014} basados en la ciencia de datos, mejoran la fase de preparación y tratamiento de datos de una metodología para el desarrollo de aplicaciones de Minería de Datos(MD) basados en el análisis organizacional, denominada \textit{MIDANO}. Esta metodología consta de tres fases: conocimiento de la organización,  preparación y tratamiento de los datos y el desarrollo de herramientas de MD. Todas las fases y actividades de esta metodología pretenden abarcar el dominio de conocimiento que puede encontrarse en una organización, por lo que es necesario tener los datos integrados en una sola vista, la cual normalmente es conocida como \textit{Vista Minable}, que está compuesta por una tabla con todas las variables del proceso y los datos a considerar en el estudio de MD. Sin embargo, en las metodologías actuales no muestran cómo lograr una vista minable adecuada, es por esto que los autores para mejorar el proceso definen dos tipos de vista minable: Vista Minable Conceptual (VMC), la cual describe en detalle cada una de las variables a tomar en cuenta para la tarea de MD  y una Vista Mineable Operativa (VMO) que surge como el resultado de cargar los datos del historial y de realizar la etapa de tratamiento de datos. Tanto en la VMC, como en la VMO, se identifican determinadas variables llamadas \textit{variables objetivo} que permiten la consecución de los objetivos de MD, ya que las mismas son las que se desean predecir, clasificar, calcular, inferir según el dominio del problema de estudio. En conclusión, los autores utilizaron el conjunto de formalismos teóricos, métodos y técnicas, para el tratamiento de grandes volúmenes de datos ofrecidos por las metodologías en ciencia de datos, para detallar las variables relevantes de diversos problemas de estudio, a partir de escenarios futuros definidos en el dominio de la organización.

\cite{Costa2020} proponen una metodología orientada a procesos para ayudar a la gestión de proyectos de ciencia de datos denominada \textit{POST-DS}\footnote{Process Organization and Scheduling electing Tools for Data Science}. Esta metodología describe la secuencia de actividades realizadas en un proyecto de ciencia de datos y se basa en el ciclo de vida de la metodología CRISP-DM la cual proporciona un plan completo para realizar un proyecto de minería de datos basada en las siguientes fases: comprensión del negocio, comprensión de los datos, preparación de los datos, modelado, evaluación e implementación. La metodología POST-DS está inspirada particularmente en la metodología CRISP-DM, pero con la diferencia de que permite la identificación de procesos, organización, programación y herramientas para la gestión de proyectos de ciencia de datos a través de componentes específicos para: la asignación de roles, el ajuste de expectativas, la definición del alcance del proyecto, los costos y el tiempo. Para lograrlo, esta metodología consta de las siguientes fases establecidas a través de un cronograma base: comprensión del negocio, comprensión de los datos, preparación de los datos, modelado, evaluación y despliegue. En conclusión, esta metodología ejecuta cada una de las fases tradicionales de un proceso en ciencia de datos teniendo como eje cada una de las fases necesarias para la gestión de proyectos.

\cite{Watson2017} aseguran que la variedad de sistemas actuales de análisis de datos comerciales y de código abierto difiere significativamente, en términos de características disponibles, funcionalidad y escalabilidad, de los sistemas de análisis de datos que respaldan el flujo de trabajo de la ciencia de datos. Los autores consideran, que se puede utilizar un punto de referencia (benchmark) para evaluar la funcionalidad y el rendimiento de un sistema. Sin embargo, afirman que no existe un punto de referencia estándar para evaluar la capacidad de estos sistemas para hacer ciencia de datos, aunque existan categorías de benchmark que están asociadas, tales como: el benchmark de bases de datos relacionales, el benchmark de sistemas de Big data y el benchmark de análisis específicos del dominio. Por esta razón, los autores presentan una metodología denominada \textit{Sanzu}, la cual permite evaluar sistemas que ejecuten tareas de procesamiento y análisis de datos. Esta metodología se basa en el flujo de trabajo tradicional de las ciencias de datos, el cual se compone de las siguientes fases: recopilación de datos, manipulación de datos, análisis estadístico, retroalimentación de resultados y toma de decisiones. Adicionalmente, esta metodología se compone de dos tipos de enfoques: el enfoque \textit{micro-benchmark} que consta de seis fases: entrada y salida de archivos planos, mantenimiento de datos, estadísticas descriptivas, estadísticas de distribución e inferenciales, análisis de series de tiempo y ML. Este enfoque está destinado a probar las tareas básicas en múltiples sistemas de análisis de datos. Dichas tareas son comunes y se utilizan todos los días para resolver problemas del mundo real y de la industria. Por otra parte, el enfoque \textit{macro-benchmark} tiene como objetivo modelar las aplicaciones cuyos dominios ejecutan el flujo de trabajo de la ciencia de datos debido al crecimiento del volumen de datos y evalúa el desempeño de los sistemas de datos de cada una. En conclusión, la metodología Sanzu sirve como punto de referencia en ciencia de datos para evaluar el rendimiento de las operaciones individuales que impactan en el análisis de datos y para representar casos de uso del mundo real.

\cite{Microsoft2022} propone una metodología llamada TDSP\footnote{Team Data Science Process}, la cual permite ejecutar soluciones ágiles e iterativas en el análisis predictivo y el desarrollo de aplicaciones inteligentes de manera eficiente. Esta metodología incluye las mejores prácticas y estructuras de Microsoft y otros líderes de la industria para ayudar a lograr una implementación exitosa de iniciativas de ciencia de datos teniendo como base el trabajo en equipo y los criterios relevantes del manifiesto ágil. El objetivo es ayudar a que las empresas aprovechen al máximo los beneficios del análisis de datos. Dicho lo anterior, la metodología TDSP proporciona un ciclo de vida que se compone de cinco etapas principales que se ejecutan de forma iterativa: comprensión del negocio, adquisición y compresión de datos, modelado, despliegue y aceptación de cliente. Hay que mencionar, que TDSP está sujeta al uso de la herramienta Microsoft Azure DevOps para realizar todo el proceso de ciencia de datos desde el entendimiento del negocio hasta la entrega de un producto de valor a un cliente especifico. En conclusión, el ciclo de vida de TDSP fue diseñado para proyectos destinados a generar aplicaciones inteligentes en donde el análisis predictivo se realiza a través de modelos de ML e AI. Sin embargo, puede ser utilizado para llevar a cabo proyectos en donde el propósito es tomar decisiones según el análisis obtenido en el reconocimiento de patrones de un conjunto de datos específicos.

\cite{Saltz2019} proponen un marco ágil para la ciencia de datos denominado SKI\footnote{Structured Kanban Iteration}. Esta metodología adopta la filosofía de tableros Kanban para proporcionar un proceso de iteración estructurado para que los equipos de análisis de datos exploren y aprendan de manera incremental a través de pruebas de hipótesis. En general, en esta metodología la agilidad está basada en una secuencia de ciclos iterativos de experimentación y adaptación a través de la implementación y el análisis de los resultados. Los equipos de SKI usan un tablero visual y se enfocan en trabajar en un elemento específico durante una iteración, que se basa en tareas, no en bloques de tiempo. Por lo tanto, una iteración se alinea más estrechamente con el concepto de extraer tareas de manera prioritaria, cuando el equipo tiene capacidad. Cada iteración puede verse como la validación o el rechazo de una hipótesis específica. A nivel general, una iteración se compone de tres etapas principales: Crear, Observar y Analizar. Las etapas anteriores se enfocan en garantizar que el trabajo que se requiere para la recopilación y el análisis de datos se incorpora directamente a las tareas del equipo para una iteración determinada. En conclusión, en comparación con Scrum, SKI define una iteración centrada en la capacidad y no en el tiempo, para brindarle a un equipo de análisis de datos la ejecución de pequeñas iteraciones lógicas con una duración desconocida.

\cite{Lei2020} proponen una metodología que realiza la unión de Scrum y Kanban (Scrumban) para la investigación ágil en el cuidado de la salud. La metodología propuesta consta de cuatro fases: Definición de preguntas clínicas, adquisición y validación de datos, desarrollo del modelo predictivo y retroalimentación médica. Los autores consideran que para que se cumpla el enfoque ágil en esta metodología debe existir una colaboración continua entre científicos de datos y médicos. Además, debe existir almacenamiento y computación basados en la nube para proporcionar una plataforma común para acceder a los datos y modelos. Los problemas complejos se pueden dividir en tareas que pueden visualizar tanto los científicos de datos como los médicos, lo que les permite comprender mejor el trabajo que los científicos de datos deben realizar dentro de cada ciclo del Sprint. La metodología fomenta el despliegue continuo de modelos predictivos en entornos clínicos durante el cual los científicos de datos pueden reunirse con los médicos y recibir comentarios sobre el rendimiento del modelo. Además, la perspicacia del médico se puede aprovechar para determinar si los resultados del modelo son creíbles. En conclusión, esta metodología prescribe un proceso rápido y de mejora continua que permite a los médicos comprender el trabajo de los científicos de datos y evaluar regularmente el rendimiento de un modelo predictivo en entornos clínicos.

\cite{Rollins2015} propone la metodología IBM Foundational Methodology for Data Science, la cual tiene algunas similitudes con las metodologías más utilizadas en minería de datos, con la diferencia que se enfatiza en las prácticas más recientes en la ciencia de datos, como el uso de grandes volúmenes de datos, la incorporación de la analítica de texto en el modelado predictivo y la automatización de procesos. La metodología consta de diez etapas: comprensión del negocio, enfoque analítico, requisitos de datos, recopilación de datos, comprensión de datos, preparación de datos, modelado, evaluación, implementación y retroalimentación. Estas etapas forman un proceso iterativo para el uso de datos con el propósito descubrir conocimiento oculto. En conclusión, esta metodología ilustra la naturaleza iterativa del proceso de resolución de problemas en donde, los científicos de datos vuelven frecuentemente a etapas previas para realizar ajustes a medida que van aprendiendo de los datos y el modelado.  

\cite{Dastgerdi2021} realizaron una revisión de diversas metodologías utilizadas en análisis de datos, en donde resaltan el ciclo de vida de la ciencia de datos basado en el DataOps Manifesto. Esta metodología ayuda a caracterizar qué prácticas ágiles, eventos, artefactos y roles son valiosos para agregar valor a los datos de la organización diariamente. Esta metodología se compone de cinco etapas generales: ideación, iniciación, investigación/desarrollo, transición/producción y Retirada. Estas etapas ayudan a los equipos de análisis de datos a ser más adaptables y colaborativos en los ciclos de retroalimentación para lograr resultados de manera eficiente. En conclusión, esta metodología se basa en la experiencia laboral de varias organizaciones y los problemas tradicionales al momento de entregar tiempos de ciclo rápidos para una alta gama de análisis de datos con un manifiesto de calidad válido.

\cite{Chen2016} proponen una metodología denominada AABA\footnote{Architecture-centric Agile Big data Analytics} para el análisis ágil de Big data centrado en la arquitectura. Esta metodología consta de seis etapas: catálogo de diseño conceptual, arquitectura BDD\footnote{Big Data system Design}, implementación, pruebas, despliegue/retroalimentación y descubrimiento del valor. Esta metodología se basa en el desempeño de la arquitectura de software como factor clave de agilidad. AABA proporciona una base para el descubrimiento de valor en los datos con las partes interesadas y abarca aspectos importantes como: la planificación, la estimación, el coste , el calendario, el apoyo a la experimentación y el uso de la metodología DevOps para la entrega rápida y continua de un producto de valor. En conclusión, esta metodología se centra con la colaboración estrecha entre los científicos de datos, las partes interesadas, el arquitecto de software y otros ingenieros clave como los diseñadores de bases de datos, esto con el propósito de determinar la propuesta de valor para el sistema que se está construyendo en función de de la mejora continua y el cumplimiento de los objetivos comerciales.

\cite{Li2016} proponen una mejora para la metodología KDD denominada KDDA\footnote{Knowledge Discovery via Data Analytics}. Esta metodología se basa en el descubrimiento de conocimiento a través del análisis de datos y no de la minería de datos. Para lograrlo consta de un proceso de concha de caracol, el cual se compone de ochos fases: formulación del problema, comprensión empresarial, comprensión de datos, preparación de datos, modelado, evaluación ,despliegue y mantenimiento. Esta metodología hereda la representación del ciclo de vida de la metodología CRISP-DM, con la diferencia de que no tiene secuencias estrictamente definidas entre las fases. Cada fase incluye diferentes tareas, y el resultado de cada tarea determina la fase o tareas particulares de una fase a realizar, lo que permite que un resultado específico de la fase de modelado puede requerir volver a la comprensión del negocio, la comprensión de los datos, la preparación de los datos o ir directamente a la evaluación. En conclusión, esta metodología utiliza las prácticas de descubrimiento de conocimiento en el entorno analítico de una organización abarcando no solo conocimientos técnicos de TI, técnicas analíticas y algoritmos matemáticos, sino también una comprensión profunda del proceso empresarial.


\newpage
\subsection{Técnicas del ML y DL}
Las siguientes investigaciones basadas en técnicas de ML y DL fueron seleccionadas por su relevancia y aporte para el diagnostico del cáncer de mama debido a que brindan una gran cantidad de información y son un punto de partida importante para aplicar la ciencia de datos con dicho fin. Las investigaciones seleccionadas como referentes a nivel de técnicas se muestran a continuación:
\\\\
\cite{Alakwaa2018} realizaron la identificación de subtipos moleculares sobre datos metabolómicos por medio de algoritmos de DL y ML para determinar el pronóstico del cáncer de mama y la selección terapéutica. Para ello, la investigación se basa en la clasificación de dicho cáncer en cuatro subtipos moleculares: Luminal A ($ER +, PR \pm$ y $HER2-$), Luminal B ($ER +, PR \pm,$ y $HER2\pm$), $HER2-$ enriquecido ($ER-, PR-$ y $HER2 +$) y triple negativo ($ER-, PR-$ y $HER2 -$). Estos subtipos son generados a partir de las proteínas de factor de crecimiento epidérmico receptor 2 (HER2), receptor de progesterona (PR) y receptor de estrógeno (ER) encargadas de estimular el crecimiento y la diferenciación celular. Los resultados de supervivencia difieren significativamente entre estos subtipos. Los subtipos luminales A y B tienen un pronóstico benigno, sin embargo, los tumores triple negativos y los tumores $HER2$ tienen un pronóstico maligno. Para la investigación se utilizó un data-set conformado por 271 muestras de cáncer de mama ($204$ $ER+ $ y $67$ $ER- $) del Departamento de Patología del Hospital Charité. Posterior a eso para complementar dicho data-set se midieron un total de 162 metabolitos con estructura química conocida mediante cromatografía de gases para todas las muestras de tejido. Para realizar la clasificación de los datos metabólicos obtenidos se utilizó una Red Neuronal FNN\footnote{Feedforward Neural Network} y los siguientes modelos de ML: RF, SVM, RPART\footnote{Recursive Partitioning}, LDA\footnote{Linear Discriminant Analysis}, PAM\footnote{Microarray Prediction Analysis} y GBM\footnote{Gradient Boosting Machine}. Posteriormente, se realizó una validación cruzada (Cross-validation) con el 80\% de los datos de entrenamiento y se probaron los modelos con el 20\% de los datos restantes. Para finalizar, se utilizaron las medidas AUC\footnote{Area Down The Curve} y ROC\footnote{Receiver Operating Characteristic Curve} para evaluar el rendimiento general de los dichos modelos. En conclusión, esta investigación determino que las FNN muestran una precisión de predicción (AUC = $0,93$) mayor sobre los métodos de ML al realizar la clasificación del cáncer de mama en los subtipos moleculares basados en el receptor de estrógeno(ER).
\\\\
\cite{Nateghi2021} propusieron un sistema basado en métodos de DL y ML para la predicción de la proliferación tumoral a partir de imágenes WSI \footnote{ Whole Slide Images } obtenidas del recuento mitótico. La proliferación tumoral, que se correlaciona con el grado del tumor, es un biomarcador crucial que indica el pronóstico de las pacientes con cáncer de mama. El método más utilizado para predecir la velocidad de proliferación tumoral es el recuento de figuras mitóticas en los portaobjetos histológicos de hematoxilina y eosina ($H\&E$). El recuento manual de mitosis consiste en identificar células tumorales implicadas en el proceso anormal de la división celular. Para la investigación se utilizó el data-set proporcionado por el desafío de evaluación de la proliferación tumoral (TUPAC16) en donde el objetivo es evaluar algoritmos que predicen las puntuaciones de proliferación tumoral a partir de imágenes WSI. Primero, al considerar el tejido epitelial como regiones de actividad de mitosis, los investigadores propusieron un método de detección de regiones de interés basado en DL para seleccionar las regiones de alta actividad de mitosis a partir de las imágenes proporcionadas en el desafío. Posteriormente, se entrenó un conjunto de redes neuronales para detectar la propagación de la mitosis en distintas áreas seleccionadas. Para finalizar, los investigadores entrenaron el modelo SVM para predecir la puntuación final de proliferación tumoral. La precisión del algoritmo entrenado logró una medida-F del 73,81\% y un coeficiente kappa ponderado de 0,612, respectivamente, superando significativamente otros algoritmos que participaron en el desafío TUPAC16. Como conclusión de la investigación, los resultados experimentales demuestran que el algoritmo propuesto basado en el paradigma de DL y ML mejoró considerablemente la precisión de la predicción de la proliferación tumoral y proporciono un modelo con una precisión confiable para respaldar la toma de decisiones a nivel médico.

\cite{McKinney2020} desarrollaron un modelo de DL para identificar el padecimiento del cáncer de mama mediante imágenes mamográficas obtenidas en el Reino Unido y EE. UU. La mamografía tiene como objetivo identificar el cáncer de mama en las primeras etapas de la enfermedad, cuando el tratamiento puede tener más éxito. A pesar de la existencia de diversas técnicas de detección del cáncer de mama en todo el mundo, la interpretación de las mamografías se ve afectada por altas tasas de falsos positivos y falsos negativos. El data-set del Reino Unido consistió en mamografías recopiladas de 25,856 mujeres entre 2012 y 2015 en dos centros médicos en Inglaterra, donde las mujeres se examinan cada tres años. Incluyó a 785 mujeres a las que se les realizó una biopsia y a 414 mujeres con un diagnóstico maligno del cáncer dentro de los 39 meses posteriores a la obtención de imágenes. El data-set de EE. UU consistió en mamografías que se recolectaron entre 2001 y 2018 de 3.097 mujeres en un centro médico académico. Se incluyeron imágenes de las 1511 mujeres a las que se les realizó una biopsia durante este período de tiempo y de un subconjunto aleatorio de mujeres que nunca se sometieron a una biopsia. Entre las mujeres que recibieron una biopsia, 686 fueron diagnosticadas con cáncer dentro de los 27 meses posteriores a la obtención de las imágenes. Los casos que fueron diagnosticados como positivos para cáncer fueron acompañados de un diagnóstico confirmado por biopsia dentro del período de seguimiento. una reducción absoluta del 5,7\% y 1,2\% (EE.UU. Y Reino Unido) en falsos positivos y del 9,4\% y 2,7\% en falsos negativos. En la investigación los data-set obtenidos no se utilizaron para entrenar ni ajustar el modelo basado IA. Como resultados, la investigación proporciono evidencia de la capacidad del modelo basado en IA y DL para generalizarse desde el Reino Unido a los EE.UU. En un estudio independiente de seis radiólogos, el sistema de IA superó el diagnóstico obtenido por dichos radiólogos con una precisión (AUC-ROC) mayor que la generada por el radiólogo promedio. Además, para comprobar la predicción del modelo los investigadores realizaron una simulación en la que el sistema de IA participó en el proceso de doble lectura de diagnóstico que se utiliza en el Reino Unido y se descubrió que el sistema de IA mantuvo un rendimiento elevado y redujo la carga de trabajo del segundo lector en un 88\%. Como conclusión, esta investigación demostró que las técnicas de IA mejoran la precisión y la eficiencia de las técnicas tradicionales para la detección del cáncer de mama.


\cite{Mullooly2019} utilizaron técnicas de DL para identificar correlaciones histológicas en biopsias de la densidad mamaria obtenidas por técnicas radiológicas para determinar el cáncer de mama entre mujeres con una densidad del tejido fibroglandular alto o bajo. La densidad de la mama es una característica radiológica que refleja el contenido de tejido fibroglandular en relación con el área o el volumen de la mama y es considerada como un factor de riesgo de padecer cáncer. En la investigación se evaluaron imágenes digitalizadas teñidas con hematoxilina y eosina ($H\&E$) de biopsias de 852 pacientes. La densidad mamaria se evaluó como volumen fibroglandular global y localizado. Posteriormente, los investigadores modelaron una Red Neuronal CNN para caracterizar la composición de $H\&E$. En total, se extrajeron 37 características de la salida de la red, que describen las cantidades de tejido y la estructura morfológica. Se entrenó un modelo de regresión por Bosque Aleatorio (RF) para identificar correlaciones predictivas del volumen fibroglandular en 588 pacientes. Se evaluaron las correlaciones entre el volumen fibroglandular predicho y cuantificado radiológicamente en 264 pacientes independientes. Se entrenó un segundo clasificador de bosque aleatorio (RF) para predecir el diagnóstico invasivo vs. el diagnóstico benigno; el rendimiento de los algoritmos se evaluó utilizando el área bajo la curva (AUC). Utilizando características extraídas, los modelos de regresión predijeron el volumen fibroglandular global con una precisión de 0,94 y localizado con una precisión de 0,93 teniendo como base el contenido del estroma graso y no graso representando las correlaciones más fuertes seguidas de la cantidad de tejido epitelial. Para predecir el cáncer entre el volumen fibroglandular elevado y bajo, el clasificador logró un valor AUC de 0,92 y 0,84, respectivamente, siendo las características de tejido epitelial las más importantes. Estos resultados sugieren que el estroma no graso, la cantidad de tejido graso y el tejido epitelial predicen el volumen fibroglandular. Como conclusión, la investigación demostró que los modelos de DL y ML permiten identificar correlaciones histológicas con riesgo de cáncer en pacientes con alta y baja densidad de tejido fibroglandular.

\cite{Hu2020} desarrollaron una técnica para el diagnóstico asistido por computadora CAD\footnote{Computer-Aided Diagnosis} bajo el paradigma de aprendizaje por transferencia profunda para realizar la diagnosis del cáncer de mama utilizando mpMRI\footnote{Multiparametric Magnetic Resonance Imaging}.El estudio incluyó imágenes clínicas de resonancia magnética de 927 lesiones únicas de 616 mujeres, en donde se excluyeron las imágenes de lesiones que no presentaban una lesión visible, lesiones que no tenían validación de diagnóstico final, o lesiones que no pudieron ser claramente asignadas en a la categoría benigna ni maligna. Cada estudio de resonancia magnética RM\footnote{Magnetic Resonance}, incluyó una secuencia con contraste dinámico DCE\footnote{Dynamic Contrast-Enhanced} y una imagen de RM ponderada en un radio T2w\footnote{T2 weighted image}. Se utilizó una Red Neuronal CNN previamente entrenada para extraer características de las imágenes DCE y T2w, y se entrenaron modelos SVM en las características de CNN para distinguir entre lesiones benignas y malignas. La información de las imágenes de resonancia magnética con contraste dinámico (DCE) y ponderada en T2w se añadió al modelo IA de tres formas diferentes: fusión de imágenes, fusión de características y fusión de clasificadores. La fusión de imágenes, se utilizó para crear una imagen compuesta RGB con base a las imágenes DCE y T2w. La fusión de características, para combinar las características generadas por las redes neuronales convolucionales con base a las imágenes DCE y T2w para posteriormente ser utilizadas como entrada en un algoritmo de máquina de vectores de soporte (SVM). La fusión clasificadora se utilizó para calcular la probabilidad de malignidad por medio de voto suave (soft-voting) con base al resultado obtenido por el algoritmo SVM al analizar las imágenes DCE y T2w. El rendimiento de los modelos de DL se evaluó utilizando la curva (ROC) y fue comparado utilizando la prueba DeLong. El método de fusión de características superó estadísticamente de manera significativa a los demás métodos. En conclusión, el método propuesto de aprendizaje de transferencia profunda CAD para mpMRI pudo mejorar el diagnóstico del cáncer de mama al reducir la tasa de falsos positivos y mejorar el valor predictivo positivo en la interpretación de imágenes obtenidas por resonancia magnética.
\\\\\\
\cite{Shia2021}, propusieron un método de ML y diseño de una metodología para el análisis de la clasificación de tumores de mama benignos y malignos en imágenes de ultrasonido sin necesidad de un procesamiento de selección regional tumoral a priori.  Las imágenes se recopilaron desde el 1 de enero de 2017 hasta el 31 de diciembre de 2018. En total, se examinaron 370 masas benignas y 418 malignas, y se estudiaron a 677 pacientes en este estudio. Los criterios de exclusión para pacientes con tumores benignos incluyeron tipos de tejido que se asociaron con las siguientes afecciones: inflamación (incluida la autoinflamación, inflamación crónica e inflamación xantogranulomatosa), abscesos y dermatitis espongiótica. Para los pacientes con tumores malignos, los criterios de exclusión incluyeron casos con tipos de tejido incompletos, clasificación de categoría BI-RADS\footnote{Breast Imaging Reporting and Data System} desconocida o informes de imágenes incompletas. Las edades de los pacientes oscilaron entre los 35 y los 75 años.  El data-set recopilado constaba de 677 imágenes (Benignas: 312, maligno: 365) obtenidas de diferentes centros oncológicos de Estados Unidos. Una vez generado el data-set de estudio se procedió a realizar la extracción de características de las imágenes a través del método de histograma de gradientes orientados HOG\footnote{Histogram of Oriente Gradients} y el método histograma piramidal de gradientes orientados PHOG\footnote{Pyramid Histogram of Orientad Gradients}. Adicionalmente, se utilizó el método FAST\footnote{Features from the Accelerated Segment Test} para determinar si se podían extraer características de clasificación importantes de las imágenes preliminares. El HOG se utilizó para extraer información útil y descartar información redundante para simplificar la clasificación de imágenes calculando y contando histogramas de gradiente de áreas específicas. La distancia entre dos descriptores de imágenes PHOG refleja la medida en que las imágenes contienen formas similares y corresponde a sus diseños espaciales y el método FAST permite un rendimiento mayor que otros métodos en la detección de esquinas de los mamogramas para extraer puntos característicos y luego rastrear y mapear objetos con características relevantes en dichas imágenes. El siguiente paso fue realizar la selección de características por medio de la técnica de selección de características basada en correlación CFS\footnote{Correlation Based Feature Selection} para evaluar los atributos importantes, y crear un subconjunto de datos considerando las habilidades predictivas junto con el grado de redundancia. La función de evaluación se utilizó para evaluar subconjuntos que contenían características que estaban altamente correlacionadas con la clase y no estaban correlacionadas entre sí. Se ignoraron las características irrelevantes y se excluyeron las características redundantes. Finalmente, se realizó la clasificación de características con base en la combinación de aprendizaje ponderado localmente LWL\footnote{Local Weighted Learning }, la optimización secuencial mínima SMO\footnote{Sequential Minimal Optimization} y el método KNN.  Para la evaluación del desempeño de la clasificación, se utilizó la técnica de validación cruzada para determinar el porcentaje de error, la media, la desviación estándar y el intervalo de confianza para los algoritmos de referencia.  La precisión diagnóstica se estimó comparando las curvas AUC y ROC con la prueba no paramétrica de DeLong. El rendimiento de clasificación del conjunto de datos de imágenes mostró una sensibilidad del 81,64\% y una especificidad del 87,76\% para imágenes malignas con AUC de 0,847. Los valores predictivos positivos y negativos fueron 84,1 y 85,8\%, respectivamente. En conclusión, la comparación de los diagnósticos médicos y el diagnóstico generado por el modelo de ML propuesto arrojó resultados similares. Esto indica la posible aplicabilidad del ML en la generación de diagnósticos clínicos.

\cite{Shen2019} desarrollaron un algoritmo de DL que puede detectar con precisión el cáncer de mama en mamografías usando un entrenamiento e2e \footnote{end-to-end} que aprovecha de manera eficiente los data-sets de imágenes mamográficas obtenidas a través de diagnósticos asistidos por ordenador CAD . En este enfoque las anotaciones para identificación de lesiones en la mama obtenidas de las imágenes mamográficas solo se utilizaron en la etapa de entrenamiento inicial, y las etapas posteriores requirieron solo las características obtenidas a nivel de imagen, eliminando la dependencia de diagnósticos obtenidos de estas anotaciones. El algoritmo generado fue testeado con dos data-set de prueba. El primer data-set fue obtenido de la base de datos de mamografías digitales del sitio web CBIS-DDSM. Este data-set almacenaba 2478 imágenes de mamografías de 1249 mujeres e incluía vistas tanto craneocaudales (CC) como oblicuas mediolaterales (MLO), además contaba con etiquetas confirmadas patológicamente en una categoría benigna o maligna para la mayoría de los exámenes. Con este data-set, el modelo logró una predicción con un valor AUC de 0,88 y el promedio de cuatro modelos entrenados posteriormente mejoró esta predicción a valor AUC de 0,91 con una sensibilidad de 86,1\% y especificidad de 80,1\%. El segundo data-set fue obtenido de la base de datos de mamografías digitales INbreast FFDM\footnote{Full-Field Digital Mammography} la cual contenía información de 115 pacientes y 410 mamografías, este data-set también incluía vistas CC y MLO. Con el segundo data-set el modelo logró una predicción con valor AUC de 0,95, y el promedio de cuatro modelos entrenados mejoró el AUC a 0,98 con una sensibilidad de 86,7\% y especificidad de 96,1\%. La investigación demostró que el entrenamiento generado en una Red Neuronal CCN con el enfoque e2e CBIS-DDSM se puede transferir a imágenes FFDM de INbreast utilizando solo un subconjunto de datos para ajustar el modelo y sin depender de la disponibilidad de anotaciones de lesiones. Estos hallazgos demostraron que los métodos de DL se pueden utilizar para lograr una alta precisión de diagnosis en mamografía heterogéneas y es muy prometedor para mejorar las herramientas clínicas para reducir la detección de falsos positivos y falsos negativos de cáncer de mama.

\cite{Naik2020} utilizaron el paradigma de DL con base al diagnóstico del cáncer de mama a partir del estado molecular del receptor de Estrógeno ERS\footnote{Estrogen Receptor Status},determinado por los patólogos a partir de la tinción inmunohistoquímica IHC\footnote{Inspection Immunohistochemistry} que destaca la presencia de antígenos de superficie celular del tejido obtenido a través de una biopsia. Para lograrlo, se recopilaron imágenes WSI teñidas con hematoxilina y eosina (H\&E) del data-set obtenido del Banco Australiano de tejidos para el cáncer de mama ABCTB\footnote{Australian Breast cáncer Tissue Bank}, que contenía 2535 imágenes de exámenes realizados a pacientes y el data-set obtenido del Atlas genómico del cáncer TCGA\footnote{The Cancer Genomic Atlas}, que contenía 1014 imágenes de 939 pacientes. Ambos conjuntos de datos informaban el estado de la presencia de receptor de Estrógeno (ER), receptor de Progesterona (PR) y el de factor de crecimiento epidérmico receptor 2 (HER2) determinado por patólogos encargados de la inspección visual de los portaobjetos teñidos mediante IHC. Los WSI se escanearon a una resolución de 20x o superior. Posteriormente, se utilizó el modelo de aprendizaje de instancias múltiples MIL\footnote{Multiple Instance Learning}, el cual fue entrenado usando tinciones H\&S y anotaciones IHC como etiquetas de datos de entrada. El modelo MIL se utilizó para estimar el ERS a partir de mosaicos seleccionados al azar de un WSI. Para lograr la interpretabilidad al localizar mosaicos discriminativos en imágenes de H\&E e identificar características histomorfológicas que se correlacionan con el crecimiento celular impulsado por hormonas, los investigadores diseñaron una arquitectura bajo el paradigma de redes neuronales profundas llamada \textit{ReceptorNet} basada en el proceso ejecutado por el modelo MIL. ReceptorNet aprende a asignar altos pesos a los mosaicos en la imagen de H\&E que tienen la máxima capacidad discriminativa, y a asignar bajos pesos a los mosaicos que son insignificantes para el proceso de aprendizaje y diagnóstico. El análisis de los pesos asignados permitió determinar los mosaicos que se utilizaron para realizar la estimación de ERS.  La arquitectura de ReceptorNet consta de tres redes neuronales interconectadas: un módulo extractor de funciones, un módulo de aprendizaje y un módulo de decisión. Adicionalmente, se comparó el método propuesto con dos métodos MIL ampliamente utilizados: Meanpool y Maxpool. En Meanpool, las representaciones de características de mosaicos se promedian para obtener una representación de características agregadas. En Maxpool, se obtiene un máximo de características para cada una de las dimensiones de características. Estos métodos se entrenaron en la misma arquitectura de modelo que ReceptorNet. El algoritmo propuesto logro una curva AUC de sensibilidad y especificidad de 0,92. En conclusión, esta investigación demostró una estimación precisa del estado del receptor de ER a partir de tinciones de H\&E utilizando una Red Neuronal lo cual impacta en la disminución del trabajo al realizar procesos patológicos. En términos más generales, el estudio represento una mejora de los métodos médicos para el diagnóstico del cáncer de mama y demostró el potencial del ML para mejorar el pronóstico y diagnóstico del cáncer mediante el aprovechamiento de señales biológicas imperceptibles para el ojo humano.

\cite{Roszkowiak2021} propusieron el modelo CHISEL\footnote{Computer assisted Histopathological Image Segmentation and Evaluation} para Segmentación y evaluación de imágenes histopatológicas asistidas por computadora. Este sistema e2e es capaz de evaluar cuantitativamente muestras digitales (benignas y malignas) con tinción nuclear inmunohistoquímica IHC de diversa intensidad y compacidad del tejido afectado por el cáncer de mama. El modelo fue capaz de procesar imágenes patológicas digitales, núcleos de microarrays de tejido TMA\footnote{Tissue Micro Array} y muestras adquiridas por cámara digital conectada a un microscopio. Una de las características principales de CHISEL es la segmentación de imágenes con base a regiones de interés ROIs\footnote{Regions of Interest}. El data-set utilizado fue recopilado con base a imágenes de tejido de cáncer de mama adquiridas de diversas universidades, laboratorios, hospitales y departamentos de patología ubicados en España. Las secciones de tejido de las biopsias de mama y ganglios auxiliares se sometieron a un procesamiento histotécnico tradicional y se convirtieron en bloques de tejido embebidos en parafina y fijados con formalina. Los TMA se formaron extrayendo pequeños cilindros de tejido colocados en una matriz en un bloque de parafina. Posteriormente, se tiñeron con los anticuerpos primarios IHC indirectos contra FOXP3\footnote{proteína P3 de la Forkhead box} y los anticuerpos secundarios, incluido el bloque de peroxidasa, polímero marcado, tamponado con sustrato cromógeno DAB+ y finalmente contrastado con hematoxilina. La tinción simultánea de múltiples muestras en TMA disminuye la variabilidad intralaboratorio de la diferencia en las concentraciones de tinción entre cortes. El TMA tiene una gran ventaja para procesar imágenes digitales porque se adquieren múltiples muestras en un escaneo bajo las mismas condiciones. En general, el uso de TMA aumenta la reproducibilidad de los estudios de patrones de biomarcadores. El modelo propuesto se centró en procesar las imágenes teñidas con IHC (especialmente DAB \& H) para posteriormente estimar la masa de células beta y el número de núcleos dentro del área teñida de insulina. El data-set estaba conformado por 20 imágenes de muestras de tejido teñidas con DAB y H. El modelo utilizo el paradigma de ML y el procesamiento local recursivo para eliminar los contornos distorsionados (inexactos) del conjunto de datos adquiridos. El método se validó utilizando dos data-set etiquetados que demostraron la relevancia de los resultados obtenidos. La evaluación se basó en el conjunto de datos IISPV de tejido de biopsia de pacientes con cáncer de mama, con marcadores de células T, junto con el conjunto de datos Warwick BetaCell de tejido teñido con DAB y H de pacientes con diabetes post-mórtem. Posteriormente, se utilizaron Redes neuronales Artificiales ANN\footnote{Artificial Neural Network} para clasificar las imágenes en categorías malignas o benignas, y se optimizó el algoritmo analizando la influencia del número de capas, el número de neuronas y las funciones de activación en los resultados de la clasificación. Finalmente, la clasificación dio como resultado una precisión media del 99,2\% que se probó con cinco validaciones cruzadas. Este modelo resuelve el complejo problema de la cuantificación de núcleos en imágenes digitalizadas de cortes de tejido teñidos inmunohistoquímicamente, logrando los mejores resultados para muestras de tejido de cáncer de mama teñidas con DAB y H. Para facilidad de manejo del modelo, se generó una interfaz gráfica (GUI) y se optimizó para utilizar completamente la potencia informática disponible. Con base a los resultados de los objetos detectados y clasificados con base a algoritmos de DL y ML, se llegó a la conclusión de que el método propuesto puede lograr resultados mejores o similares a los métodos de última generación. 

\cite{Negahbani2021} propusieron un conjunto de datos denominado SHIDC-BC-Ki-67 basado en la proteína nuclear Ki-67 y los linfocitos infiltrantes de tumores TIL \footnote{Tumor Infiltrating Lymphocytes},los cuales son factores determinantes para predecir la progresión de un tumor cancerígeno y  la respuesta probable del mismo a la quimioterapia. Teniendo en cuenta que la estimación de ambos factores depende de la observación de patólogos profesionales y que también pueden existir variaciones interindividuales, los métodos automatizados que utilizan algoritmos de ML, específicamente los enfoques basados en el DL, son llamativos para realizar el diagnóstico y pronóstico del cáncer de mama. Sin embargo, los modelos de DL necesitan una cantidad considerable de datos etiquetados y este fue el motivo por lo cual se propuso la creación del data-set SHIDC-BC-Ki-67. Adicionalmente, en la investigación realizada se presenta una arquitectura en pipeline, para la estimación de la proteína Ki-67 y la determinación simultánea de la puntación de TIL intratumoral en células cancerígenas. El data-set recopilado estaba formado por imágenes digitales de muestras microscópicas con tinción nuclear inmunohistoquímica IHC obtenidas a través de biopsias realizadas con aguja gruesa(CNB) de tumores de mama malignos del tipo de carcinoma ductal invasivo. Además, se usó hematoxilina para realizar la tinción nuclear y semicuantificar el grado de inmunotinción. Las muestras de la biopsia fueron recolectadas durante un estudio clínico de 2017 a 2020. El data-set SHIDC-B-Ki-67 contiene 1656 datos de entrenamiento y 701 de prueba. Todos los pacientes que participaron en este estudio eran pacientes con un diagnóstico patológicamente confirmado de cáncer de mama cuyas biopsias fueron tomadas en los laboratorios de patología de los hospitales afiliados a la Universidad de Ciencias Médicas de Shiraz en Shiraz, Irán. Para la adquisición de las imágenes se realizaron los siguientes pasos: Primero, el patólogo identifica el tumor y define una región de interés ROI. Para cubrir todo el ROI, se capturan varias imágenes que pueden superponerse. El patólogo selecciona preferentemente imágenes del área tumoral. Luego, patólogos expertos etiquetaron las imágenes teñidas como células tumorales positivas y negativas para Ki-67 y células tumorales con linfocitos infiltrantes positivos. Finalmente, se realizó la clasificación celular y la detección de Ki-67 y TIL, para lograrlo, se usaron Redes Neuronales Convergentes(CNN) para extraer características relevantes y estimar mapas de densidad a partir de una imagen RGB\footnote{Red, Green, Blue} de entrada. PathoNet primero extrae características de las imágenes de entrada y luego predice los píxeles candidatos para las células inmuno-positivas e inmuno-negativas Ki-67, y también los linfocitos con sus valores de densidad correspondientes. El backend propuesto utiliza capas convolucionales para la detección, clasificación y recuento de células en imágenes histopatológicas. Posteriormente se utilizó el algoritmo de Watershed para asignar imágenes en escala de grises a un espacio de relieve topográfico. La arquitectura en pipeline propuesta estaba conformada de una red PathoNet, posprocesamiento y el algoritmo de Watershed. En conclusión, en esta investigación se propuso un nuevo punto de referencia para la detección celular, la clasificación, la estimación del índice de proliferación y la puntuación de TIL mediante imágenes histopatológicas procesadas con modelos de DL para la extracción de características que permiten generar un data-set para la estimación intratumoral en células de cáncer de mama.


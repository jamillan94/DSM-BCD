\section*{Introducción}
El Cáncer de mama ocupa el primer lugar con mayor número de muertes en Colombia ocupando el primer puesto en la tasa de letalidad sobre los demás tipos de cáncer afectando a mujeres de todas las edades. En el año 2020 los casos detectados de cáncer de mama en Colombia fueron 15.509 de los cuales 4.411 terminaron en muerte\cite{InternationalAgencyforResearchonCancer2020}. Este tipo de cáncer se origina cuando las células mamarias comienzan a crecer sin control convirtiéndose en células malignas que normalmente forman un tumor que a menudo se puede observar en una radiografía o se puede palpar como una masa o bulto\cite{Sauer2019}. El pronóstico anticipado de esta enfermedad se ha convertido en una necesidad de investigación debido a que puede facilitar el tratamiento preventivo para evitar su letalidad en un estado avanzado. Muchos investigadores han puesto sus esfuerzos en los diagnósticos y pronósticos del cáncer de mama, cada técnica tiene una tasa de precisión diferente que varía según las diferentes situaciones, herramientas y conjuntos de datos que se utilizan. En la actualidad la ciencia de datos es utilizada por diferentes investigadores para modelar la progresión y el tratamiento de afecciones cancerosas debido a su capacidad para detectar características significativas en conjuntos de datos complejos. La medicina basada en datos tiene la capacidad no solo de mejorar la velocidad y precisión del diagnóstico de enfermedades genéticas, sino también de desbloquear la posibilidad de tratamientos médicos personalizados\cite{Baker2018}. Una parte fundamental de la ciencia de datos es el uso de algoritmos de ML\footnote{Machine Learning} y DL\footnote{Deep Learning} los cuales se componen de tres estrategias principales que consisten en preprocesamiento, extracción de características y clasificación. La extracción de características es fundamental en el diagnóstico del cáncer de mama ya que ayuda a identificar datos relevantes en una categoría maligna o benigna\cite{Fatima2020}. Adicionalmente, en los últimos años, el aumento de la potencia de las computadoras, junto con los avances matemáticos, ha permitido el uso de las redes neuronales complejas de múltiples capas (profundas) las cuales han mejorado el rendimiento de la interpretación automática de imágenes oncológicas altamente estandarizadas\cite{Mann2020}.

En otra instancia la literatura muestra que la mayoría de los casos de estudio de investigación científica y de desarrollo de aplicaciones se han dado sobre la aplicación de estas diferentes técnicas a imágenes médicas. Asimismo, otra forma de obtener información relevante es a través de técnicas de detección por Biopsia como es el caso de la aspiración por aguja Fina (FNA\footnote{Fine Needle Aspiration}) y aspiración por aguja gruesa (CNB\footnote{Core Needle Biopsy}) y las técnicas basadas en el análisis de receptores de estrógeno en datos metabolómicos. 
\\\\
Esta investigación se va centrar esencialmente en diseñar una metodología aplicada a técnicas en ciencias de datos partiendo del análisis de un data-set conformado por dicha información para poder comprobar la efectividad de estos métodos médicos en la detección y el diagnóstico del cáncer de mama, lo que nos permitirá evaluar la validez de tener imágenes diagnosticas u otros tipos de datos oncológicos. Esto va a repercutir, en la facilidad de la obtención de los datos y el procesamiento de la información para el diagnóstico y pronostico del padecimiento de esta enfermedad.

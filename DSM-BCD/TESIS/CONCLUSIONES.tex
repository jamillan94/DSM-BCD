\chapter{Análisis y Resultados}

%--------------------ANALISIS--------------------------------------------
\section{Análisis}
\begin{itemize}[label=\HandPencilLeft]
	\item Para concluir, es plausible afirmar que gracias a la aplicación de la metodología \textit{DSM-BCD (Data Science Methodology for Breast Cancer Diagnosis)}, fue posible extraer información significativa de muestras de tumores cancerígenos mamarios presentados en 817 pacientes recopilados por medio de las intervenciones quirúrgicas de \textit{aspiración con aguja fina (FNA)} y \textit{biopsia con aguja gruesa (CNB)}, a través del aprendizaje automático no supervisado basado en la técnica de agrupación y el modelo \textit{BIRCH}, lo que permitió responder las preguntas planteadas en el \textit{BCQM}, proporcionando información suficiente para diagnosticar el cáncer de mama y la identificación de  rasgos genómicos característicos del carcinoma ductal invasivo(IDC), lobulillar invasivo(ILC) y de tumores mixtos (MDLC), generando un valor agregado al dominio medico al confirmar que el cáncer ILC presenta características genéticas molecularmente diferentes a los demás tipos de cáncer de mama, que  la proteína HER2 positiva es un rasgo genético necesario para diagnosticar el cáncer IDC pero no suficiente para diagnosticar el cáncer ILC y adicional que es posible clasificar el cáncer MDLC en subgrupos de tipo LBC o IDC según sus propiedades genéticas.
\end{itemize}

\newpage
%--------------------RESULTADOS--------------------------------------------
\section{Resultados}
\begin{itemize}[label=\HandPencilLeft]
	\item Con base a las preguntas sobre el cáncer de mama planteadas para esta investigación y el conjunto de datos genómicos recopilados a través de biopsias realizadas a 817 pacientes que fueron diagnosticados con carcinoma lobulillar invasivo(ILC), carcinoma ductal invasivo(IDC) y carcinoma de tumores mixtos(MDLC), se realizó la evaluación de los algoritmos de agrupamiento (\textit{Clustering}): K-Means, Affinity Propagation, Mean Shift, Spectral, Agglomerative, Density-Based Spatial, OPTICS, BIRCH y K-modes. Luego, se utilizaron las métricas de validación interna basadas en el índice de \textit{Davies-Bouldin(DB)} y el \textit{Coeficiente de silhouette} para determinar la congruencia de los \text{clusters} entrenados. Dado lo anterior, el modelo \textit{BIRCH} genero 4 clusters con un coeficiente de Silhouette del $0.1286$, una inercia de $k = 5 $, y un indice de DB del $1.8703$. En otras palabras, el modelo produjo un número adecuado de clusters con una estructura compacta y centros considerablemente separados los unos de los otros. De modo que la precisión del modelo BIRCH fue superior a la de los demás modelos de Machine Learning implementados. Por consiguiente, es plausible afirmar que el modelo BIRCH es el más adecuado para agrupar datos de origen genómico obtenidos por biopsias realizadas por medio de las técnicas FNA y CNB.
\end{itemize}



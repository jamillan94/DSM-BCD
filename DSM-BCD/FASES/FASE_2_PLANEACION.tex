\section{Fase 2: Planeación de actividades}
En esta fase el \textit{Data Analysis Team} basado en las preguntas realizadas en el BCQM analiza todas las tareas que hay que llevar a cabo, las estiman en tiempo y las distribuyen entre las personas que las van a realizar durante el \textit{Release}. Dado que el BCQM nos permite conocer de antemano el tipo de cáncer de mama y la técnica para el diagnostico de esta enfermedad, el científico de datos con ayuda del medico puede definir el origen de datos, lo cual va a permitir conocer el tipo, cantidad y peso de la información. Dado lo anterior, es recomendable que el equipo tenga al menos un ingeniero datos, ya que el es el encargado de tomar los datos y convertirlos en información significativa para que el científico pueda realizar el respectivo análisis.   
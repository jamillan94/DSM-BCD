\section{Fase 5: Procesamiento y transformación de datos oncológicos}
En esta fase, se abarcan todas las actividades para construir el conjunto de datos o imágenes que se utilizará en la siguiente etapa de modelado y ejecución. Entre las actividades del procesamiento y transformación de datos oncológicos, están la limpieza de datos, combinar datos de múltiples fuentes y transformar los datos en variables de valor. En esta fase, es importante el trabajo en equipo y la comunicación continua entre el ingeniero y el científico de datos para tratar los valores no válidos o faltantes, eliminar duplicados, dar un formato adecuado y combinar archivos, tablas y plataformas. Adicionalmente, el medico experto en oncología deberá proporcionar un visto bueno para proceder con la siguiente fase. Esto dado que al ser experto en el tema de dominio tiene un conocimiento mas profundo de las variables o imágenes que esta observando, y si existiese información innecesaria para el diagnostico del cáncer de mama es posible depurar dicha información para que no afecte el entrenamiento y posterior ejecución del modelo de ML y DL.
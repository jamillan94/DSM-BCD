\section{Fase 3: Adquisición de datos oncológicos}
En esta fase, con base a la tareas realizadas en la planeación de actividades, el medico experto en oncología junto con el ingeniero y el científico de datos identifican y reúnen los recursos de datos disponibles (estructurados, no estructurados y semiestructurados) y relevantes para solucionar las preguntas planteadas en el \textit{BCQM}. Cabe resaltar, que en la metodología \textit{\textit{DSM-BCD}} es factible tener varios conjuntos de datos o imágenes que están relacionados a un tipo de cáncer de mama y una técnica de diagnostico, por lo tanto  el \textit{Data Analysis Team} puede tener a varios científicos respondiendo preguntas diferentes en un mismo \textit{Release}. Como consecuencia, al final se pueden obtener como resultado múltiples respuestas y una posible correlación entre las diversas variables oncológicas.  Asimismo, en esta fase el \textit{Data Analysis Team} debe definir la infraestructura de datos necesaria según la cantidad de información a procesar, lo cual permitirá proyectar la escalabilidad, alcance y distribución de dicha información. 

Para este caso de estudio, se utilizaron variables genéticas características de marcadores tumorales  basados en los tipos de cáncer de mama  \textit{Carcinoma ductal invasivo (IDC)} y \textit{Carcinoma lobulillar invasivo (LBC)}. Estas variables fueron obtenidas del conjunto de datos denominado \textit{“Breast Invasive Carcinoma (TCGA, Cell 2015)”} creado a partir del proyecto de carcinoma invasivo de mama \textit{Comprehensive Molecular Portraits of InvasiveLobular Breast Cancer} \cite{Ciriello2015} basado en el \textit {Atlas del Genoma del Cáncer (TCGA\footnote{Acrónimo de “The Cancer Genome Atlas (TCGA)”, en inglés })} cuya finalidad es catalogar cambios moleculares de importancia biológica responsables de la aparición de cáncer haciendo uso de la secuenciación genómica y la bioinformática \cite{TCGA2023}. Los datos fueron descargados del sitio público \textit{cBioPortal} para la genómica del cáncer  (\url{https://www.cbioportal.org/study/summary?id=brca_tcga_pub2015}). 

Cabe resaltar, que el conjunto de datos contiene un total de 817 muestras de tumores de mama que  se perfilaron con 6 plataformas moleculares: Análisis del número de copias somáticas basado en array, Secuenciación del exoma completo, perfil de metilación del ADN basado en array, secuenciación del ARN mensajero, secuenciación de microARN (miARN) y array de proteínas en fase inversa (RPPA), como se ha descrito previamente en \cite{Bass2014}. Un comité de patología revisó y clasificó todos los tumores en 490 IDC, 127 LBC, 88 casos con características mixtas de IDC y LBC, y 112 con otras histologías. Este conjunto de datos consta de un tamaño de $818$ filas y $110$ columnas. Las variables se describen en la tabla \ref{brca_tcga_pub2015_clinical_data}. 

\begin{table*} [!htb]
	\footnotesize
	\begin{threeparttable}
		\caption{Conjunto de datos del Carcinoma invasivo de mama (TCGA, Cell 2015).}
		\label{brca_tcga_pub2015_clinical_data}
		\begin{tabular}{p{1cm} p{4cm} p{10cm}} \toprule	
			\begin{center}$N$\end{center}   
			&\begin{center}Variable\end{center}             
			&\begin{center}Descripción\end{center}      
			\\ \hline	1	&	Study ID	&	Código de identificación del estudio
			\\ \hline	2	&	Patient ID	&	Código de identificación del paciente
			\\ \hline	3	&	Sample ID	&	Código de identificación de la muestra
			\\ \hline	4	&	Diagnosis Age	&	Edad a la que se diagnosticó por primera vez una afección o enfermedad
			\\ \hline	5	&	American Joint Committee on Cancer Metastasis Stage Code	&	Código para representar la ausencia o presencia definida de diseminación a distancia o metástasis (M) a localizaciones a través de canales vasculares o linfáticos más allá de los ganglios linfáticos regionales, utilizando los criterios establecidos por el Comité Conjunto Americano del Cáncer (AJCC)
			\\ \hline	6	&	Neoplasm Disease Lymph Node Stage American Joint Committee on Cancer Code	&	Los códigos que representan el estadio del cáncer en función de los ganglios presentes (estadio N) según criterios basados en múltiples ediciones del Manual de Estadificación del Cáncer de la AJCC
			\\ \hline	7	&	Neoplasm Disease Stage American Joint Committee on Cancer Code	&	Estadio de la extensión de un cáncer, especialmente si la enfermedad se ha propagado desde el sitio original a otras partes del cuerpo según los criterios de estadificación del AJCC
			\\ \hline	8	&	American Joint Committee on Cancer Publication Version Type	&	Versión o edición del American Joint Committee on Cancer Cancer Staging Handbooks, publicación del grupo formado con el propósito de desarrollar un sistema de estadificación clínica del cáncer que sea aceptable para la profesión médica estadounidense y compatible con otras clasificaciones aceptadas
			\\ \hline	9	&	American Joint Committee on Cancer Tumor Stage Code	&	Código de T patológico (tumor primario) para definir el tamaño o la extensión contigua del tumor primario (T), utilizando los criterios de estadificación del AJCC
			\\ \hline	10	&	Brachytherapy first reference point administered total dose	&	Primer punto de referencia dosis total administrada en la Braquiterapia
			\\ \hline	11	&	Cancer Type	&	Tipo de cáncer
			\\ \hline	12	&	Cancer Type Detailed	&	Detalle del tipo de cáncer
			\\ \hline	13	&	Cent17 Copy Number	&	Intervalo de resultados de la señal del cromosoma 17 del procedimiento de diagnóstico de hibridación in situ con fluorescencia.
			\\ \hline	14	&	Birth from Initial Pathologic Diagnosis Date	&	Intervalo de tiempo desde la fecha de nacimiento de una persona hasta la fecha del diagnóstico patológico inicial, representado como un número calculado de días.
			\\ \hline	15	&	Days to Sample Collection.	&	Días para la recolección de muestras
			\\ \hline	16	&	Death from Initial Pathologic Diagnosis Date	&	Intervalo de tiempo desde la fecha de muerte de una persona hasta la fecha del diagnóstico patológico inicial, representado como un número calculado de días.
			\\ \hline	17	&	Last Alive Less Initial Pathologic Diagnosis Date Calculated Day Value	&	Intervalo de tiempo desde el último día en que se sabe que una persona está viva hasta la fecha del diagnóstico patológico inicial, representado como un número calculado de días.
			\\ \hline	18	&	Days to Last Followup	&	Intervalo de tiempo desde la fecha del último seguimiento hasta la fecha del diagnóstico patológico inicial, representado como un número calculado de días.
			\\ \hline	19	&	Disease Free (Months)	&	Sin enfermedad (meses)
			\\ \hline	20	&	Disease Free Status	&	Estado libre de enfermedad
			\\ \hline	21	&	Disease code	&	Código de la enfermedad
			\\ \hline	22	&	ER positivity scale other	&	Escala de medición de otro receptor de estrógeno de hallazgo positivo
			\\ \hline	23	&	ER positivity scale used	&	Escala de hallazgo ER positivo de inmunohistoquímica de carcinoma de mama
			\\ \hline	24	&	ER Status By IHC	&	Estado del receptor de progesterona del carcinoma de mama
			\\ \hline
		\end{tabular}
	\end{threeparttable}
\end{table*}

\begin{table*} [!htb]
	\footnotesize
	\begin{threeparttable}
		\begin{tabular}{p{1cm} p{4cm} p{10cm}}         
			\\ \hline	25	&	ER Status IHC Percent Positive	&	Nivel de receptor de progesterona categoría de porcentaje celular
			\\ \hline	26	&	Ethnicity Category	&	Información sobre el origen étnico.
			\\ \hline	27	&	First surgical procedure other	&	Propósito del procedimiento quirúrgico
			\\ \hline	28	&	Form completion date	&	Fecha de finalización del formulario
			\\ \hline	29	&	Fraction Genome Altered	&	Fracción de genoma alterado
			\\ \hline	30	&	HER2 and cent17 cells count	&	HER2 neu y centrómero 17 número de copia análisis entrada total número recuento
			\\ \hline	31	&	HER2 and cent17 scale other	&	HER2 y centrómero 17 resultados positivos otra escala de medición
			\\ \hline	32	&	HER2 cent17 ratio	&	Valor de la relación de señal del cromosoma 17 de HER2 neu
			\\ \hline	33	&	HER2 copy number	&	Número total de entrada de análisis de copia de carcinoma de mama HER2 neu
			\\ \hline	34	&	HER2 fish method	&	Método de cálculo de hibridación in situ de fluorescencia de HER2 erbb pos
			\\ \hline	35	&	HER2 fish status	&	Procedimiento de laboratorio Tipo de resultado híbrido in situ HER2 neu
			\\ \hline	36	&	HER2 ihc percent positive	&	HER2 ihc positivo en porcentaje
			\\ \hline	37	&	HER2 ihc score	&	Resultado del nivel de inmunohistoquímica HER2
			\\ \hline	38	&	HER2 positivity method text	&	Método de positividad HER2
			\\ \hline	39	&	HER2 positivity scale other	&	Otra medida escala pos hallazgo HER2 erbb2 
			\\ \hline	40	&	Neoplasm Histologic Type Name	&	Término que designa el patrón estructural de las células cancerosas utilizado para definir un diagnóstico microscópico.
			\\ \hline	41	&	Tumor Other Histologic Subtype	&	Subtipo histológico de un tumor o el diagnóstico mixto que es diferente de las opciones especificadas anteriormente.
			\\ \hline	42	&	Neoadjuvant Therapy Type Administered Prior To Resection Text	&	Término para describir el historial de tratamiento neoadyuvante del paciente y el tipo de tratamiento administrado antes de la resección del tumor.
			\\ \hline	43	&	Prior Cancer Diagnosis Occurence	&	Término para describir los antecedentes de diagnóstico previo de cáncer del paciente y la ubicación espacial de cualquier aparición previa de cáncer.
			\\ \hline	44	&	ICD-10 Classification	&	Decima revisión de la Clasificación Estadística Internacional de Enfermedades y Problemas Relacionados con la Salud.
			\\ \hline	45	&	International Classification of Diseases for Oncology, Third Edition ICD-O-3 Histology Code	&	Tercera edición de la Clasificación Internacional de Enfermedades Oncológicas, publicada en 2000, utilizada principalmente en los registros de tumores y cáncer para codificar la localización (topografía) y la histología (morfología) de las neoplasias. Estudio de la estructura de las células y su disposición para constituir tejidos y, finalmente, la asociación entre éstos para formar órganos. En patología, proceso microscópico de identificación de las características morfológicas normales y anormales de los tejidos mediante el empleo de diversas tinciones citoquímicas e inmunocitoquímicas.
			\\ \hline	46	&	International Classification of Diseases for Oncology, Third Edition ICD-O-3 Site Code	&	Tercera edición de la Clasificación Internacional de Enfermedades Oncológicas, publicada en 2000, utilizada principalmente en los registros de tumores y cáncer para codificar la localización (topografía) y la histología (morfología) de las neoplasias. Sistema de categorías numeradas para la representación de datos.
			\\ \hline	47	&	IHC-HER2	&	Término que designa el estad de la prueba IHC-HER2
			\\ \hline	48	&	IHC Score	&	Puntuación IHC
			\\ \hline	49	&	Informed consent verified	&	Consentimiento informado verificado
			\\ \hline	50	&	Year Cancer Initial Diagnosis	&	Año del diagnóstico patológico inicial de cáncer de un individuo
			\\ \hline	51	&	Is FFPE	&	Si la muestra es de tejido fijado con formalina e incrustado en parafina (FFPE)
			\\ \hline
		\end{tabular}
	\end{threeparttable}
\end{table*}

\begin{table*} [!htb]
	\footnotesize
	\begin{threeparttable}
		\begin{tabular}{p{1cm} p{4cm} p{10cm}}
			\\ \hline	52	&	Primary Lymph Node Presentation Assessment Ind-3	&	Término que indica si se realizó una evaluación de los ganglios linfáticos en la presentación primaria de la enfermedad.
			\\ \hline	53	&	Positive Finding Lymph Node Hematoxylin and Eosin Staining Microscopy Count	&	Recuento de ganglios linfáticos positivos identificados mediante microscopía óptica con tinción de hematoxilina y eosina (H\&E).
			\\ \hline	54	&	Positive Finding Lymph Node Keratin Immunohistochemistry Staining Method Count	&	Recuento de ganglios linfáticos positivos identificados a través del método de tinción de inmunohistoquímica (IHC) de queratina
			\\ \hline	55	&	Lymph Node(s) Examined Number	&	Número total de ganglios linfáticos extirpados y evaluados patológicamente para la enfermedad
			\\ \hline	56	&	Margin status reexcision	&	Estado de los márgenes de la cirugía del cáncer de mama
			\\ \hline	57	&	Menopause Status	&	Estado de la menopausia de una mujer, el cese permanente de la menstruación, generalmente definido por 6 a 12 meses de amenorrea.
			\\ \hline	58	&	Metastatic Site	&	Localización anatómica a la que se ha extendido el cáncer
			\\ \hline	59	&	Metastatic Site Other	&	Otra localización anatómica a la que se ha extendido el cáncer
			\\ \hline	60	&	Metastatic tumor indicator	&	Presencia de metástasis
			\\ \hline	61	&	First Pathologic Diagnosis Biospecimen Acquisition Method Type	&	Nombre del procedimiento para asegurar el tejido utilizado para el diagnóstico patológico original
			\\ \hline	62	&	First Pathologic Diagnosis Biospecimen Acquisition Other Method Type	&	Método utilizado para obtener tejido para un diagnóstico patológico original que es diferente de otros métodos identificados
			\\ \hline	63	&	Micromet detection by ihc	&	Detección Micromet por ihc
			\\ \hline	64	&	Mutation Count	&	Recuento de mutaciones
			\\ \hline	65	&	New Neoplasm Event Post Initial Therapy Indicator	&	Indicador para identificar si un paciente ha tenido un nuevo evento tumoral después del tratamiento inicial
			\\ \hline	66	&	Nte cent 17 HER2 ratio	&	Valor de la relación señal HER2 neu cromosoma 17 en el carcinoma metastásico de mama 
			\\ \hline	67	&	Nte er ihc intensity score	&	Carcinoma metastásico de mama inmunohistoquímica con puntuación de celulas con ER positivo
			\\ \hline	68	&	Nte er status	&	Estado del receptor de estrógenos del carcinoma metastásico de mama
			\\ \hline	69	&	Nte er status ihc positive	&	Categoría porcentual de células de carcinoma de mama metastásico con nivel de receptores de estrógenos
			\\ \hline	70	&	Nte HER2 fish status	&	Tipo de resultado de hibridación in situ de HER2 neu de procedimiento de laboratorio de carcinoma de mama metastásico
			\\ \hline	71	&	Nte HER2 positivity ihc score	&	Resultado del nivel de inmunohistoquímica erbb2 de carcinoma de mama metastásico
			\\ \hline	72	&	Nte HER2 status	&	Término que indica si se realizó el proceso de laboratorio de carcinoma de mama metastásico estado del receptor de inmunohistoquímica HER2 neu
			\\ \hline	73	&	Nte HER2 status ihc positive	&	Porcentaje de células Carcinoma de mama metastásico HER2 erbb positivos 
			\\ \hline	74	&	Nte pr ihc intensity score	&	Puntuación de intensidad Nte pr ihc
			\\ \hline	75	&	Nte pr status by ihc	&	Estado del receptor de progesterona del carcinoma de mama metastásico
			\\ \hline	76	&	Nte pr status ihc positive	&	Porcentaje de células de nivel de receptor de progesterona de carcinoma de mama metastásico
			\\ \hline	77	&	Oct embedded	&	Término que indica si se realizón una incrustación OCT
			\\ \hline	78	&	Oncotree Code	&	Código del tipo de cancer en formato Oncotree para cBioPortal
			\\ \hline	79	&	Overall Survival (Months)	&	Supervivencia general (meses)
			\\ \hline	80	&	Overall Survival Status	&	Estado de supervivencia global del paciente.
			\\ \hline	81	&	Other Patient ID	&	Otro código de identificación del paciente
			\\ \hline
		\end{tabular}
	\end{threeparttable}
\end{table*}

\begin{table*} [!htb]
	\footnotesize
	\begin{threeparttable}
		\begin{tabular}{p{1cm} p{4cm} p{10cm}}
			\\ \hline	82	&	Other Sample ID	&	Otro código de identificación de la muestra
			\\ \hline	83	&	Pathology Report File Name	&	Nombre del archivo del informe de patología
			\\ \hline	84	&	Disease Surgical Margin Status	&	Resultados concluyentes tras el examen de los márgenes del tejido para detectar la presencia de enfermedad
			\\ \hline	85	&	Adjuvant Postoperative Pharmaceutical Therapy Administered Indicator	&	Término para indicar si el paciente tuvo o no terapia farmacéutica.
			\\ \hline	86	&	Primary Tumor Site	&	Sitio del tumor para identificar la subdivisión de órganos en un individuo con cáncer
			\\ \hline	87	&	Project code	&	Código de proyecto
			\\ \hline	88	&	Tissue Prospective Collection Indicator	&	Indicador de recolección prospectiva de tejido
			\\ \hline	89	&	PR positivity define method	&	Método utilizado para definir PR positiva
			\\ \hline	90	&	PR positivity ihc intensity score	&	Puntuación de intensidad ihc positiva de PR
			\\ \hline	91	&	PR positivity scale other	&	Otra medida de Porcentaje positivo del receptor de progesterona
			\\ \hline	92	&	PR positivity scale used	&	Escala de inmunohistoquímica para el hallazgo de receptores de progesterona positiva en el  carcinoma de mama
			\\ \hline	93	&	PR status by ihc	&	Estado del receptor de progesterona del carcinoma de mama
			\\ \hline	94	&	PR status ihc percent positive	&	Categoría de porcentaje celular del nivel del receptor de progesterona
			\\ \hline	95	&	Race Category	&	Información sobre la raza
			\\ \hline	96	&	Did patient start adjuvant postoperative radiotherapy?	&	Término para indicar si el paciente inició radioterapia postoperatoria adyuvante.
			\\ \hline	97	&	Tissue Retrospective Collection Indicator	&	Término para indicar el marco de tiempo de obtención de tejido, identificando si el tejido fue obtenido y almacenado antes del inicio del proyecto.
			\\ \hline	98	&	Number of Samples Per Patient	&	Número de muestras por paciente
			\\ \hline	99	&	Sample Type	&	Tipo de muestra
			\\ \hline	100	&	Sex	&	Sexo del paciente
			\\ \hline	101	&	Somatic Status	&	Estado somático
			\\ \hline	102	&	Staging System	&	Sistema de estadificación
			\\ \hline	103	&	Staging System\_1	&	Otro sistema de estadificación
			\\ \hline	104	&	Surgery for positive margins	&	Nombre del procedimiento quirúrgico primario de carcinoma de mama
			\\ \hline	105	&	Surgery for positive margins other	&	Nombre del procedimiento quirúrgico para márgenes positivos
			\\ \hline	106	&	Surgical procedure first	&	Nombre del procedimiento quirúrgico de carcinoma de mama
			\\ \hline	107	&	Tissue Source Site	&	Sitio fuente recopilado de la muestra (tejido, células o sangre) y metadatos clínicos que luego se envían al recurso principal de bioespecímenes.
			\\ \hline	108	&	TMB (nonsynonymous)	&	Número total de mutaciones (cambios) que se encuentran en el ADN de las células cancerosas
			\\ \hline	109	&	Person Neoplasm Status	&	El estado o condición de la neoplasia de un individuo en un momento determinado
			\\ \hline	110	&	Tumor Disease Anatomic Site	&	Término que describe el sitio anatómico del tumor o enfermedad
			\\ \hline
		\end{tabular}
	\end{threeparttable}
\end{table*}


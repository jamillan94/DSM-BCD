\section{Fase 8: Retroalimentación medica }
En esta fase, el medico experto en oncología determina si los resultados generados por el modelo de ML o DL lograron responder las preguntas planteadas en el \textit{BCQM} y si la nueva informacion obtenida es suficiente para diagnosticar el cáncer de mama o si dichos resultados generaron información relevante para determinar la causa u origen de esta enfermedad, en pocas palabras, si los datos analizados produjeron un valor agregado al dominio medico. En el caso de que los resultados obtenidos no lograsen dar valor a los datos, el \textit{Data Analysis Team} deberá decidir si es necesario re-plantear las preguntas o si se debe adquirir nuevos datos para ajustar el modelo generado. Ademas, el experto en compañía del científico y el ingeniero de datos, basado en su perspicacia medica, deberá ayudar a decidir cual estrategia es las mas apropiada para generar resultados significativos. De forma similar, si el resultado fue satisfactorio el medico debe emitir un dictamen del \textit{nivel de impacto} que tuvo la información generada por los modelos al diagnosticar el padecimiento del cáncer de mama a un determinado paciente y unas vez comprobada la informacion, junto al \textit{Data Analysis Team} alimentar un conjunto de datos con la informacion obtenida de los diagnósticos generados a cada individuo. Lo anterior con el proposito de mejorar el desempeño de los modelos existentes y aumentar su precisión. Finalmente, en cada \textit{Release} se debe garantizar que el tiempo de diagnostico sea cada vez menor o que se genere nueva información que el medico pueda utilizar en sus funciones diarias y que ayude a determinar el origen, relación o posible tratamiento de esta enfermedad.
\section{Fase 9: Bitácora para el diagnostico del cáncer de mama (BCDL) }
En esta fase, se propone el uso de una bitácora para el diagnostico del cáncer de mama (BCDL). El proposito del \textit{BCDL} es almacenar la respuestas obtenidas por cada pregunta planteada en el \textit{BCQM} y la relación de estas preguntas y respuestas con determinado modelo de ML o DL. Esta bitácora solamente debe ser alimentada cuando la informacion obtenida generó valor agregado al dominio medico. Su principal proposito es evitar la redundancia de la informacion y la duplicidad de preguntas planteadas en el \textit{BCQM}, garantizando que en cada \textit{Release} se genere nuevo conocimiento relacionado al cáncer de mama. Se recomienda que la bitácora sea diseñada por medio de un \textit{modelo entidad relación (MER)} que este conformado por entidades como: modelo, tipo de cáncer de mama, técnica de diagnostico, conjunto de datos, pregunta y respuesta. Dado lo anterior, se sugiere que  los diferentes conjuntos de datos o imágenes utilizados en los análisis realizados, sean almacenados en un servicio de alojamiento de informacion en la nube (Amazon Cloud, Google Drive, One Drive, etc.) y que dicho informacion este identificada con un código único que facilite su búsqueda cuando sea requerido. De igual manera, los diferentes algoritmos generados deben ser almacenados en un sistema de control de versiones (GitLab, GitHub, Bitbucket, etc.) con su respectivo \textit{Readme} de funcionamiento y un código de identificación único para que pueda ser consultado fácilmente por base de datos. Por consiguiente, el uso del \textit{BCDL} permite tener una trazabilidad detallada de los avances obtenidos en cada \textit{Release} con respecto al diagnostico del cáncer de mama, esto con el proposito de que el \textit{Data Analysis Team} tenga un punto de partida solido para innovar en nuevos modelos de ML y DL a través de la comparación  y la mejora continua de modelos existentes que lograron agregar valor a los diferentes tipos de datos oncológicos.
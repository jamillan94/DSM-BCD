\section{INTRODUCTION}

In 2020, the number of cases of breast cancer worldwide amounted to 2,261,419 and 684,996 of these cases ended in death. In Colombia, there were 15,509 cases and 4,411 deaths \cite{InternationalAgencyforResearchonCancer2020}, ranking first in the lethality rate among all types of cancer affecting women of all ages. This type of cancer originates when breast cells begin to grow out of control becoming malignant cells that usually form a tumor that can often be seen on an x-ray or can be palpated as a mass or lump \cite{Sauer2019}. Colombia has limitations with respect to access to early detection and diagnosis of cancer, caused in most cases by factors such as socio-economic status, health insurance coverage, origin and accessibility. On average, the waiting time for a patient is 90 days from the onset of symptoms to the diagnosis of cancer. The first action to reduce the mortality rate from breast cancer should be focused on the speed of diagnosis and timely access to care\cite{Duarte2021}. Early prognosis of this disease has become a research need because it may facilitate preventative treatment to avoid its progression lethality in an advanced stage. An alternative to reduce this mortality rate is to be able to predict and identify, based on the analysis of a set of data obtained from examinations performed by various medical methods to the individual, what is the probability of contracting breast cancer and what are the variables that most influence the suffering of this disease, and according to these results provide a preventive treatment to combat cancer before it metastasizes or reaches an advanced stage where it is more difficult to treat.

The analysis obtained through data science allows detecting cancer in a shorter time, because ML (Machine Learning) and DL (Deep Learning) algorithms clearly impact exploratory studies that aim to identify the biological principles of the disease which can benefit patients and physicians by speeding up diagnosis and providing support to make better and faster decisions at the clinical level\cite{Turin2020}. In recent years, the increase in computer power, coupled with mathematical advances, has enabled the use of multilayer (deep) complex neural networks which have improved the performance of automatic interpretation of highly standardized oncological images\cite{Mann2020}.

On the other hand, science, in the language of the scientific method, is to formulate hypotheses or conjectures based on observations of the world around us in order to validate or invalidate those hypotheses by conducting experiments. However, unlike the pure sciences, working with data does not necessarily require conducting experiments. Rather, often the data have already been collected and organized previously. So, the scientific method, applied to data, can be summarized as: \textit{“Formulate hypotheses based on the world around us and then analyze the relevant data to validate or invalidate those hypotheses”}.

Data science is currently used by different researchers to model the progression and treatment of cancerous conditions because of its ability to detect significant features in complex data sets. Data-driven medicine has the ability not only to improve the speed and accuracy of diagnosis of genetic diseases, but also to unlock the possibility of personalized medical treatments\cite{Baker2018}. To learn means: \textit{“To acquire knowledge or skills in something through experience”}. Therefore, ML could be framed as the way in which a machine gains or acquires knowledge through experience. But, how does a machine gain experience? All inputs to a machine are essentially binary strings of 0 and 1, which in the domain of computer science are simply data. Therefore, ML is really the way a computer acquires knowledge through data. \\

Data science is fundamentally a process, while ML is a tool that can be immensely useful in carrying out that process\cite{Pillai2020}. Of course, this does not give any idea of the how at all; it simply summarizes the process as something that is done with the input data to generate this knowledge as output. To make a mathematical analogy, ML is a function $f$ such that:

\begin{equation}
	knowledge = f(Data)
\end{equation}

It should also be mentioned that a methodology is a general strategy that serves as a guide for the processes and activities that are within a given domain. Methodology does not rely on specific technologies or tools, nor is it a set of techniques or recipes. Rather, methodology provides the data scientist with a framework for how to proceed with the methods, processes and arguments that will be used to obtain answers or results\cite{Rollins2015}.
Thus, the objective of this paper is to propose a methodology in data science for the diagnosis of breast cancer, this with the purpose of applying the use of this computational science as a multidisciplinary field that allows to give value to the data obtained through medical techniques for the diagnosis of breast cancer through a process that facilitates the analysis and selection of techniques in order to make a decision that validates the veracity of a hypothesis raised. The use of the proposed methodology has an impact on the ease of obtaining data and processing information for the diagnosis and prognosis of this disease.

\section{BREAST CANCER SCREENING AND DIAGNOSIS}
Breast cancer, with its uncertain cause, has captured the attention of surgeons throughout the ages. Despite centuries of theoretical mazes and scientific questions, breast cancer is still one of the most dreaded human diseases\cite{Bland2009}. According to \cite{Fatima2020}, breast cancer originates through malignant tumors, when the cell growth gets out of control causing many fatty tissues and fibrous tissues of the breast initiate abnormal growth, which results in cancer cells spreading through the tumors causing the different stages of cancer. \cite{Fatima2020} state that there are different types of breast cancer, which occur when the affected cells and tissues spread throughout the body \cite{Sun2017}. For example, the first type of cancer called \textit{Ductal Carcinoma in situ (DCIS)} is a non- invasive type of cancer\cite{Hou2020}, which occurs when abnormal cells spread outside the breast. The second type of cancer is \textit{Invasive Ductal Carcinoma (IDC)}, also known as infiltrating ductal carcinoma \cite{Chaudhury2011}. This type of cancer occurs when abnormal breast cells spread throughout the breast tissues. Usually, this type of cancer is found in men \cite{Page1982}. The third type of cancer is \textit{Mixed Tumor Cancer (MTBC)} also known as invasive breast cancer\cite{Tuck1997}  caused by abnormal duct cells and lobular cells\cite{Lee2017}. The fourth type of cancer is \textit{Lobular Breast Cancer (LBC)}\cite{Masciari2007}  which occurs within the lobule and increases the chances of other invasive cancers. The fifth type of cancer is \textit{Mucinous Breast Cancer (MBC)} or \textit{Colloid Breast Cancer}\cite{Memis2000} that occurs due to invasive ductal cells when abnormal tissue spreads around the duct \cite{Gradilone2011}. The sixth and last type of cancer is \textit{Inflammatory Breast Cancer (IBC)}, which causes swelling and redness of the breast. This type of breast cancer is fast growing, and begins to appear when lymphatic vessels become clogged in ruptured cells \cite{Robertson2010}. According to \cite{Brunicardi2010}, in the majority of detected cases the woman discovers a lump in her breast. Other signs and symptoms that occur less often include: breast growth or asymmetry, nipple alterations and retraction or telorrhea, ulceration or erythema of the breast skin, an axillary mass, and musculoskeletal discomfort. It should be noted that if any of the above symptoms are detected, this type of cancer can be detected by means of procedures based on \textit{physical examination}, \textit{imaging techniques} and \textit{biopsies}. At the physical examination level, breast cancer can be detected by the oncologist using the methods of \textit{inspection} and \textit{palpation}. With these methods, the symmetry, size and shape of the breast are recorded, as well as any evidence of edema (orange peel), nipple or skin retraction and erythema \cite{Brunicardi2010}. Currently, many \textit{imaging techniques} are widely used to provide an accurate diagnosis of breast lesions \cite{Tamam2021}, among these techniques the most relevant are the following: \textit{Mammography} which makes use of a mammographic unit consisting of an X-ray tube encapsulating a cathode and an anode. The breast is placed on the detector and is compressed by a device of parallel plates, which keeps the breast immobile and prevents blurring by movement, this in order to reduce the thickness of the tissue that must pass through the X-rays \cite{Ebrahimi2019}; \textit{Ductography} that identifies lesions in patients with nipple discharge. This method is effective in localizing and identifying intraductal lesions by means of a mammographic examination performed after retrograde filling of the lactiferous ducts with contrast material \cite{Hirose2007}; \textit{Ultrasound}, which allows high-resolution images to be obtained by means of a small high- frequency transducer (tube) that sends inaudible sound waves into the breast and receives the echo of waves from internal organs, fluids and tissues \cite{Hasan2019}; and \textit{Magnetic Resonance Imaging (MRI)}, which is used when breast lesions cannot be easily assessed by other techniques. To achieve this, it uses radiofrequency (RF) receiver coils to detect a signal emitted by tissues upon excitation of an electromagnetic field that forces protons to align to the anatomy of the area of interest in size and shape \cite{Tse2014}. It should also be mentioned that there are currently there are two ways to obtain a diagnosis by \textit{biopsy} for a patient with a breast abnormality. On the one hand we have the biopsy for breasts with p\textit{alpable lesions} also called \textit{percutaneous} or \textit{minimally invasive}. These biopsies include Fine Needle Aspiration (FNA) and Core Needle Aspiration (CNB). Open surgical biopsies are sometimes referred to as excisional biopsies or incisional biopsies. \textit{Excisional} biopsy indicates complete removal of the lesion, whereas \textit{incisional} biopsy indicates removal of part of the lesion \cite{Greenfield2012}. For nonpalpable lesions, imaging modalities such as ultrasound (US), mammography, and magnetic resonance imaging (MRI) are useful adjuncts to identify and localize the lesion of interest. The decision of when to perform a breast biopsy depends on the patient's history, physical examination findings, and radiologic imaging. The primary goal of biopsy is to obtain a tissue diagnosis that can help dictate treatment and preoperative planning, if indicated. Therefore, it is imperative to choose a biopsy technique that optimizes the chances of obtaining an accurate diagnosis while minimizing costs, limiting patient discomfort and reducing the need for repeat procedures\cite{Samilia2018}.

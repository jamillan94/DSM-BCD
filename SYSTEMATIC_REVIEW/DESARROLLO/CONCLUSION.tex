\section{CONCLUSIÓN}
Con respecto a las metodologías revisadas cabe señalar que no tienen un dominio especifico, por lo tanto queda abierto para los interesados en el análisis de datos la selección de la metodología para un problema determinado. Dicho lo anterior, en la revisión se encontraron metodologías, que aunque abarcan el proceso necesario para el análisis de datos, no generan el valor suficiente para la solución de un problema determinado. Por esta razón, autores como \citep{Martinez2021} consideran que aunque la comunidad de investigación en ciencia de datos está creciendo día a día, esté explorando nuevos dominios, creando nuevos roles especializados y adicional este realizando un gran esfuerzo de investigación para desarrollar análisis avanzados, mejorar modelos y generar nuevos algoritmos apoyados de los campos de las matemáticas, la estadística y la informática, estas habilidades no son suficientes para su aplicación en proyectos reales, puesto que, presentan problemas organizativos y socio-técnicos ,tales como: la falta de visión, objetivos poco claros, un énfasis sesgado en cuestiones técnicas, un bajo nivel de madurez para proyectos ad-hoc y la ambigüedad de los roles. Aunque estos problemas existen en proyectos de ciencia de datos del mundo real, la comunidad no se ha preocupado demasiado por ellos y no se ha escrito lo suficiente sobre las soluciones para abordar estos problemas.

Para comprender mejor, autores como\citep{Schroer2021} consideran que aunque CRISP-DM sigue siendo un estándar por defecto en la minería de datos tiene inconvenientes, ya que la mayoría de los estudios no prevén una fase de implementación. De igual modo, \citep{Martinez2021} consideran que aunque CRISP-DM sigue siendo la metodología más popular para proyectos de análisis, minería de datos y ciencia de datos, no explica cómo deben organizarse equipos de trabajo para llevar a cabo procesos de gestión que se alineen con el software, las metodologías de desarrollo ágiles, y las actividades individuales dentro de cada unas de las etapas que la componen. 

Algo semejante sucede con las metodologías KDD y SEMMA, antecesoras de CRISP-DM. Por el lado de KDD, tenemos que aunque se parte del entendimiento del dominio, su ciclo de vida termina en la visualización e interpretación del conocimiento obtenido sin incluir la retroalimentación por parte del experto; Por lo tanto, aunque se evalué la precisión de los modelos utilizados, esto no es suficiente para determinar si el conocimiento obtenido fue suficiente para tomar una decisión que ayude a solucionar un problema en especifico.   

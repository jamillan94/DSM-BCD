\begin{abstract}
	\begin{itemize}[label=]
		\item \textbf{\color{darkblue} Introducción.} En el año 2020 los casos detectados de cáncer de mama en Colombia fueron 15.509 de los cuales 4.411 terminaron en muerte. El pronóstico anticipado de esta enfermedad se ha convertido en una necesidad de investigación debido a que puede facilitar el tratamiento preventivo para evitar su letalidad en un estado avanzado. Esta investigación se centra esencialmente en la exploración de metodologías en ciencia de datos existentes para la detección y el diagnóstico del cáncer de mama.
		\\
		\item \textbf{\color{darkblue} Objetivo.} Analizar las metodologías en ciencia de datos propuestas para el diagnostico del cáncer de mama. 
		\\
		\item \textbf{\color{darkblue} Métodos.} Se efectuó una revisión sistemática de 20 metodologías en ciencia de datos propuestas por la comunidad científica en IEEE Access y Science Direct siguiendo las directrices PRISMA .
		\\
		\item \textbf{\color{darkblue} Resultados y conclusiones.} Los resultados obtenidos permiten inferir que en las investigaciones revisadas no existe definida una metodología en ciencia de datos para el diagnóstico del cáncer de mama. Sin embargo, diversos autores proponen metodologías que permiten la aplicación de esta ciencia en proyectos reales y que pueden utilizarse en su totalidad en el dominio de las ciencias de la salud, teniendo como eje principal la comprensión de aspectos como: la comunicación constante con los interesados, el uso del enfoque ágil, la definición de roles y funciones, el valor agregado de los datos, la retroalimentación continua y la aceptación del experto en oncología para la posterior toma de decisiones que ayude a reducir de manera eficaz la morbilidad causada por esta enfermedad.
		\\
		\item \textbf{\color{darkblue} Palabras Clave.} Metodología, Ciencia de datos, Cáncer de mama, Diagnostico.
	\end{itemize}
\end{abstract}


\section{Introducción}
\label{sec:int}

El Cáncer de mama ocupa el primer lugar con mayor número de muertes en Colombia ocupando el primer puesto en la tasa de letalidad sobre los demás tipos de cáncer afectando a mujeres de todas las edades. En el año 2020 los casos detectados de cáncer de mama en Colombia fueron 15.509 de los cuales 4.411 terminaron en muerte\citep{InternationalAgencyforResearchonCancer2020}. Este tipo de cáncer se origina cuando las células mamarias comienzan a crecer sin control convirtiéndose en células malignas que normalmente forman un tumor que a menudo se puede observar en una radiografía o se puede palpar como una masa o bulto \citep{Sauer2019}.

Colombia presenta limitaciones con respecto al acceso de la detección y el diagnóstico temprano del cáncer, provocado en la mayoría de los casos por factores como el estrato socio-económico, la cobertura del seguro de salud, el origen y la accesibilidad. En promedio, el tiempo de espera de un paciente es de 90 días desde la aparición de los síntomas hasta el diagnóstico de dicho cáncer. La primera acción para reducir la tasa de mortalidad por cáncer de mama debe estar enfocada en la agilidad del diagnóstico y el acceso oportuno a la atención \citep{Duarte2021}. 

El pronóstico anticipado de esta enfermedad se ha convertido en una necesidad de investigación debido a que puede facilitar el tratamiento preventivo para evitar su letalidad en un estado avanzado. Una alternativa para disminuir esta tasa de mortalidad es poder predecir e identificar, con base al análisis de un conjunto de datos obtenidos de exámenes realizados por diversos métodos médicos al individuo, que probabilidad tiene de contraer el cáncer de mama y cuales son las variables que más influyen en el padecimiento de esta enfermedad,  y según estos resultados brindar un tratamiento preventivo que permita combatir el cáncer antes de que el mismo haga metástasis o que llegue a un estado avanzado en donde sea más difícil de tratar. 

El análisis obtenido por medio de la ciencia de datos permite detectar el cáncer en un menor tiempo, debido a que los algoritmos de clasificación de ML y DL impactan claramente en los estudios exploratorios que tienen como objetivo identificar los principios biológicos de la enfermedad lo que puede beneficiar a pacientes y médicos al acelerar el diagnóstico y brindar apoyo para tomar mejores y más rápidas decisiones a nivel clínico\cite{Turin2020}. 

Debido a que la ciencia de datos es un campo multidisciplinario el cual incorpora herramientas computacionales que permiten dar valor a los datos y aprender de los mismos para poder tomar una decisión que valide la veracidad de una hipótesis planteada, es la mejor opción para generar un diagnóstico significativo en la detección de esta enfermedad. Muchos investigadores han puesto sus esfuerzos en los diagnósticos y pronósticos del cáncer de mama, cada técnica tiene una tasa de precisión diferente que varía según las diferentes situaciones, herramientas y conjuntos de datos que se utilizan. 

La ciencia, en el lenguaje del método científico, es formular hipótesis o conjeturas sobre cómo funciona el mundo, basadas en observaciones del mundo que nos rodea para validar o invalidar esas hipótesis mediante la realización de experimentos. Sin embargo, a diferencia de las ciencias puras, trabajar con datos no requiere necesariamente realizar experimentos. Más bien, muchas veces los datos ya han sido recopilados y organizados previamente. Entonces, el método científico, aplicado a los datos, se puede resumir como: \textit{“Formular hipótesis basadas en el mundo que nos rodea y luego analizar los datos relevantes para validar o invalidar dichas hipótesis”}. 

En la actualidad la ciencia de datos es utilizada por diferentes investigadores para modelar la progresión y el tratamiento de afecciones cancerosas debido a su capacidad para detectar características significativas en conjuntos de datos complejos. La medicina basada en datos tiene la capacidad no solo de mejorar la velocidad y precisión del diagnóstico de enfermedades genéticas, sino también de desbloquear la posibilidad de tratamientos médicos personalizados\citep{Baker2018}. Una parte fundamental de la ciencia de datos es el uso de algoritmos de Machine Learning(ML) y Deep Learning(DL).

Aprender significa: \textit{“Adquirir conocimientos o habilidades en algo a través de la experiencia”}.  Por lo tanto, se podría enmarcar al ML cómo la manera en la cual una máquina gana o adquiere conocimiento a través de la experiencia. Pero ¿Cómo adquiere experiencia una máquina? Todas las entradas de una máquina son esencialmente cadenas binarias de 0 y 1, que en el dominio de las ciencias de la computación dichos binarios son simple y llanamente datos. Por consiguiente, el ML es realmente la forma en que una computadora adquiere conocimiento a través de los datos.  

La ciencia de datos es fundamentalmente un proceso, mientras que el ML es una herramienta que puede ser inmensamente útil para llevar a cabo dicho proceso \citep{Pillai2020}. Por supuesto, esto no da ninguna idea del \textit{como} en absoluto; simplemente se resume el proceso como algo que se hace con los datos de entrada para generar este conocimiento como salida. Para hacer una analogía matemática, el ML es una función $f$ tal que:

\begin{equation}
Conocimiento=f(Datos)
\end{equation}

Adicionalmente, en los últimos años, el aumento de la potencia de las computadoras, junto con los avances matemáticos, ha permitido el uso de las redes neuronales complejas de múltiples capas (profundas) las cuales han mejorado el rendimiento de la interpretación automática de imágenes oncológicas altamente estandarizadas\citep{Mann2020}.

En otra instancia la literatura muestra que la mayoría de los casos de estudio de investigación científica y de desarrollo de aplicaciones se han dado sobre la aplicación de estas diferentes técnicas a imágenes medicas. Asimismo, otra forma de obtener información relevante es a través de técnicas de detección por Biopsia como es el caso de la aspiración por aguja Fina (FNA\footnote{Fine Needle Aspiration}) y aspiración por aguja gruesa (CNB\footnote{Core Needle Biopsy}) y las técnicas basadas en el análisis de receptores de estrógeno en datos metabolómicos. 

A medida que las capacidades de analítica de datos se vuelven más accesibles y prevalentes, los científicos de datos necesitan una metodología fundamental capaz de proporcionar una estrategia de orientación, que sea independiente de las tecnologías, los volúmenes de datos o los enfoques involucrados. Una metodología es una estrategia general que sirve de guía para los procesos y actividades que están dentro de un dominio determinado. La metodología no depende de tecnologías ni herramientas específicas, ni es un conjunto de técnicas o recetas. Más bien, la metodología proporciona al científico de datos un marco sobre cómo proceder con los métodos, procesos y argumentos que se utilizarán para obtener respuestas o resultados\citep{Rollins2015} .

Así, el objetivo de esta revisión es analizar de forma sistemática las investigaciones disponibles acerca de las metodologías en ciencia de datos para el diagnostico del cáncer de mama, esto con el propósito de tener un panorama real acerca del uso de esta ciencia computacional como un campo interdisciplinario que permita dar valor a los datos obtenidos por medio técnicas medicas para el diagnostico del cáncer de mama a través de una metodología que facilite abordar el análisis y la selección de técnicas para poder tomar una decisión que valide la veracidad de una hipótesis planteada. El uso de una metodología repercute en la facilidad de la obtención de los datos y el procesamiento de la información para el diagnostico y pronostico del padecimiento de esta enfermedad.
\section{Discusión}
Según los resultados obtenidos, se infiere que en la información revisada no existe una metodología en ciencia de datos definida para el diagnóstico del cáncer de mama. En efecto, la mayoría de la literatura científica apunta directamente al uso de técnicas de ML y DL para el diagnóstico o pronostico del cáncer de mama exponiendo el nivel de precisión, cantidad de falsos positivos, gasto computacional y modelos algorítmicos utilizados para determinar el posible padecimiento de esta enfermedad. Y aunque estas investigaciones, brindan información valiosa para mejorar la precisión, sensibilidad y especificidad de las técnicas de ML y DL en el diagnóstico del cáncer, carecen de una metodología clara en donde la idea principal gire entorno de la comprensión del dominio y la toma de decisiones por parte de los oncólogos. En particular, la mayoría de investigaciones llegan a resultados en términos de precisión y exactitud, pero no profundizan en el valor real que el especialista en oncología atribuye a los datos para tomar una decisión y el impacto que dicha decisión tiene en la usabilidad del modelo generado, a sabiendas que el experto a través de su perspicacia medica es quien finalmente evalúa si los resultados obtenidos por los algoritmos son veraces y permiten diagnosticar el cáncer de forma ágil, generando un valor agregado que cumpla con los objetivos de las ciencias de la salud. Por consiguiente, aunque la comunidad de investigación en ciencia de datos este en crecimiento constante, esté explorando nuevos dominios, creando nuevos roles especializados y este realizando un gran esfuerzo de investigación para desarrollar análisis avanzados, mejorar modelos de datos y generar nuevos algoritmos apoyados de los campos de las matemáticas, la estadística y la informática, estas habilidades no son suficientes para la aplicación de la ciencia de datos en proyectos reales \citep{Martinez2021}, puesto que la mayoría de proyectos basados en datos presentan problemas organizativos y socio-técnicos, tales como: una falta de visión y claridad en los objetivos, un énfasis sesgado en cuestiones técnicas y ambigüedad de los roles. Dicho lo anterior, aunque las metodologías seleccionadas en la revisión sistemática no giren en torno al cáncer, proponen etapas que abarcan aspectos relevantes en la organización, a nivel de planteamiento y ejecución de un proyecto en ciencia de datos que tiene como eje el dominio basado en la prevención, el diagnostico y el tratamiento del cáncer de mama. Cabe resaltar que las metodologías CRISP-DM y KDD fueron seleccionadas debido a que comprenden todas las fases básicas que debe cumplir cualquier proyecto que tenga en su contexto el análisis de datos, ademas de que son bastante utilizadas por una cantidad considerable de investigadores, sin embargo aunque sean las metodologías más utilizadas, carecen de lineamientos que profundicen en la organización de los equipos de trabajo para llevar a cabo procesos de gestión que se alienen con el software y las metodologías de desarrollo ágiles. Ademas, autores como \citep{Martinez2021} sugieren que para ofrecer una solución integral para la ejecución exitosa de una metodología en ciencia de datos se deben cubrir estrictamente tres áreas: gestión de proyectos, gestión de equipos y gestión de datos e información. 
Por consiguiente, tras el esfuerzo por integrar los resultados analizados en este trabajo, parece bastante plausible afirmar que no existe una metodología puntual en ciencia de datos que se enfoque en el diagnóstico del cáncer de mama, sin embargo, diversos autores proponen metodologías que permiten la aplicación de esta ciencia en el dominio de las ciencias de la salud, teniendo como eje principal la comprensión de aspectos como: la comunicación constante con los interesados, el uso del enfoque ágil, la definición de roles y funciones, el valor agregado de los datos, la retroalimentación continua y la aceptación del experto en oncología para la posterior toma de decisiones que ayude a reducir de manera eficaz la morbilidad causada por esta enfermedad.
